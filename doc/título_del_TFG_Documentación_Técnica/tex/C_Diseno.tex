\apendice{Especificación de diseño}

\section{Introducción}

Este anexo recoge el diseño técnico del sistema \textit{TerapiTrack}, incluyendo la estructura de datos, la arquitectura general y los principales procedimientos implementados. El objetivo es documentar las decisiones de diseño que permiten garantizar la robustez, escalabilidad y mantenibilidad del sistema, así como mostrar la coherencia entre el modelo de datos, los casos de uso y la implementación realizada. 

\section{Diseño de datos}

El diseño de datos se ha realizado siguiendo un enfoque relacional, asegurando la integridad y normalización de la información mediante claves primarias, foráneas y restricciones de unicidad y dominio. A continuación se presentan los principales diagramas y el diccionario de datos que describen cómo se modelan usuarios, pacientes, profesionales, ejercicios, sesiones y evaluaciones dentro de la aplicación. 

\subsection{Diagrama entidad-relación}

El diagrama entidad-relación representa las entidades conceptuales del sistema (Usuario, Paciente, Profesional, Sesión, Ejercicio, Vídeo de respuesta, Evaluación, etc.) y las relaciones existentes entre ellas. En él se distinguen claramente las relaciones uno a uno (por ejemplo, Usuario–Paciente o Usuario–Profesional), las relaciones muchos a muchos resueltas mediante tablas intermedias (Paciente–Profesional, Sesión–Ejercicio) y las restricciones principales que garantizan la trazabilidad de cada sesión y de sus ejercicios asociados. Esta representación facilita la comprensión de cómo se vinculan los datos clínicos con los usuarios y las sesiones terapéuticas. 

\begin{figure}[H]
  \centering
  \includegraphics[width=\textwidth]{Diagrama Entidad-Relacion.png}
  \caption{Diagrama entidad-relación del sistema.}
\end{figure}

\subsection{Diagrama relacional}

El diagrama relacional muestra la traducción del modelo conceptual a tablas concretas de una base de datos relacional, indicando claves primarias, claves foráneas e índices relevantes. En él puede verse cómo se han introducido tablas puente para las relaciones muchos a muchos (como \texttt{Paciente\_Profesional} o \texttt{Ejercicio\_Sesion}) y cómo se asegura la integridad referencial entre sesiones, ejercicios, vídeos y evaluaciones. Asimismo, el diagrama refleja el uso de campos de estado, fechas y restricciones de dominio que permiten controlar el ciclo de vida de las sesiones y la expiración de los vídeos.

\begin{figure}[H]
  \centering
  \includegraphics[width=1.3\textwidth]{Diagrama Relacional.png}
  \caption{Diagrama relacional de la base de datos.}
\end{figure}


\subsection{Diccionario de datos}

El diccionario de datos detalla la estructura de cada tabla, sus campos, tipos de datos, claves y restricciones. A continuación se resumen las principales entidades del sistema:

% ---------- TABLA Usuario ----------
\begin{table}[H]
\small
\centering
\begin{tabular}{|p{2.6cm}|p{1.2cm}|p{1.3cm}|p{3.8cm}|p{5.3cm}|}
\hline
\textbf{Campo} & \textbf{Tipo} & \textbf{PK/FK} & \textbf{Restricciones} & \textbf{Descripción} \\
\hline
Id & integer & PK & autoincrement, not null & Identificador único del usuario \\
Nombre & text &  & not null & Nombre del usuario \\
Apellidos & text &  & not null & Apellidos del usuario \\
Email & text &  & unique, not null & Correo electrónico de acceso \\
Contrasena & text &  & not null & Contraseña cifrada \\
Rol\_Id & integer &  & not null, check (0,1,2) & Rol: 0=Admin, 1=Paciente, 2=Profesional \\
Fecha\_Registro & text &  & not null & Fecha de alta (YYYY-MM-DD) \\
Estado & integer &  & not null, check (0,1) & 0=Inactivo, 1=Activo \\
\hline
\end{tabular}
\caption{Diccionario de datos de la tabla \texttt{Usuario}}
\end{table}

% ---------- TABLA Paciente ----------
\begin{table}[H]
\small
\centering
\begin{tabular}{|p{3.1cm}|p{1.2cm}|p{1.3cm}|p{3.5cm}|p{4.5cm}|}
\hline
\textbf{Campo} & \textbf{Tipo} & \textbf{PK/FK} & \textbf{Restricciones} & \textbf{Descripción} \\
\hline
Usuario\_Id & integer & PK, FK & fk $\rightarrow$ Usuario(Id), not null & Identificador (FK \texttt{Usuario}) \\
Fecha\_Nacimiento & text &  & not null & Fecha de nacimiento (YYYY-MM-DD) \\
Condicion\_Medica & text &  &  & Condición médica principal \\
Notas & text &  &  & Observaciones adicionales \\
\hline
\end{tabular}
\caption{Diccionario de datos de la tabla \texttt{Paciente}}
\end{table}

% ---------- TABLA Profesional ----------
\begin{table}[H]
\small
\centering
\begin{tabular}{|p{2.8cm}|p{1.2cm}|p{1.3cm}|p{5.5cm}|p{3.2cm}|}
\hline
\textbf{Campo} & \textbf{Tipo} & \textbf{PK/FK} & \textbf{Restricciones} & \textbf{Descripción} \\
\hline
Usuario\_Id & integer & PK, FK & fk $\rightarrow$ Usuario(Id), not null & Identificador (FK \texttt{Usuario}) \\
Especialidad & text &  & not null & Especialidad del profesional \\
Tipo\_Profesional & text &  & check ('MEDICO', 'TERAPEUTA', 'ENFERMERO', 'PSICOLOGO'), not null & Tipo de profesional sanitario \\
\hline
\end{tabular}
\caption{Diccionario de datos de la tabla \texttt{Profesional}}
\end{table}

% ---------- TABLA Paciente_Profesional ----------
\begin{table}[H]
\small
\centering
\begin{tabular}{|p{2.9cm}|p{1.2cm}|p{1.3cm}|p{5.0cm}|p{3.2cm}|}
\hline
\textbf{Campo} & \textbf{Tipo} & \textbf{PK/FK} & \textbf{Restricciones} & \textbf{Descripción} \\
\hline
Paciente\_Id & integer & PK, FK & fk $\rightarrow$ Paciente(Usuario\_Id), not null & Identificador (FK \texttt{Paciente}) \\
Profesional\_Id & integer & PK, FK & fk $\rightarrow$ Profesional(Usuario\_Id), not null & Identificador (FK \texttt{Profesional}) \\
Fecha\_Asignacion & text &  & not null & Fecha de asignación (YYYY-MM-DD) \\
\hline
\end{tabular}
\caption{Diccionario de datos de la tabla \texttt{Paciente\_Profesional}}
\end{table}

% ---------- TABLA Ejercicio ----------
\begin{table}[H]
\small
\centering
\begin{tabular}{|p{1.9cm}|p{1.2cm}|p{1.3cm}|p{3.8cm}|p{5.1cm}|}
\hline
\textbf{Campo} & \textbf{Tipo} & \textbf{PK/FK} & \textbf{Restricciones} & \textbf{Descripción} \\
\hline
Id & integer & PK & autoincrement, not null & Identificador del ejercicio \\
Nombre & text &  & not null & Nombre del ejercicio \\
Descripcion & text &  & not null & Descripción del ejercicio \\
Tipo & text &  & not null & Tipo o categoría del ejercicio \\
Video & text &  & not null & Ruta al vídeo demostrativo \\
Duracion & integer &  & not null & Duración estimada (segundos) \\
\hline
\end{tabular}
\caption{Diccionario de datos de la tabla \texttt{Ejercicio}}
\end{table}

% ---------- TABLA Ejercicio_Profesional ----------
\begin{table}[H]
\small
\centering
\begin{tabular}{|p{2.0cm}|p{1.2cm}|p{1.3cm}|p{5.2cm}|p{3.1cm}|}
\hline
\textbf{Campo} & \textbf{Tipo} & \textbf{PK/FK} & \textbf{Restricciones} & \textbf{Descripción} \\
\hline
Usuario\_Id & integer & PK, FK & fk $\rightarrow$ Profesional(Usuario\_Id), not null & Identificador (FK \texttt{Profesional}) \\
Ejercicio\_Id & integer & PK, FK & fk $\rightarrow$ Ejercicio(Id), not null & Identificador (FK \texttt{Ejercicio}) \\
\hline
\end{tabular}
\caption{Diccionario de datos de la tabla \texttt{Ejercicio\_Profesional}}
\end{table}

% ---------- TABLA Sesion ----------
\begin{table}[H]
\small
\centering
\begin{tabular}{|p{3.1cm}|p{1.2cm}|p{1.3cm}|p{5.2cm}|p{4.4cm}|}
\hline
\textbf{Campo} & \textbf{Tipo} & \textbf{PK/FK} & \textbf{Restricciones} & \textbf{Descripción} \\
\hline
Id & integer & PK & autoincrement, not null & Identificador de la sesión \\
Paciente\_Id & integer & FK & fk $\rightarrow$ Paciente(Usuario\_Id), not null & Paciente al que se asigna la sesión \\
Profesional\_Id & integer & FK & fk $\rightarrow$ Profesional(Usuario\_Id), not null & Profesional responsable \\
Fecha\_Creacion & text &  & not null & Fecha de creación (YYYY-MM-DD) \\
Estado & text &  & check ('PENDIENTE', 'COMPLETADA', 'CANCELADA'), not null & Estado de la sesión \\
Fecha\_Programada & text &  & not null & Fecha programada (YYYY-MM-DD) \\
\hline
\end{tabular}
\caption{Diccionario de datos de la tabla \texttt{Sesion}}
\end{table}

% ---------- TABLA Ejercicio_Sesion ----------
\begin{table}[H]
\small
\centering
\begin{tabular}{|p{2.1cm}|p{1.2cm}|p{1.3cm}|p{4.7cm}|p{5.7cm}|}
\hline
\textbf{Campo} & \textbf{Tipo} & \textbf{PK/FK} & \textbf{Restricciones} & \textbf{Descripción} \\
\hline
Id & integer & PK & autoincrement, not null & Identificador del ejercicio en sesión \\
Sesion\_Id & integer & FK & fk $\rightarrow$ Sesion(Id), not null & FK a \texttt{Sesion} \\
Ejercicio\_Id & integer & FK & fk $\rightarrow$ Ejercicio(Id), not null & FK a \texttt{Ejercicio} \\
\hline
\end{tabular}
\caption{Diccionario de datos de la tabla \texttt{Ejercicio\_Sesion}}
\end{table}

% ---------- TABLA Video_Respuesta ----------
\begin{table}[H]
\small
\centering
\begin{tabular}{|p{3.7cm}|p{1.2cm}|p{1.3cm}|p{4.5cm}|p{4.2cm}|}
\hline
\textbf{Campo} & \textbf{Tipo} & \textbf{PK/FK} & \textbf{Restricciones} & \textbf{Descripción} \\
\hline
Ejercicio\_Sesion\_Id & integer & PK, FK & fk $\rightarrow$ Ejercicio\_Sesion(Id), not null & Identificador (FK \texttt{Ejercicio\_Sesion}) \\
Ruta\_Almacenamiento & text &  & not null & Ruta del archivo de vídeo \\
Fecha\_Expiracion & text &  &  & Fecha de expiración (YYYY-MM-DD) \\
\hline
\end{tabular}
\caption{Diccionario de datos de la tabla \texttt{Video\_Respuesta}}
\end{table}

% ---------- TABLA Evaluacion ----------
\begin{table}[H]
\small
\centering
\begin{tabular}{|p{3.3cm}|p{1.2cm}|p{1.3cm}|p{4.3cm}|p{4.0cm}|}
\hline
\textbf{Campo} & \textbf{Tipo} & \textbf{PK/FK} & \textbf{Restricciones} & \textbf{Descripción} \\
\hline
Ejercicio\_Sesion\_Id & integer & PK, FK & fk $\rightarrow$ Ejercicio\_Sesion(Id), not null & Identificador (FK \texttt{Ejercicio\_Sesion}) \\
Puntuacion & numeric &  & check (>=1 and <=5), not null & Puntuación (1–5) asignada por profesional \\
Comentarios & text &  &  & Observaciones de la evaluación \\
fechas\_Evaluacion & text &  & not null & Fecha de evaluación (YYYY-MM-DD) \\
\hline
\end{tabular}
\caption{Diccionario de datos de la tabla \texttt{Evaluacion}}
\end{table}

\textbf{Leyenda de simbología:}
\begin{itemize}
  \item PK: Clave primaria (Primary Key).
  \item FK: Clave foránea (Foreign Key).
  \item Unique: Valor único en la tabla.
  \item Check: Restricción de valores permitidos.
  \item Autoincrement: Incremento automático de la clave primaria.
  \item Not null: No puede ser un valor nulo.
\end{itemize}

\section{Diseño arquitectónico}

El sistema sigue una arquitectura modular basada en el patrón Modelo–Vista–Controlador (MVC), que facilita la separación de responsabilidades y la escalabilidad del desarrollo. En el caso de \textit{TerapiTrack}, esta arquitectura se implementa sobre Flask mediante módulos y \textit{blueprints} que agrupan funcionalidades por dominios (usuarios, sesiones, ejercicios, evaluaciones).

A nivel lógico, la arquitectura se organiza en tres capas principales:

\begin{itemize}
  \item \textbf{Capa de presentación}: Incluye las vistas de Flask y las plantillas Jinja que generan las páginas HTML. En ella se definen los formularios, los mensajes de validación y los componentes visuales construidos con Bootstrap y Bootswatch, diferenciando las interfaces de administrador, profesional y paciente.
  \item \textbf{Capa de negocio}: Implementada en los \textit{blueprints} de \texttt{src/controladores}, contiene la lógica de aplicación: gestión de usuarios y roles, asignación de pacientes a profesionales, planificación de sesiones, grabación y evaluación de ejercicios, así como las comprobaciones de permisos antes de cada operación.
  \item \textbf{Capa de datos}: Formada por los modelos SQLAlchemy de \texttt{src/modelos} y por la base de datos SQLite. Se encarga de mapear las entidades del diccionario de datos a tablas, aplicar las restricciones de integridad definidas en el diseño de datos y proporcionar métodos para consultas y actualizaciones transaccionales.
\end{itemize}

La estructura de carpetas del proyecto refleja esta organización:

\begin{itemize}
  \item \texttt{src/modelos}: Definición de entidades y relaciones (modelos SQLAlchemy) correspondientes a las tablas del diccionario de datos.
  \item \texttt{src/controladores}: Lógica de negocio, rutas y \textit{blueprints} de Flask, incluyendo la gestión de autenticación, permisos y validación de formularios.
  \item \texttt{src/vistas}: Plantillas HTML, ficheros estáticos (CSS, JS) y recursos gráficos empleados para construir la interfaz web.
  \item \texttt{src/tests}: Pruebas unitarias y de integración sobre modelos y controladores, que permiten comprobar la corrección de la lógica implementada.
\end{itemize}

De forma transversal a estas capas, el sistema incorpora varios servicios comunes. La autenticación y gestión de sesión de usuarios se resuelve mediante Flask‑Login, combinada con una tabla de \texttt{Usuario} que almacena contraseñas cifradas y un campo de rol que condiciona el acceso a cada sección de la aplicación. La validación de formularios se apoya en WTForms y en comprobaciones adicionales en los controladores para garantizar la coherencia de los datos antes de almacenarlos en la base de datos. Además, la aplicación está preparada para su despliegue en un entorno \textit{Platform as a Service} como Heroku, donde el servidor Flask se expone a través de un servidor WSGI y se configuran las credenciales y rutas de base de datos mediante variables de entorno.


\section{Diseño procedimental}

En esta sección se describen los principales flujos y procedimientos implementados, que se apoyan en la arquitectura anterior y en el modelo de datos descrito:

\begin{itemize}
  \item \textbf{Gestión de usuarios}: Incluye el alta, baja lógica, modificación y autenticación de usuarios, con control de roles y permisos. El flujo típico comienza con el administrador creando una cuenta, continúa con el acceso del usuario mediante correo y contraseña, y termina con la asignación de un panel específico según su rol.
  \item \textbf{Asignación de pacientes a profesionales}: El administrador establece relaciones muchos a muchos entre pacientes y profesionales mediante la tabla \texttt{Paciente\_Profesional}, permitiendo que un profesional gestione a varios pacientes y que un paciente pueda estar vinculado a diferentes perfiles sanitarios.
  \item \textbf{Gestión de ejercicios y sesiones}: Los profesionales crean ejercicios con su vídeo demostrativo y configuran sesiones terapéuticas combinando varios ejercicios. Posteriormente asignan estas sesiones a pacientes concretos, definiendo fechas programadas y controlando el estado de cada sesión (pendiente, completada o cancelada).
  \item \textbf{Grabación y evaluación}: Durante la realización de una sesión, el sistema registra la ejecución de cada ejercicio mediante la cámara del dispositivo y almacena el vídeo asociado. Más tarde, los profesionales revisan esos vídeos, asignan una puntuación y añaden comentarios, generando un histórico de evaluaciones que se utiliza para visualizar la evolución del paciente a lo largo del tiempo.
\end{itemize}

En una versión ampliada de este anexo podrían incorporarse diagramas de actividad para estos procedimientos, mostrando para cada caso el actor implicado (paciente, profesional o administrador), las llamadas a los controladores y las interacciones principales con los modelos y la base de datos.
