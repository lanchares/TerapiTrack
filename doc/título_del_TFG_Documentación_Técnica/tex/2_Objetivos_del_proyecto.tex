\capitulo{2}{Objetivos del proyecto}

Este capítulo expone de manera clara los propósitos principales del desarrollo de TerapiTrack, distinguiendo entre los objetivos funcionales ―enfocados en las necesidades detectadas para usuarios y profesionales― y los objetivos técnicos, que marcan las directrices para la construcción y despliegue de la plataforma. Esta diferenciación permite visualizar tanto la utilidad práctica del sistema como los aspectos tecnológicos que respaldan su funcionamiento.

\section{Objetivos funcionales}

\begin{itemize}
    \item Gestionar diferentes perfiles de usuario (administrador, paciente, personal sanitario: médico, terapeuta, psicólogo, enfermero) con permisos diferenciados.
    \item Permitir la vinculación de pacientes con los profesionales responsables de su seguimiento.
    \item Ofrecer una biblioteca de ejercicios y recursos terapéuticos en formato audiovisual, filtrable según necesidades clínicas.
    \item Facilitar la creación y programación de sesiones personalizadas de rehabilitación.
    \item Registrar y evaluar el progreso de cada paciente mediante grabaciones, informes y revisión de ejercicios.
    \item Impulsar la comunicación bidireccional y las notificaciones entre los usuarios y los profesionales, para un acompañamiento real y adaptado.
    \item Asegurar la trazabilidad de la evolución terapéutica, con un acceso a la información ajustado a la privacidad y rol de cada usuario.
\end{itemize}

\section{Objetivos técnicos}

\begin{itemize}
    \item Desarrollar la aplicación sobre un framework robusto (Flask) y modular, que facilite tanto el mantenimiento futuro como la ampliación de funcionalidades.
    \item Diseñar e implementar una base de datos relacional segura y eficiente, que garantice la integridad y la protección de la información sensible.
    \item Crear una interfaz accesible y sencilla, especialmente adaptada a personas con dificultades motoras y/o cognitivas.
    \item Aplicar mecanismos de cifrado y control de accesos para preservar la privacidad y la seguridad de los datos almacenados y transferidos.
    \item Implementar el despliegue de la aplicación en la nube (por ejemplo, a través de Heroku), asegurando así disponibilidad remota y alta accesibilidad.
    \item Desarrollar una batería de pruebas automatizadas que permita validar el correcto funcionamiento y la calidad del sistema.
\end{itemize}
