\capitulo{3}{Conceptos teóricos}
Este capítulo sintetiza los aspectos teóricos esenciales para entender la naturaleza y el desarrollo del proyecto TerapiTrack. Se abordan tanto conceptos del ámbito sanitario como fundamentos tecnológicos y de ingeniería del software aplicados durante la implementación.

\section{Telemedicina y rehabilitación remota}

La telemedicina hace referencia al uso de las tecnologías de la información y la comunicación para prestar servicios sanitarios a distancia. En el caso concreto de la rehabilitación, estas herramientas permiten realizar el seguimiento y diseño de terapias físicas o cognitivas sin que paciente y profesional coincidan en el mismo lugar, facilitando así la continuidad asistencial en enfermedades crónicas como el Parkinson y superando barreras geográficas.

\section{Accesibilidad y usabilidad en aplicaciones sanitarias}

Las aplicaciones orientadas a la salud deben contemplar criterios estrictos de accesibilidad y usabilidad para garantizar que cualquier paciente, independientemente de sus capacidades motrices o cognitivas, pueda manejar la plataforma con facilidad. Algunos principios fundamentales son:

\begin{itemize}
    \item \textbf{Perceptible}: Que todos los elementos y mensajes sean claros y apreciables por los usuarios.
    \item \textbf{Operable}: Los controles y formularios deben ser sencillos y adaptados para personas con limitaciones motoras.
    \item \textbf{Comprensible}: El lenguaje, la navegación y las acciones deben ser directas y sin ambigüedades.
    \item \textbf{Robusto}: El sistema ha de ser compatible con distintos dispositivos y tecnologías de apoyo.
\end{itemize}

\section{Arquitectura Modelo-Vista-Controlador (MVC)}

El patrón de diseño MVC es una estrategia habitual para el desarrollo de aplicaciones web, como TerapiTrack, y divide la lógica en tres partes diferenciadas:

\begin{itemize}
    \item \textbf{Modelo}: Encargado de la gestión y estructura de los datos (usuarios, pacientes, profesionales, ejercicios, sesiones, etc.).
    \item \textbf{Vista}: Responsable de mostrar la información utilizando una interfaz clara, jerárquica y accesible.
    \item \textbf{Controlador}: Interpreta las acciones del usuario, comunica modelo y vista, y coordina la lógica de la aplicación (ejemplo: iniciar sesión, asignar sesiones, evaluar ejercicios, etc.).
\end{itemize}

\section{Bases de datos relacionales y modelos de información}

El almacenamiento de información clínica o terapéutica requiere garantizar la integridad, trazabilidad y seguridad de los datos. Se utilizan bases de datos relacionales (en este caso, SQLite mediante SQLAlchemy) para organizar usuarios, pacientes, profesionales, ejercicios y sesiones. Destacan dos conceptos clave:

\subsection{Normalización}

El proceso de normalización evita la redundancia y asegura la integridad referencial. En TerapiTrack, las tablas principales (Usuario, Paciente, Profesional, Ejercicio, Sesión) mantienen relaciones claras y funcionales entre sí.

\subsection{Relaciones de muchos a muchos}

Para reflejar la realidad donde un paciente puede trabajar con varios profesionales, y un ejercicio puede asignarse a muchas sesiones, se utilizan tablas intermedias que permiten una asignación flexible y transparente.

\section{Tecnologías empleadas en el desarrollo del sistema}

En la construcción de TerapiTrack se han utilizado las siguientes tecnologías principales:

\begin{itemize}
    \item \textbf{Flask}: Framework web en Python para la gestión de rutas y lógica de servidor (backend).
    \item \textbf{SQLAlchemy}: Herramienta ORM que gestiona la interacción con bases de datos relacionales.
    \item \textbf{SQLite}: Motor de base de datos embebido, ideal para pruebas y despliegues sencillos.
    \item \textbf{Jinja}: Motor de plantillas para renderizado dinámico de vistas e interfaces HTML.
    \item \textbf{Bootswatch}: Colección de temas para Bootstrap que facilita una apariencia visual moderna y accesible.
    \item \textbf{Jira}: Plataforma de gestión de proyectos y seguimiento de tareas.
    \item \textbf{GitBash} y \textbf{GitHub}: Control del versionado de código y colaboración entre desarrolladores.
\end{itemize}

\section{Consideraciones de seguridad y privacidad}

Por tratarse de datos sensibles, el desarrollo tiene en cuenta aspectos de seguridad:

\begin{itemize}
    \item Cifrado de contraseñas.
    \item Control de acceso por roles (administrador, profesional, paciente).
    \item Políticas de retención y eliminación de vídeos terapéuticos.
    \item Uso de sesiones y validación para autenticación segura.
\end{itemize}

\section{Buenas prácticas y validación}

Se fomenta la calidad y mantenibilidad del software mediante:

\begin{itemize}
    \item Pruebas automatizadas para comprobar el funcionamiento de módulos clave.
    \item Documentación de código, manuales de usuario y de instalación.
    \item Uso de metodologías ágiles en la organización del trabajo (por ejemplo, con Jira).
\end{itemize}

\section{Ejemplos prácticos de uso de la plantilla \LaTeX{}}

A continuación se muestran algunos ejemplos de comandos útiles de la plantilla, siguiendo el formato recomendado por la documentación:

\subsection{Secciones y subsecciones}

Las secciones se incluyen con el comando:
\begin{verbatim}
\section{Nombre de sección}
\end{verbatim}
Las subsecciones se incluyen de forma análoga:
\begin{verbatim}
\subsection{Nombre de subsección}
\end{verbatim}

\subsubsection{Subsubsecciones}

Para mayor detalle se puede emplear el comando:
\begin{verbatim}
\subsubsection{Nombre de subsubsección}
\end{verbatim}

\section{Referencias}

Las referencias a artículos, webs o libros se incluyen con cite, como en el siguiente ejemplo: cite~\cite{wiki:latex}. Para citar varios recursos: ~\cite{koza92, bortolot2005}.

\section{Imágenes}

La plantilla dispone de comandos propios para incorporar imágenes, por ejemplo:

\imagen{escudoInfor}{Escudo de la Universidad de Burgos}{.5}

\section{Listas de items}

Existen diversas formas de listar:

\begin{itemize}
    \item Primer ítem de la lista.
    \item Segundo ítem relevante.
\end{itemize}

\begin{enumerate}
    \item Primer elemento.
    \item Segundo elemento.
\end{enumerate}

\begin{description}
    \item[Primer ítem] Descripción más detallada.
    \item[Segundo ítem] Segunda descripción relevante.
\end{description}

\section{Tablas}

Se pueden emplear comandos clásicos o los de la plantilla. Ejemplo:

\tablaSmall{Herramientas utilizadas en el proyecto}{l l l}
{herramientas_terapitrack}
{ \textbf{Herramienta} & \textbf{Ámbito} & \textbf{Descripción} \\}
{
Flask & Backend & Desarrollo de la lógica de servidor y API \\
SQLAlchemy & Base de datos & ORM para conexión y gestión relacional de datos \\
SQLite & Base de datos & Motor ligero y embebido para el almacenamiento de información \\
Jinja & Plantillas & Renderizado dinámico de las páginas HTML \\
Bootswatch & Frontend & Apariencia visual basada en Bootstrap \\
Jira & Gestión & Organización, planificación y seguimiento del proyecto \\
GitBash & Versionado & Control del código mediante terminal \\
GitHub & Repositorio & Colaboración y almacenamiento del proyecto \\
}

Este capítulo proporciona así el contexto teórico y tecnológico que sustenta toda la memoria y el desarrollo de la aplicación.
