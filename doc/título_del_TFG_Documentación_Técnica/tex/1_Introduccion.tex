\capitulo{1}{Introducción}
A medida que la población envejece y los servicios sanitarios se concentran principalmente en ciudades, muchas personas que residen en zonas rurales encuentran dificultades para acceder a terapias de rehabilitación. Esta situación afecta de manera especial a quienes padecen determinados problemas de salud crónicos, como la enfermedad del Parkinson. Ésta es una patología progresiva que provoca rigidez muscular, temblores y dificultad en el movimiento, y para la cual actualmente no existe cura. Sin embargo, mediante terapia, los pacientes pueden ralentizar el avance de la enfermedad, mejorar su movilidad y preservar durante más tiempo su independencia y calidad de vida. Las limitaciones de movilidad y la lejanía de los centros de referencia complican aún más el acceso, repercutiendo en el bienestar de los pacientes y en el esfuerzo de sus familias.

Ante esta problemática, surge \textbf{TerapiTrack}, una plataforma web diseñada para facilitar el acompañamiento y el seguimiento de terapias, principalmente para personas diagnosticadas con la enfermedad de Parkinson. Sin embargo, su enfoque flexible y modular permite también su uso en otros casos que requieran terapias a distancia, como personas con otras enfermedades crónicas. TerapiTrack busca eliminar barreras geográficas y de accesibilidad, adaptándose a las necesidades y capacidades de distintos tipos de pacientes.

\section{Contexto del problema}
La Enfermedad de Parkinson es una enfermedad neurodegenerativa que suele requerir una combinación de medicación y diversos ejercicios de fisioterapia o estimulación cognitiva, adaptados a cada fase y persona. En España, esta problemática se acentúa especialmente en la conocida “España Vaciada”, donde la mayoría de la población son personas de edad avanzada, grupo especialmente propenso a padecer este tipo de enfermedades. Para estos pacientes, acudir con regularidad a consultas y sesiones presenciales supone un reto considerable, no solo por la falta de servicios sanitarios especializados en muchas regiones rurales, sino también por la distancia a los centros de salud de referencia y las propias limitaciones físicas que provoca la enfermedad. Esta situación también se da en otros perfiles de pacientes que necesitan un seguimiento constante para evitar recaídas y para mantener su autonomía el mayor tiempo posible.

\section{Propuesta de solución}
El objetivo principal de TerapiTrack es acercar la rehabilitación y el control terapéutico al entorno cotidiano del paciente, apoyándose en la tecnología para romper barreras tradicionales. Las principales funcionalidades que aporta la herramienta son:

\begin{itemize}
    \item Permitir que los profesionales sanitarios creen ejercicios y rutinas personalizadas.
    \item Ofrecer la posibilidad de organizar sesiones y ajustar las actividades según la evolución de cada paciente.
    \item Facilitar el seguimiento remoto mediante la grabación y evaluación de los ejercicios realizados.
    \item Proporcionar una vía de comunicación sencilla y segura entre pacientes y especialistas, ayudando a resolver dudas y ajustar tratamientos sin desplazamientos.
    \item Garantizar una interfaz comprensible, accesible y adaptada a usuarios con dificultades motoras o cognitivas.
\end{itemize}

\section{Estructura de la memoria}
La presente memoria se divide en diferentes capítulos que buscan ofrecer una visión global del trabajo realizado:

\begin{itemize}
    \item \textbf{Capítulo 1. Introducción}: Expone el origen del proyecto y contextualiza el problema a resolver.
    \item \textbf{Capítulo 2. Objetivos del proyecto}: Presenta los objetivos perseguidos, tanto generales como específicos.
    \item \textbf{Capítulo 3. Conceptos teóricos}: Revisa los fundamentos técnicos y sanitarios necesarios para comprender la solución desarrollada.
    \item \textbf{Capítulo 4. Técnicas y herramientas}: Describe las tecnologías, herramientas y metodologías empleadas.
    \item \textbf{Capítulo 5. Aspectos relevantes del desarrollo}: Explica las fases principales del desarrollo y las decisiones adoptadas.
    \item \textbf{Capítulo 6. Trabajos relacionados}: Analiza otras soluciones similares y posiciona TerapiTrack respecto a ellas.
    \item \textbf{Capítulo 7. Conclusiones y líneas de trabajo futuras}: Resume los resultados alcanzados y plantea posibles mejoras o ampliaciones.
\end{itemize}

Junto a estos capítulos, se incluyen anexos donde se recopila documentación técnica, manuales y otra información complementaria.

\section{Materiales entregados}
Para facilitar la validación, el uso y la posible evolución del sistema, se entrega junto con la memoria un conjunto de materiales adicionales:

\begin{itemize}
    \item Código fuente completo del proyecto y su historial de versiones.
    \item Definición de la estructura de la base de datos y un conjunto de datos de prueba.
    \item Documentación técnica detallada en los anexos.
    \item Manual de usuario y de instalación.
    \item Batería de pruebas del sistema, con los resultados obtenidos.
\end{itemize}

Este conjunto de materiales pretende que cualquier persona interesada pueda comprender, utilizar y mejorar la solución presentada, contribuyendo así a un mejor acceso a la rehabilitación y al acompañamiento terapéutico a través de medios digitales.
