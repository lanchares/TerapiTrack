\apendice{Anexo de sostenibilización curricular}

\section{Introducción}

Este Trabajo Fin de Grado se ha desarrollado en el marco de una colaboración con el Servicio de Neurología del Hospital Universitario de Burgos, con el objetivo de facilitar el acceso a programas de rehabilitación a personas con enfermedad de Parkinson que residen en zonas rurales o alejadas de los centros de referencia. Esta orientación sitúa a \textit{TerapiTrack} en la intersección entre la transformación digital de los servicios sanitarios, la reducción de desigualdades en el acceso a la atención y la mejora de la calidad de vida de colectivos vulnerables, aspectos estrechamente relacionados con los principios de sostenibilidad social definidos por la CRUE y con los Objetivos de Desarrollo Sostenible, especialmente el ODS 3 (Salud y bienestar) y el ODS 10 (Reducción de las desigualdades).

A partir de estas directrices, en este anexo se reflexiona sobre cómo el proyecto ha contribuido al desarrollo de competencias transversales en sostenibilidad, tanto en la contextualización crítica del problema como en la utilización responsable de los recursos tecnológicos, la participación en iniciativas comunitarias y la incorporación de consideraciones éticas en el diseño e implementación del sistema.

\section{Contextualización crítica del problema (SOS1)}

Durante la fase de análisis, el trabajo ha exigido entender la telerehabilitación no sólo como un reto técnico, sino como una respuesta a una problemática social concreta, especialmente en el caso de personas mayores que viven en la <<España vaciada>> y encuentran dificultades para acceder de forma regular a terapias de rehabilitación especializadas.
El diseño de \textit{TerapiTrack} parte de esa realidad y busca reducir la brecha entre el entorno rural y los servicios de referencia, permitiendo que parte del tratamiento se desarrolle en el domicilio del paciente sin perder el seguimiento por parte del profesional sanitario.

Esta reflexión ha ayudado a contextualizar el uso de tecnologías web, vídeo y telemedicina desde una perspectiva más amplia, en la que el éxito del proyecto no se mide únicamente por el funcionamiento del software, sino por su capacidad para mejorar la continuidad asistencial, favorecer la autonomía de los pacientes y aliviar la carga de desplazamientos y organización que recae sobre las familias.
De esta manera, el proyecto contribuye a desarrollar la competencia de contextualización crítica del conocimiento, al establecer vínculos explícitos entre la solución técnica y la problemática social, económica y territorial en la que se aplica.

\section{Uso responsable de recursos y reducción de impactos (SOS2)}

En el diseño de la solución se han tomado decisiones orientadas a aprovechar al máximo los recursos existentes y minimizar la necesidad de infraestructura adicional. El sistema se ejecuta en un navegador web estándar y funciona sobre equipos de sobremesa o portátiles ya presentes en muchos hogares o centros sanitarios, de modo que no requiere dispositivos específicos ni hardware de alto coste. La compatibilidad con un mando USB tipo SNES se ha planteado precisamente para reutilizar periféricos sencillos y robustos, evitando depender de hardware propietario difícil de mantener o sustituir a largo plazo.

La arquitectura técnica también busca un uso eficiente de los recursos de cómputo y almacenamiento. El proyecto se ha desplegado en plataformas en la nube que permiten escalar gradualmente en función del número de usuarios, evitando infraestructuras sobredimensionadas para fases iniciales o pilotos.
Además, se han incorporado políticas de retención de vídeos configurables por el administrador, de forma que los recursos multimedia sólo se conservan el tiempo necesario para la supervisión y evaluación terapéutica, reduciendo el volumen de datos almacenados y, por extensión, el consumo asociado de almacenamiento y copias de seguridad.

\section{Participación y colaboración en procesos comunitarios (SOS3)}

Aunque se trata de un proyecto individual de fin de grado, \textit{TerapiTrack} se inserta en una línea de trabajo más amplia en colaboración con un servicio hospitalario y un grupo de investigación universitario. El desarrollo de la herramienta se ha apoyado en reuniones periódicas con profesionales sanitarios para validar requisitos y flujos de trabajo, así como en la revisión de iniciativas similares en telerehabilitación y fisioterapia digital.

Esta colaboración ha permitido comprender mejor las necesidades reales de los distintos actores implicados (pacientes, familiares, terapeutas, neurólogos) y ha fomentado una actitud de escucha activa y adaptación del diseño a las condiciones del entorno asistencial. Desde el punto de vista de la sostenibilidad, esta experiencia refuerza la importancia de trabajar en equipos multidisciplinares y de considerar la tecnología como un medio al servicio de procesos comunitarios ya existentes, y no como un fin en sí mismo.

\section{Dimensión ética y equidad en el acceso (SOS4)}

La sostenibilidad también tiene una dimensión ética, ligada a la protección de datos personales, al respeto a la autonomía de los pacientes y a la equidad en el acceso a los servicios. En el desarrollo de \textit{TerapiTrack} se ha prestado especial atención a la gestión de información sensible, incorporando mecanismos de autenticación, control de acceso por roles y cifrado de contraseñas, así como opciones de configuración que permiten limitar la duración del almacenamiento de vídeos y evaluaciones.

Asimismo, el diseño de la interfaz y el soporte para el mando SNES buscan que las limitaciones motoras o la poca experiencia tecnológica no se conviertan en una barrera de entrada. Esta preocupación por la accesibilidad se alinea con la idea de que las soluciones tecnológicas deben reducir desigualdades y no generarlas, ofreciendo alternativas de uso adaptadas a diferentes capacidades y contextos.
En este sentido, el proyecto ha servido para tomar conciencia de que la calidad del software en el ámbito sanitario no puede desligarse de sus implicaciones éticas y sociales, y que cualquier mejora técnica debe evaluarse también desde el punto de vista de la dignidad, la privacidad y la autonomía de las personas usuarias.

\section{Aprendizajes personales en sostenibilidad}

Desde una perspectiva personal, el desarrollo de este TFG ha permitido integrar los conocimientos técnicos adquiridos a lo largo del grado con una visión más amplia de la responsabilidad social del ingeniero informático. Trabajar en una solución orientada a la telerehabilitación ha puesto de manifiesto que las decisiones sobre arquitectura, diseño de interfaz o modelos de datos tienen consecuencias directas sobre la inclusión, la accesibilidad y la equidad en el acceso a la atención sanitaria.

Esta experiencia ha reforzado la importancia de incorporar criterios de sostenibilidad y de reflexión ética desde las fases iniciales de cualquier proyecto tecnológico, especialmente cuando afecta a colectivos vulnerables o a servicios públicos esenciales. En futuros desarrollos profesionales, la intención es seguir aplicando estas competencias, buscando soluciones que no sólo sean técnicamente correctas, sino también socialmente responsables, eficientes en el uso de recursos y respetuosas con los valores de la sostenibilidad definidos en las directrices de la CRUE.
