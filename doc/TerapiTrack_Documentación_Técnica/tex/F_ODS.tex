\apendice{Anexo de sostenibilización curricular}

\section{Introducción}

Este Trabajo Fin de Grado se ha desarrollado en colaboración con el Servicio de Neurología del Hospital Universitario de Burgos con el objetivo de facilitar la rehabilitación a personas con enfermedad de Parkinson en zonas rurales.
Esta orientación sitúa a TerapiTrack en la intersección entre la transformación digital sanitaria, la reducción de desigualdades y la mejora de la calidad de vida de colectivos vulnerables,
alineándose con los principios de sostenibilidad social de la CRUE y los Objetivos de Desarrollo Sostenible, especialmente el ODS 3 (Salud y bienestar) y el ODS 10 (Reducción de las desigualdades).

A continuación, se reflexiona sobre la contribución del proyecto a las competencias transversales en sostenibilidad: contextualización crítica, uso responsable de recursos, participación comunitaria y ética.

\section{Contextualización crítica del problema (SOS1)}

El trabajo aborda la telerehabilitación no solo como un reto técnico, sino como respuesta a una problemática social concreta en la <<España vaciada>>, donde existen barreras para acceder a terapias especializadas.
El diseño de TerapiTrack parte de esta realidad para reducir la brecha territorial, permitiendo el tratamiento domiciliario sin perder el seguimiento profesional.

Esta perspectiva implica que el éxito del software no se mide solo por su funcionamiento, sino por su capacidad para mejorar la continuidad asistencial y favorecer la autonomía del paciente.
Así, el proyecto conecta explícitamente la solución técnica con la problemática social y territorial, desarrollando la competencia de contextualización crítica del conocimiento.

\section{Uso responsable de recursos y reducción de impactos (SOS2)}

El diseño ha priorizado la eficiencia para minimizar la infraestructura necesaria. El sistema funciona en navegadores estándar sobre equipos ya existentes en los hogares, sin requerir hardware costoso.
La compatibilidad con mandos USB tipo SNES fomenta la reutilización de periféricos sencillos y robustos, evitando la dependencia de hardware propietario.

A nivel de arquitectura, el despliegue en la nube permite escalar los recursos de cómputo según la demanda real, evitando el sobredimensionamiento inicial.
Asimismo, las políticas configurables de retención de vídeos aseguran que los datos multimedia solo se conserven el tiempo útil para la terapia, reduciendo el consumo de almacenamiento y energía.

Además de la eficiencia tecnológica, el impacto ambiental más directo reside en la reducción de la huella de carbono asociada al transporte.
Al habilitar la rehabilitación domiciliaria, se eliminan desplazamientos recurrentes en vehículo privado o ambulancia desde zonas rurales al hospital.
Esto conlleva un ahorro significativo de combustible y una reducción directa de emisiones de CO2, alineándose con las estrategias de descarbonización.

\section{Dimensión ética y equidad en el acceso (SOS4)}

La sostenibilidad incluye una dimensión ética ligada a la protección de datos y la equidad.
Se ha prestado especial atención a la seguridad de la información sensible (cifrado, roles, autenticación) y a la privacidad en la gestión de los vídeos.

Asimismo, la interfaz accesible y el soporte para mandos simplificados buscan eliminar barreras de entrada por brecha digital o limitaciones motoras.
La tecnología debe reducir desigualdades, no generarlas. El proyecto evidencia que la calidad del software sanitario implica evaluar sus repercusiones éticas, priorizando siempre la dignidad y autonomía de las personas usuarias.

\section{Aprendizajes personales en sostenibilidad}

El desarrollo de este TFG ha permitido integrar conocimientos técnicos con la responsabilidad social de la ingeniería.
He comprobado cómo las decisiones de arquitectura e interfaz impactan directamente en la inclusión y la equidad del acceso sanitario.

Esta experiencia confirma la necesidad de incorporar criterios de sostenibilidad y ética desde el inicio de cualquier proyecto.
En mi futuro profesional, mantendré el compromiso de buscar soluciones técnicamente robustas, pero también socialmente responsables y eficientes en el uso de recursos.