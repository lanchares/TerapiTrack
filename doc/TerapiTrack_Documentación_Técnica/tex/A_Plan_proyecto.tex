\apendice{Plan de Proyecto Software}
\section{Introducción}

Este anexo describe el plan de proyecto seguido para el desarrollo de TerapiTrack, incluyendo la planificación temporal del trabajo y el estudio de viabilidad correspondiente. El propósito es mostrar de forma ordenada cómo se ha organizado el proceso de desarrollo para alcanzar los objetivos planteados y justificar que el proyecto resulta asumible desde el punto de vista económico y adecuado en términos legales, teniendo en cuenta que se tratan datos de carácter sanitario.

\section{Planificación temporal}
El desarrollo de \textit{TerapiTrack} se ha organizado siguiendo una metodología ágil inspirada en Scrum, utilizando Jira para planificar y hacer el seguimiento del trabajo. La planificación inicial se estructuró en sprints, pero el avance real se fue ajustando a medida que aparecían imprevistos técnicos y se concretaban mejor los requisitos. Por este motivo, el tablero de Jira refleja tanto la planificación prevista como los cambios introducidos durante el proyecto.

Se ha dado prioridad a completar primero los módulos y tareas considerados esenciales, procurando mantener una entrega de funcionalidad lo más continua posible. En las páginas siguientes se describen los sprints realizados y se incluyen los gráficos de evolución extraídos de Jira, donde puede verse la progresión del trabajo en cada iteración.

Aunque se ha utilizado la estructura de sprints para organizar el proyecto, la ejecución no ha seguido de forma estricta el calendario inicial. En varios casos la mayor parte del trabajo se concentró al final del sprint y algunos se cerraron más tarde de la fecha prevista. Esto se debe a que el desarrollo se ha realizado de manera individual, lo que ha obligado a reorganizar el tiempo disponible y a reagrupar tareas. Aun así, la planificación por sprints y el uso de Jira han resultado útiles para mantener una visión global del avance y de las prioridades en cada momento, por lo que los gráficos deben interpretarse como una guía del trabajo realizado más que como una aplicación estricta de Scrum.

\subsubsection{Sprint 1: Configuración inicial (11--17 marzo 2025)}

Este primer sprint se dedicó a poner en marcha el entorno de trabajo y a conocer las tecnologías básicas del proyecto. Se configuró el entorno con Flask, se creó el repositorio en GitHub y se definió una estructura inicial de carpetas y dependencias.\cite{flask_doc} Paralelamente se revisó la documentación de Flask y Jinja y se realizaron las primeras pruebas de conexión con SQLite para comprobar la persistencia de datos.\cite{flask_doc,jinja_doc_es}

Quedaron pendientes tareas como el diseño completo de la base de datos y la documentación detallada de requisitos, que se trasladaron al sprint siguiente. El gráfico burndown muestra que la mayor parte del trabajo se concentró al final del periodo, algo lógico en esta fase inicial de familiarización con las herramientas.

\begin{figure}[H]
  \centering
  \includegraphics[width=0.9\textwidth]{img/sprint_1.png}
  \caption{Gráfico burndown del Sprint 1 obtenido de Jira.}
\end{figure}

\subsubsection{Sprint 2: Estudio (18--24 marzo 2025)}

En el segundo sprint el esfuerzo se centró en profundizar en las tecnologías clave y en definir con más detalle el modelo de datos. Se completó el estudio práctico de Flask, SQLAlchemy y Jinja mediante ejemplos y pequeñas pruebas, se diseñó la base de datos y se elaboró el primer modelo entidad--relación.\cite{flask_doc,sqlalchemy_doc,jinja_doc_es} También se validó la integración entre Flask y SQLite con pruebas de lectura y escritura, apoyándose en herramientas como DB Browser for SQLite para inspeccionar la base de datos.\cite{dbbrowser_sqlite}

Algunas tareas de implementación de modelos en SQLAlchemy no llegaron a completarse y se arrastraron al sprint 3, donde se terminó de consolidar la capa de datos. El burndown refleja que las finalizaciones se concentran al final del sprint, más como registro de lo realizado que como una ejecución estricta del plan diario.

\begin{figure}[H]
  \centering
  \includegraphics[width=0.9\textwidth]{img/sprint_2.png}
  \caption{Gráfico burndown del Sprint 2 obtenido de Jira.}
\end{figure}

\subsubsection{Sprint 3: Diseño (25--31 marzo 2025)}

El tercer sprint se orientó al diseño detallado del sistema y a la implementación de los primeros modelos completos. Se ampliaron y ajustaron las tablas de la base de datos, se implementaron los modelos principales (usuario, paciente, profesional) en SQLAlchemy y se comprobó la estabilidad de la integración con SQLite.\cite{sqlalchemy_doc} Además, se avanzó en el diseño de las plantillas de Jinja y se revisó la documentación de requisitos para alinearla con las decisiones tomadas.\cite{jinja_doc_es}

La creación de un sistema de permisos basado en Flask-Login quedó únicamente esbozada en este sprint y se abordó con más profundidad en los siguientes.\cite{flask_login} El gráfico burndown muestra cómo gran parte de los puntos de historia se cierran hacia el final, cuando se integran y prueban los distintos elementos de diseño.

\begin{figure}[H]
  \centering
  \includegraphics[width=0.9\textwidth]{img/sprint_3.png}
  \caption{Gráfico burndown del Sprint 3 obtenido de Jira.}
\end{figure}

\subsubsection{Sprint 4: Base de datos (2--7 abril 2025)}

En el sprint 4 se consolidó el trabajo previo sobre la base de datos y se empezó a reforzar la capa de seguridad y pruebas. Se completó la definición del modelo relacional, se ajustaron las relaciones entre entidades y se mejoró la integración con SQLAlchemy.\cite{sqlalchemy_doc} También se avanzó en la definición de permisos por rol y se continuó actualizando la documentación funcional y técnica.

Las pruebas unitarias sobre los modelos y la redacción sistemática de la memoria en \LaTeX{} comenzaron en este sprint pero se fueron extendiendo a los posteriores.

\begin{figure}[H]
  \centering
  \includegraphics[width=0.9\textwidth]{img/sprint_4.png}
  \caption{Gráfico burndown del Sprint 4 obtenido de Jira.}
\end{figure}

\subsubsection{Sprint 5: Diseño de la interfaz (8--13 abril 2025)}

El quinto sprint se centró en la parte visible de la aplicación. Se diseñó la interfaz de administración de usuarios, se trabajó en los formularios y en la validación básica de datos, y se fueron integrando estos elementos con la capa de modelos existente.\cite{flask_doc,flask_login} También se revisaron los requisitos funcionales para asegurarse de que la interfaz cubría los flujos principales de gestión.

Algunas tareas de pruebas unitarias y de depuración más exhaustiva quedaron pendientes y se trasladaron a sprints posteriores.

\begin{figure}[H]
  \centering
  \includegraphics[width=0.9\textwidth]{img/sprint_5.png}
  \caption{Gráfico burndown del Sprint 5 obtenido de Jira.}
\end{figure}

\subsubsection{Sprint 6: Base funcional (3--9 junio 2025)}

Tras una pausa en el calendario académico, el sprint 6 se dedicó a dotar al sistema de la base funcional completa. Se implementaron los modelos relacionados con ejercicios, sesiones, evaluaciones y la relación paciente--profesional, y se definieron los flujos principales de trabajo (asignación de ejercicios, registro de sesiones, evaluación, etc.). Además, se incorporó el cifrado de contraseñas con \textit{bcrypt} y se reforzó la autenticación de usuarios.\cite{flask_login}

Quedaron para más adelante algunos aspectos de la interfaz de administración y parte de las pruebas automáticas. El burndown de este sprint muestra un incremento importante de trabajo completado en pocos días, reflejando que muchas tareas se habían preparado en los sprints anteriores y se integraron en este bloque.

\begin{figure}[H]
  \centering
  \includegraphics[width=0.9\textwidth]{img/sprint_6.png}
  \caption{Gráfico burndown del Sprint 6 obtenido de Jira.}
\end{figure}

\subsubsection{Sprint 7: Base funcional 2 (17--22 junio 2025)}

En el sprint 7 se completaron y pulieron muchas de las funcionalidades introducidas en el sprint anterior. Se reforzó la seguridad (protección CSRF, gestión avanzada de sesiones), se implementaron rutas CRUD para usuarios, sesiones, ejercicios y evaluaciones, y se añadieron características clave como la subida de vídeos por parte de los profesionales y la generación de diagramas UML de apoyo.\cite{flask_login,flask_blueprints_doc} También se avanzó en la documentación de casos de uso y en la preparación de material para la memoria.

Algunas pruebas unitarias siguieron abiertas y se terminaron de consolidar en la fase final del proyecto.

\begin{figure}[H]
  \centering
  \includegraphics[width=0.9\textwidth]{img/sprint_7.png}
  \caption{Gráfico burndown del Sprint 7 obtenido de Jira.}
\end{figure}

\subsubsection{Sprint 8: Controladores y vistas (23--29 junio 2025)}

En este sprint se continuó cerrando la integración entre controladores y vistas y se añadieron funcionalidades de apoyo a la explotación diaria de la herramienta. Se completaron los paneles de gestión para profesionales y pacientes, se refinaron los decoradores de autorización, se incorporó lógica de notificaciones y se implementó el borrado automático de vídeos pasados ciertos plazos para controlar el espacio de almacenamiento.\cite{flask_blueprints_doc,realpython_blueprints} Paralelamente se realizaron ajustes en los formularios y en la validación de datos en cliente y servidor.

Durante este periodo también se avanzó en tareas de documentación técnica y en la elaboración del diccionario de datos, aunque parte de este trabajo se siguió afinando después fuera del marco de los sprints. El burndown muestra una evolución casi plana hasta las fechas cercanas al cierre, en las que se registran la mayoría de los puntos completados, reflejando la concentración de integración y pruebas en los últimos días.

\begin{figure}[H]
  \centering
  \includegraphics[width=0.9\textwidth]{img/sprint_8.png}
  \caption{Gráfico burndown del Sprint 8 obtenido de Jira.}
\end{figure}

\subsubsection{Sprint 9: Pruebas iniciales y refactorización (17--25 diciembre 2025)}

El sprint 9 marcó el inicio de la fase de pruebas intensivas. Se preparó la batería de tests automatizados, se refactorizaron algunos controladores para simplificar su mantenimiento y se revisaron los modelos de datos para asegurar su coherencia con los casos de uso definitivos.\cite{GithubTerapitrack} También se corrigieron pequeños errores detectados durante las primeras ejecuciones completas de la aplicación.

El gráfico burndown muestra un sprint corto en el que apenas se registran incidencias resueltas hasta los últimos días. Esto se debe a que buena parte del trabajo consistió en tareas de limpieza y reestructuración interna, que se cerraron en bloque una vez verificado que el sistema seguía funcionando correctamente.

\begin{figure}[H]
  \centering
  \includegraphics[width=0.9\textwidth]{img/sprint_9.png}
  \caption{Gráfico burndown del Sprint 9 centrado en pruebas iniciales y refactorización.}
\end{figure}

\subsubsection{Sprint 10: Cierre de incidencias (27--31 diciembre 2025)}

En el sprint 10 el objetivo principal fue cerrar incidencias acumuladas de sprints anteriores. Se revisaron formularios y mensajes de error, se ajustaron algunas rutas de Flask, se pulieron detalles de la interfaz y se solucionaron problemas menores relacionados con validaciones y permisos.\cite{GithubTerapitrack} Además, se aprovecharon estos cambios para mejorar la experiencia de uso en los flujos más habituales de profesionales y pacientes.

El burndown refleja una línea de trabajo planificada estable y una curva de progreso real que sube en escalones pronunciados al final del sprint. Cada escalón corresponde al cierre de grupos de incidencias que se habían ido registrando durante las pruebas de uso reales.

\begin{figure}[H]
  \centering
  \includegraphics[width=0.9\textwidth]{img/sprint_10.png}
  \caption{Gráfico burndown del Sprint 10 con cierre de incidencias acumuladas.}
\end{figure}

\subsubsection{Sprint 11: Pruebas finales y documentación (1--6 enero 2026)}

El sprint 11 se dedicó casi por completo a las pruebas finales y a la documentación. Se amplió la batería de tests hasta alcanzar una cobertura cercana al 99\,\%, se ejecutaron pruebas manuales completas sobre los flujos críticos y se revisaron los anexos técnicos, incluyendo la documentación de instalación, despliegue y pruebas.\cite{GithubTerapitrack} También se realizaron pequeños ajustes en la interfaz basados en estas pruebas finales.

En el gráfico burndown se observa cómo se planifica un número elevado de incidencias y el progreso real avanza en grandes escalones. Esto refleja que muchas tareas correspondían a bloques de testing o redacción que sólo podían cerrarse cuando se completaba el conjunto de cambios asociado.

\begin{figure}[H]
  \centering
  \includegraphics[width=0.9\textwidth]{img/sprint_11.png}
  \caption{Gráfico burndown del Sprint 11 dedicado a pruebas finales y documentación.}
\end{figure}

\subsubsection{Sprint 12: Cierre y pulido (1--6 enero 2026)}

%Falta añadir

\section{Estudio de viabilidad}

Antes de abordar los aspectos economicos y legales, en este apartado se valora si el proyecto \textit{TerapiTrack} seria viable en un escenario real, suponiendo que su desarrollo y despliegue se encargan a una empresa de software a partir de los requisitos y del prototipo implementado en este Trabajo Fin de Grado.

\subsection{Viabilidad económica}

La viabilidad económica de \textit{TerapiTrack} se ha analizado suponiendo que una pequeña empresa de desarrollo recibe el encargo de construir una solución similar a la implementada en este TFG. El objetivo sería disponer de una primera versión funcional para un piloto con un número reducido de pacientes, desplegada en la nube y con costes de infraestructura ajustados.

Tomando como referencia la planificación temporal del proyecto y el alcance funcional conseguido, puede estimarse que el desarrollo de esta primera versión requeriría en torno a seis meses de trabajo de un desarrollador junior, con un esfuerzo aproximado de 400 horas efectivas. Esta estimación incluye el análisis de requisitos, el diseño de modelos y controladores, la implementación de vistas y formularios, la integración con la base de datos, la lógica asociada al mando y la configuración de pruebas automatizadas y despliegue. Considerando un coste medio de 24\,€/h para este tipo de perfil, el coste de personal se situaría en torno a 9\,600\,€.\cite{homiaDesarrolladorWeb}

En cuanto a recursos materiales para el desarrollo, la empresa necesitaría al menos un equipo de trabajo estándar (ordenador portátil o de sobremesa) valorado en torno a 1\,000\,€. Suponiendo una amortización de tres años y que el proyecto ocupa aproximadamente medio año de dedicación, el coste imputable de hardware sería del orden de 170\,€.

Para el entorno piloto se supone que los pacientes disponen de un ordenador con navegador y cámara en su domicilio, de modo que no es necesario adquirir equipos específicos. Sí sería necesario proporcionar un mando USB tipo SNES a cada paciente. En el mercado existen packs de dos mandos por unos 12,95\,€, lo que equivale a unos 6,50\,€ por dispositivo.\cite{omniretroMandosSNES} Para un piloto con 16 pacientes bastaría con adquirir 16 mandos, con un coste aproximado de 104\,€.

En lo relativo a infraestructura, se asume un despliegue en Heroku utilizando un dyno de tipo \emph{Basic} (antiguo plan \emph{Hobby}) y una base de datos Heroku Postgres en un plan de entrada. El precio de referencia de un dyno Basic es de unos 7\,US\$ al mes\footnote{En torno a 6--7\,€ mensuales según el tipo de cambio.}, mientras que el plan Mini de Heroku Postgres parte de 5\,US\$ al mes.\cite{HerokuDocs,herokuPostgresMini} Considerando seis meses de proyecto y una fase piloto limitada, el coste conjunto de aplicación y base de datos en Heroku se situaría en el entorno de 70--80\,€.

Para el almacenamiento de vídeos se ha optado por Cloudinary, que ofrece un plan gratuito con un cupo mensual de créditos suficiente para proyectos pequeños si se controla el tamaño y la duración de los vídeos.\cite{CloudinaryFreePlan} En un escenario de piloto con pocos pacientes y políticas de borrado periódico, es razonable suponer que no sería necesario contratar un plan de pago, por lo que el coste de Cloudinary se puede considerar nulo en esta primera fase.

Además de la infraestructura básica, habría que contemplar un pequeño presupuesto para dominio, certificados y servicios auxiliares (copias de seguridad adicionales, monitorización, etc.). Para una primera versión limitada, un margen de 100\,€ permite cubrir estos conceptos sin necesidad de soluciones complejas.

La tabla~\ref{tab:costes_terapitrack} resume de forma orientativa los principales conceptos de coste considerados para una primera versión profesional de \textit{TerapiTrack} en un horizonte de seis meses y un piloto con dieciséis pacientes:

\begin{table}[H]
\centering
\small
\begin{tabular}{p{0.58\textwidth} r}
\hline
Concepto & Coste estimado \\
\hline
Coste de personal (400 h a 24 €/h) & 9\,600 € \\
Coste imputable de hardware de desarrollo & 170 € \\
Mandos SNES USB para 16 pacientes & 104 € \\
Dyno Basic de Heroku (6 meses) & 40–45 € \\
Base de datos Heroku Postgres Mini (6 meses) & 30–35 € \\
Dominio, copias de seguridad y servicios auxiliares & 100 € \\
\hline
Coste total aproximado primera versión & 10\,000–10\,100 € \\
\hline
\end{tabular}
\caption{Resumen orientativo de costes de la primera versión de \textit{TerapiTrack} en un escenario de seis meses de proyecto y piloto con dieciséis pacientes.}
\label{tab:costes_terapitrack}
\end{table}

En conjunto, combinando el coste de personal, la amortización de hardware, la provisión de mandos para los pacientes y un presupuesto realista de infraestructura en la nube, el coste total de una primera versión desplegable de \textit{TerapiTrack} se situaría en torno a 10\,000\,€. Esta cifra no incluye servicios posteriores de soporte continuado, mantenimiento evolutivo o integración con otros sistemas clínicos, que requerirían un contrato específico adicional.

\subsection{Viabilidad legal}

El desarrollo de \textit{TerapiTrack} se ha llevado a cabo utilizando de forma casi exclusiva software libre y de código abierto, lo que ha permitido reducir significativamente los costes de licenciamiento y facilita tanto la explotación académica como un eventual despliegue profesional sin restricciones legales importantes.

El backend está construido sobre Python y el microframework Flask para la gestión de rutas y controladores, mientras que la persistencia de datos se resuelve con SQLAlchemy sobre SQLite.\cite{flask_doc,sqlalchemy_doc} Estas tecnologías fundamentales se distribuyen bajo licencias permisivas de tipo BSD o MIT, que autorizan el uso, modificación y redistribución del software siempre que se mantengan los avisos de derechos de autor y las notas de licencia de los autores originales.

Flask se apoya en varias bibliotecas del ecosistema Pallets que se utilizan de forma indirecta a través del framework. Werkzeug proporciona la infraestructura WSGI de bajo nivel para gestionar peticiones y respuestas HTTP, cookies y depuración. Jinja2 actúa como motor de plantillas para la generación de vistas HTML. Click gestiona la interfaz de línea de comandos, ItsDangerous se encarga de la firma segura de tokens, y MarkupSafe proporciona funciones de escape para prevenir inyecciones de código.\cite{flask_doc,jinja_doc_es} Todas estas dependencias se distribuyen bajo licencias permisivas, por lo que su presencia en el proyecto no introduce restricciones adicionales.

Para el control de acceso de usuarios se emplean Flask-Login para la gestión de sesiones, Flask-WTF y WTForms para la validación de formularios, y bcrypt junto con Flask-Bcrypt para el cifrado robusto de contraseñas.\cite{flask_login} Estas herramientas se publican bajo licencias MIT o BSD, permitiendo su integración en proyectos propios sin obligación de publicar el código fuente, siempre que se conserven las atribuciones exigidas.

La interacción con el mando tipo SNES se resuelve mediante Pygame, una biblioteca de código abierto basada en SDL distribuida bajo licencia LGPL. Esta licencia permite el uso en aplicaciones propietarias siempre que la biblioteca se enlace de forma dinámica. En el caso de \textit{TerapiTrack}, Pygame se utiliza sin modificar su código interno, únicamente para gestionar la comunicación con el mando, por lo que su integración es totalmente compatible con un despliegue profesional.

Además de las dependencias listadas, el proyecto utiliza herramientas de desarrollo como Visual Studio Code, Git, GitHub y pytest, todas ellas accesibles bajo modalidades gratuitas o de código abierto sin coste de licencia. El análisis conjunto de todas las dependencias y herramientas confirma que \textit{TerapiTrack} es completamente viable desde el punto de vista legal, siempre que se respeten los avisos de licencia de terceros y se cumpla la normativa de protección de datos aplicable al tratamiento de información sanitaria.

La tabla \ref{tab:dependencias_terapitrack} recoge el análisis de licencias de todas las dependencias principales del proyecto.

\begin{table}[!ht]
\centering
\small
\begin{tabular}{p{3.5cm} p{2.2cm} p{6cm}}
\toprule
\textbf{Dependencia} & \textbf{Versión} & \textbf{Licencia y uso} \\
\midrule
Flask & 3.0.0 & BSD-3-Clause. Microframework web. \\
Werkzeug & 3.1.4 & BSD-3-Clause. Librería WSGI (uso indirecto vía Flask). \\
Jinja2 & 3.1.2 & BSD. Motor de plantillas. \\
SQLAlchemy & 2.0.23 & MIT. ORM para BD relacional. \\
Flask-SQLAlchemy & 3.1.1 & BSD. Integración SQLAlchemy--Flask. \\
SQLite & (embebido) & Dominio público. BD embebida. \\
Flask-Login & 0.6.3 & MIT. Gestión de sesiones de usuario. \\
Flask-WTF & 1.2.1 & BSD-3-Clause. Formularios con CSRF. \\
WTForms & 3.1.1 & BSD-3-Clause. Validación de formularios. \\
bcrypt & 5.0.0 & Apache-2.0. Cifrado de contraseñas. \\
Flask-Bcrypt & 1.0.1 & BSD. Integración bcrypt--Flask. \\
Pygame & 2.5.2 & LGPL. Comunicación con el mando SNES. \\
Bootstrap & 5.3.x & MIT. Framework CSS para la interfaz. \\
Bootswatch & (temas) & MIT. Colección de temas para Bootstrap. \\
Click & 8.3.1 & BSD. CLI interna de Flask. \\
ItsDangerous & 2.2.0 & BSD. Firma segura de tokens. \\
MarkupSafe & 3.0.3 & BSD. Escape de HTML en plantillas. \\
Blinker & 1.9.0 & MIT. Sistema de señales en Flask. \\
Greenlet & 3.2.4 & MIT. Soporte de contextos ligeros. \\
Colorama & 0.4.6 & BSD. Colores en la salida de terminal. \\
Typing-Extensions & 4.15.0 & PSF. Tipos adicionales para Python. \\
\bottomrule
\end{tabular}
\caption{Análisis de dependencias y licencias de \textit{TerapiTrack}.}
\label{tab:dependencias_terapitrack}
\end{table}
