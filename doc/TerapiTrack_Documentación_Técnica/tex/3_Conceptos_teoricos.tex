\capitulo{3}{Conceptos teóricos}
En este capítulo se revisan los conceptos teóricos básicos que sirven de soporte al proyecto TerapiTrack.
Primero se abordan nociones relacionadas con la enfermedad de Parkinson y la rehabilitación, y después se describen los principios de accesibilidad, los patrones de diseño y los modelos de datos que se han tenido en cuenta durante el desarrollo.

\section{Enfermedades degenerativas y enfermedad de Parkinson}
Las enfermedades degenerativas son trastornos en los que determinadas estructuras del organismo se deterioran de forma progresiva, lo que provoca una pérdida lenta pero continua de funciones.
Cuando el sistema nervioso central se ve afectado, este deterioro suele traducirse en problemas de movimiento, equilibrio, memoria o comunicación que requieren tratamientos prolongados y un seguimiento estrecho \cite{who_parkinson_2023,ninds_parkinson}.

La enfermedad de Parkinson es un trastorno neurodegenerativo crónico que afecta principalmente al control del movimiento.
Se caracteriza por temblor en reposo, rigidez muscular, lentitud en la iniciación de los movimientos y alteraciones del equilibrio, y actualmente no tiene cura.
La combinación de tratamiento farmacológico y programas de rehabilitación específicos permite reducir la intensidad de los síntomas, mantener la autonomía durante más tiempo y mejorar la calidad de vida de los pacientes \cite{cleveland_parkinson,who_parkinson_2023}.

\section{Terapia ocupacional y rehabilitación}
La terapia ocupacional es una disciplina sanitaria que ayuda a las personas a participar de forma lo más independiente posible en las actividades significativas de su vida diaria, adaptando tanto las tareas como el entorno cuando existe una enfermedad, lesión o discapacidad \cite{nbcot_ot,rehab_hss}.
En pacientes con enfermedad de Parkinson, la terapia ocupacional trabaja aspectos como el vestido, la higiene personal, la movilidad en el hogar o la organización de rutinas, con el objetivo de prolongar la autonomía en las actividades básicas y sociales \cite{rehab_hss}.

La rehabilitación de trastornos del movimiento suele combinar fisioterapia, terapia ocupacional, ejercicio físico estructurado y, en algunos casos, logopedia.
Este tipo de intervención requiere sesiones frecuentes y un seguimiento continuado, por lo que acceder de forma regular a los servicios especializados resulta especialmente complicado para pacientes que viven lejos de los centros de referencia o que tienen dificultades para desplazarse \cite{parkinson_telerehab_review}.

\section{Telemedicina y telerehabilitación}
La telemedicina hace referencia al uso de las tecnologías de la información y la comunicación para prestar servicios sanitarios a distancia, permitiendo que paciente y profesional interactúen sin encontrarse en el mismo lugar físico \cite{rosen2009_telerehab_technologies}.
Dentro de este ámbito, la telerehabilitación se centra en el diseño, la supervisión y la evaluación de programas de rehabilitación mediante herramientas remotas, como videoconferencia o plataformas web específicas \cite{rosen2009_telerehab_technologies}.

En el caso de la enfermedad de Parkinson, la telerehabilitación ofrece la posibilidad de mantener la frecuencia de las terapias aunque el paciente resida en la “España vaciada” o tenga movilidad reducida.
Permite reducir desplazamientos, facilita la adherencia al tratamiento y hace posible que el profesional revise de forma diferida los ejercicios realizados en casa \cite{parkinson_telerehab_review,parkinson_telerehab_protocol}.

\section{Accesibilidad y usabilidad en aplicaciones sanitarias}
Las aplicaciones orientadas a la salud deben diseñarse pensando en usuarios con diferentes capacidades motoras, sensoriales o cognitivas, y también en personas de edad avanzada con poca experiencia en el uso de ordenadores o dispositivos móviles.
Ni la edad ni el nivel de conocimiento tecnológico deberían convertirse en una barrera a la hora de acceder a programas de telerehabilitación \cite{rosen2009_telerehab_technologies,telehealth_accessibility}.

Los estándares de accesibilidad web, como las pautas WCAG, recogen cuatro principios generales que sirven de guía para el diseño de interfaces accesibles \cite{wcag22}.

\begin{itemize}
    \item \textbf{Perceptibilidad}: La información y los componentes de la interfaz deben presentarse de forma que puedan percibirse por distintos tipos de usuarios, por ejemplo cuidando el contraste de color, el tamaño de la tipografía o el uso de textos alternativos en imágenes \cite{wcag22}.
    \item \textbf{Operabilidad}: Los controles deben poder manejarse con diferentes dispositivos de entrada (ratón, teclado, pantallas táctiles o mandos) y con un número reducido de acciones, algo especialmente relevante en personas con limitaciones motoras \cite{rosen2009_telerehab_technologies,telehealth_accessibility}.
    \item \textbf{Comprensibilidad}: Los textos, mensajes de error y flujos de navegación han de ser claros y coherentes, evitando tecnicismos innecesarios y sobrecarga de información \cite{wcag22}.
    \item \textbf{Robustez}: El contenido debe mostrarse correctamente en distintos navegadores y dispositivos, y ser compatible con tecnologías de apoyo como lectores de pantalla o ampliadores de pantalla \cite{wcag22}.
\end{itemize}

Estos principios han guiado el diseño de la interfaz de TerapiTrack, priorizando menús sencillos, textos claros y un número reducido de acciones por pantalla.

\section{Patrones de diseño y arquitectura Modelo–Vista–Controlador}
En el desarrollo de aplicaciones web es habitual apoyarse en patrones de diseño que separan responsabilidades y facilitan la evolución del sistema.
Uno de los más extendidos es el patrón Modelo–Vista–Controlador (MVC), que organiza la aplicación en tres componentes principales \cite{gamma_design_patterns}:

\begin{itemize}
    \item \textbf{Modelo}: Representa los datos y las reglas de negocio asociadas a cada entidad del dominio, incluyendo la definición de estructuras de almacenamiento y restricciones de integridad.
    \item \textbf{Vista}: Define cómo se presenta la información al usuario, normalmente mediante páginas web o plantillas que muestran formularios, listados y mensajes.
    \item \textbf{Controlador}: Actúa como intermediario entre el modelo y la vista, recibiendo las peticiones del usuario, invocando la lógica necesaria y seleccionando la respuesta adecuada.
\end{itemize}

Este tipo de arquitectura favorece que el código relacionado con la persistencia de datos, la lógica de aplicación y la interfaz de usuario pueda evolucionar de manera relativamente independiente, algo especialmente útil en proyectos que pueden ampliarse en el futuro \cite{gamma_design_patterns}.

\section{Bases de datos relacionales}
Las bases de datos relacionales permiten organizar la información en tablas relacionadas entre sí mediante claves primarias y foráneas, garantizando la integridad de los datos mediante restricciones y reglas de consistencia \cite{date_db}.
Este modelo resulta adecuado para almacenar información clínica y de seguimiento, donde es importante mantener un historial coherente de usuarios, sesiones y evaluaciones.

La \textbf{normalización} es el proceso mediante el cual se descompone la información en tablas más pequeñas para reducir la redundancia y evitar anomalías en las operaciones de inserción, actualización y borrado.
Aplicar niveles adecuados de normalización ayuda a mantener la coherencia de los datos y a simplificar su mantenimiento a largo plazo \cite{date_db}.

Una \textbf{relación de muchos a muchos} aparece cuando varios registros de una tabla pueden asociarse con varios registros de otra.
En estos casos se introduce una tabla intermedia que almacena los pares de elementos relacionados y permite gestionar de forma clara y trazable esas asociaciones (por ejemplo, la vinculación entre pacientes y profesionales o la asignación de ejercicios a sesiones) \cite{date_db}.

\section{Validación y pruebas de software}
La validación de un sistema de software busca comprobar que la solución desarrollada cumple los requisitos definidos y que se comporta de forma correcta en los escenarios de uso previstos.
Para ello se utilizan diferentes tipos de pruebas, que pueden combinarse según las necesidades del proyecto \cite{pressman_ingenieria_sw}.

Entre las más habituales se encuentran las \textbf{pruebas unitarias}, centradas en módulos o funciones concretas; las \textbf{pruebas de integración}, que comprueban el funcionamiento conjunto de varios componentes; y las \textbf{pruebas funcionales} o de sistema, orientadas a verificar flujos de usuario o casos de uso completos.
Automatizar parte de estas pruebas permite detectar errores de diseño o programación de manera temprana y reduce el riesgo de fallos en fases avanzadas del desarrollo \cite{pressman_ingenieria_sw}.

Este conjunto de conceptos proporciona el contexto teórico y tecnológico que sustenta la memoria y ayuda a entender las decisiones que se detallan en los capítulos posteriores.
