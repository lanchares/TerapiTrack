\capitulo{3}{Conceptos teóricos}

Este capítulo revisa los conceptos teóricos básicos que sirven de soporte a TerapiTrack, tanto en su diseño como en su implementación.
Primero se abordan nociones relacionadas con la enfermedad de Parkinson y la rehabilitación y, a continuación, se tratan aspectos de accesibilidad en aplicaciones sanitarias, los patrones de diseño, los modelos de datos y los fundamentos de validación y pruebas de software que se han tenido en cuenta durante el desarrollo.

\section{Enfermedades degenerativas y enfermedad de Parkinson}

Las enfermedades degenerativas son trastornos en los que determinadas estructuras del organismo se deterioran de forma progresiva, lo que provoca una pérdida lenta pero continua de funciones.
Cuando el sistema nervioso central se ve afectado, este deterioro suele traducirse en problemas de movimiento, equilibrio, memoria o comunicación, que requieren tratamientos prolongados y un seguimiento intensivo~\cite{ninds_parkinson,who_parkinson_2023}.

La enfermedad de Parkinson es un trastorno neurodegenerativo crónico que afecta principalmente al control del movimiento.
Se caracteriza por temblor en reposo, rigidez muscular, lentitud en la iniciación de los movimientos y alteraciones del equilibrio, y actualmente no tiene cura.
Aunque puede aparecer en adultos jóvenes, afecta sobre todo a personas de edad avanzada; se estima que aproximadamente una de cada cien personas mayores de 60 años y en torno a un 2\% de la población española mayor de 65 años padecen la enfermedad,
lo que implica que muchos pacientes presentan limitaciones de movilidad y dependen de terceras personas para desplazarse, especialmente cuando viven en ámbitos rurales~\cite{medicosypacientes_parkinson_mayores60,sen_parkinson_2porciento,who_parkinson_2023}.

La combinación de tratamiento farmacológico y programas de rehabilitación específicos permite reducir la intensidad de los síntomas, mantener la autonomía durante más tiempo y mejorar la calidad de vida de los pacientes~\cite{cleveland_parkinson,who_parkinson_2023}.

\section{Terapia ocupacional y rehabilitación}

La terapia ocupacional es una disciplina sanitaria que ayuda a las personas a mantener o mejorar su autonomía en las actividades significativas de su vida diaria, adaptando tanto las tareas como el entorno cuando existe una enfermedad, lesión o discapacidad~\cite{rehab_hss,nbcot_ot}.
En pacientes con enfermedad de Parkinson, la terapia ocupacional trabaja aspectos como vestirse, la higiene personal, la movilidad dentro del domicilio o la organización de rutinas, con el objetivo de prolongar la autonomía en las actividades básicas y sociales~\cite{rehab_hss}.

La rehabilitación de trastornos del movimiento suele combinar fisioterapia, terapia ocupacional, ejercicio físico estructurado y, en algunos casos, logopedia.
Este tipo de intervención requiere sesiones frecuentes y un seguimiento continuado, por lo que acceder de forma regular a los servicios especializados resulta especialmente complicado para pacientes que viven lejos de los centros de referencia o que tienen dificultades para desplazarse~\cite{parkinson_telerehab_review}.

\section{Telemedicina y telerehabilitación}

La telemedicina hace referencia al uso de las tecnologías de la información y la comunicación para prestar servicios sanitarios a distancia, permitiendo que paciente y profesional interactúen sin encontrarse en el mismo lugar físico.
Dentro de este ámbito, la telerehabilitación se centra en el diseño, la supervisión y la evaluación de programas de rehabilitación mediante herramientas remotas, como la videoconferencia o plataformas web específicas~\cite{rosen2009_telerehab_technologies}.

En el caso de la enfermedad de Parkinson, la telerehabilitación ofrece la posibilidad de mantener la frecuencia de las terapias aunque el paciente resida en la <<España vaciada>> o tenga movilidad reducida.
Permite reducir desplazamientos, facilita la adherencia al tratamiento y hace posible que el profesional revise de forma diferida los ejercicios realizados en casa~\cite{parkinson_telerehab_review,parkinson_telerehab_protocol}.

\section{Accesibilidad y usabilidad en aplicaciones sanitarias}

En el contexto de la telerehabilitación para personas con enfermedad de Parkinson, la accesibilidad de una aplicación no es un añadido opcional, sino una condición para que la herramienta pueda utilizarse en la práctica diaria.
Si los menús son complejos, los textos difíciles de leer o se requieren demasiados pasos para completar una tarea, es probable que muchos pacientes abandonen el uso de la plataforma aunque el contenido terapéutico sea adecuado~\cite{telehealth_accessibility}.

Por ello, este tipo de aplicaciones se diseña pensando especialmente en usuarios con posibles limitaciones motoras y menor experiencia con dispositivos digitales, como ocurre con muchos pacientes de Parkinson.
En TerapiTrack las decisiones de accesibilidad se han centrado principalmente en la interfaz dirigida al paciente, mientras que las vistas destinadas al profesional sanitario mantienen una estructura más cercana a la de una aplicación web convencional, asumiendo un mayor grado de familiaridad con estas herramientas.

Las pautas Web Content Accessibility Guidelines (WCAG) ofrecen recomendaciones sobre contraste, tipografía, estructura y compatibilidad con tecnologías de apoyo, que sirven como referencia para adaptar la interfaz a personas mayores o con dificultades motoras~\cite{wcag22}.
En la parte de paciente de TerapiTrack estas ideas se han traducido en un menú sencillo, textos breves y claros, pocos elementos interactivos por pantalla y compatibilidad con dispositivos alternativos, como el mando SNES integrado únicamente en las vistas de paciente.

\section{Patrones de diseño y arquitectura Modelo-Vista-Controlador}

En el desarrollo de aplicaciones de software es habitual apoyarse en patrones de diseño que separan responsabilidades y facilitan la evolución del sistema; en el caso de TerapiTrack, esto se traduce en el uso de una arquitectura de estilo Modelo-Vista-Controlador (MVC)~\cite{gamma_design_patterns}.
Este patrón organiza la aplicación en tres componentes principales:

\begin{itemize}
    \item \textbf{Modelo}: representa los datos y las reglas de negocio asociadas a cada entidad del dominio, incluyendo la definición de estructuras de almacenamiento y restricciones de integridad.
    \item \textbf{Vista}: define cómo se presenta la información al usuario, normalmente mediante páginas web o plantillas que muestran formularios, listados y mensajes.
    \item \textbf{Controlador}: actúa como intermediario entre el modelo y la vista, recibiendo las peticiones del usuario, invocando la lógica necesaria y seleccionando la respuesta adecuada.
\end{itemize}

Este tipo de arquitectura favorece que el código relacionado con la persistencia de datos, la lógica de aplicación y la interfaz de usuario puedan evolucionar de manera relativamente independiente, algo especialmente útil en proyectos que pueden ampliarse en el futuro~\cite{gamma_design_patterns}.

\section{Bases de datos relacionales}

Las bases de datos relacionales permiten organizar la información en tablas relacionadas entre sí mediante claves primarias y foráneas, garantizando la integridad de los datos mediante restricciones y reglas de consistencia~\cite{date_db}.
Este modelo es el que se ha aplicado en TerapiTrack para organizar usuarios, pacientes, profesionales sanitarios, ejercicios, sesiones, evaluaciones y vídeos de respuesta.

En este contexto aparecen relaciones de distinto tipo: \textbf{uno a muchos}, como la que se da entre un paciente y las sesiones que tiene programadas, o entre un profesional sanitario y las sesiones que coordina;
\textbf{uno a uno}, como la que existe entre la tabla de usuarios y las tablas de paciente o profesional, donde cada registro de usuario se especializa en un único tipo de perfil;
y \textbf{muchos a muchos}, como ocurre al vincular profesionales sanitarios con varios pacientes o al asociar ejercicios concretos a distintas sesiones, que se representan mediante tablas intermedias específicas~\cite{date_db}.

La \textbf{normalización} es el proceso mediante el cual se descompone la información en tablas más pequeñas para reducir la redundancia y evitar anomalías en las operaciones de inserción, actualización y borrado.
En términos prácticos, suele implicar diseñar el esquema de forma que cumpla las principales formas normales: primera forma normal (1FN), que exige atributos con valores atómicos;
segunda forma normal (2FN), que elimina dependencias parciales respecto a claves compuestas; y tercera forma normal (3FN), que evita dependencias transitivas entre atributos que no forman parte de la clave~\cite{date_db}.

En el caso de TerapiTrack, estas formas normales se han tenido en cuenta al distribuir los atributos en tablas especializadas y al introducir tablas intermedias como \textit{Paciente\_Profesional} o \textit{Ejercicio\_Sesion},
lo que facilita mantener la coherencia y la trazabilidad del modelo de datos~\cite{date_db}.

\section{Validación y pruebas de software}

La validación de un sistema de software busca comprobar que la solución desarrollada cumple los requisitos definidos y que se comporta de forma correcta en los escenarios de uso previstos.
Para ello se utilizan diferentes tipos de pruebas, que pueden combinarse según las necesidades del proyecto~\cite{pressman_ingenieria_sw}.

Entre las más habituales se encuentran las \textbf{pruebas unitarias}, centradas en módulos o funciones concretas; las \textbf{pruebas de integración}, que comprueban el funcionamiento conjunto de varios componentes; y las \textbf{pruebas funcionales} o de sistema, orientadas a verificar flujos de usuario o casos de uso completos.
Automatizar parte de estas pruebas permite detectar errores de diseño o programación de manera temprana y reduce el riesgo de fallos en fases avanzadas del desarrollo~\cite{pressman_ingenieria_sw}.


Este conjunto de conceptos proporciona el contexto teórico y tecnológico que sustenta la memoria y ayuda a entender las decisiones que se detallan en los capítulos posteriores.
