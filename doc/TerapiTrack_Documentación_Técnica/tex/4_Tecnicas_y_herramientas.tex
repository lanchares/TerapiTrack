\capitulo{4}{Técnicas y herramientas}
Este capítulo presenta las metodologías, tecnologías y utilidades empleadas en el desarrollo de TerapiTrack. 
El propósito es ofrecer una visión ordenada del entorno de trabajo y de la base tecnológica sobre la que se construye la aplicación, de manera que sirva de referencia para entender las decisiones de diseño descritas en los capítulos siguientes~\cite{UbuPlantillaLatex,GithubTerapitrack}.

\section{Metodología de desarrollo}

\subsection{Metodologías ágiles y Jira}
Las metodologías ágiles plantean una forma de organizar los proyectos de software en iteraciones cortas, con entregas frecuentes y una revisión continua de las tareas pendientes~\cite{pressman_ingenieria_sw}. 
En lugar de planificar todo el desarrollo al detalle desde el principio, se trabaja con ciclos de duración limitada en los que se van cerrando bloques pequeños de funcionalidad.

En TerapiTrack se ha utilizado \textbf{Jira} como herramienta de apoyo a esta forma de trabajo~\cite{JiraTerapitrack}.
Jira permite crear tareas, agruparlas en sprints, registrar incidencias y visualizar el estado del proyecto en tableros, lo que facilita ver de un vistazo qué está hecho, qué está pendiente y qué se ha replanificado durante el desarrollo.

\section{Tecnologías de backend}

\subsection{Python y Flask}
El \textit(backend) de TerapiTrack está desarrollado en \textbf{Python}, un lenguaje de programación de propósito general muy extendido en el ámbito del desarrollo web, la ciencia de datos y la automatización~\cite{pressman_ingenieria_sw}. 
Entre sus ventajas destacan una sintaxis relativamente sencilla y una comunidad amplia que mantiene librerías para la mayoría de necesidades habituales en un proyecto.

Sobre Python se ha utilizado el microframework web \textbf{Flask}, que proporciona las piezas básicas para construir una aplicación web:
definición de rutas, manejo de peticiones y respuestas HTTP, integración con sistemas de plantillas y soporte para extensiones que cubren necesidades como la autenticación o el acceso a bases de datos~\cite{FlaskDocs,JinjaDocs}.

En el proyecto la aplicación principal se define en \texttt{app.py} y se organiza en varios módulos de controladores que agrupan las rutas según el rol del usuario (\texttt{admin\_controlador.py}, \texttt{profesional\_controlador.py}, \texttt{auth\_controlador.py}, etc.), siguiendo el patrón de \emph{blueprints} descrito en la documentación oficial y en guías prácticas específicas~\cite{FlaskBlueprintsDocs,RealPythonBlueprints}.
Además, se han definido decoradores propios en \texttt{decoradores.py} para comprobar el rol y otras condiciones de acceso antes de ejecutar determinadas vistas.

\subsection{SQLAlchemy}
Para la interacción con la base de datos se emplea \textbf{SQLAlchemy}, una biblioteca de mapeo objeto–relacional (ORM, \textit{Object–Relational Mapping}) ampliamente utilizada en el ecosistema Python~\cite{SQLAlchemyDocs}. 
Un ORM permite trabajar con las tablas de la base de datos a través de clases y objetos, de forma que las consultas y actualizaciones se expresan en código Python y la biblioteca se encarga de traducirlas a SQL~\cite{FlaskSQLAlchemyPattern}.

En TerapiTrack los modelos se encuentran separados en ficheros como \texttt{usuario.py}, \texttt{paciente.py}, \texttt{profesional.py}, \texttt{ejercicio.py}, \texttt{sesion.py} o \texttt{evaluacion.py}, y se complementan con un módulo de asociaciones para las relaciones de muchos a muchos.
SQLAlchemy se inicializa en el módulo de extensiones (\texttt{extensiones.py}), lo que permite reutilizar la misma instancia en toda la aplicación~\cite{SQLAlchemyDocs}.

\section{Gestión de datos}

\subsection{Bases de datos en desarrollo y producción}
Durante el desarrollo local se ha utilizado \textbf{SQLite} como motor de base de datos principal~\cite{date_db}. 
SQLite es un sistema de base de datos relacional embebido que almacena toda la información en un único fichero en disco y no requiere un servidor independiente, lo que simplifica la configuración del entorno de trabajo.

Para el despliegue en la nube se ha configurado la aplicación para trabajar con \textbf{PostgreSQL}, un gestor de bases de datos relacional de código abierto ampliamente usado en entornos de producción.
En Heroku se emplea el complemento de base de datos PostgreSQL, al que la aplicación se conecta mediante la cadena de conexión definida en las variables de entorno~\cite{HerokuTerapitrackApp,HerokuDashboard}.

La selección del motor de base de datos en cada entorno se realiza a través del fichero \texttt{config.py}, donde se define la \texttt{SQLALCHEMY\_DATABASE\_URI}.
Además, se dispone del script \texttt{poblar\_bd.py}, que permite inicializar la base de datos con datos de ejemplo (usuarios, pacientes, ejercicios, sesiones y evaluaciones) para facilitar las pruebas.

\subsection{Herramienta complementaria: DB Browser for SQLite}
Como apoyo al trabajo con la base de datos local se ha utilizado \textbf{DB Browser for SQLite}, una herramienta gráfica que permite inspeccionar tablas, ejecutar consultas y modificar registros de forma visual~\cite{DBBrowser}.
Su uso ha sido especialmente útil en las primeras iteraciones del diseño del esquema y durante la depuración de datos de prueba.

\section{Tecnologías de frontend}

\subsection{HTML5, JavaScript y Jinja}
La capa de presentación de TerapiTrack se construye sobre varias tecnologías web:

\begin{itemize}
    \item \textbf{HTML5} (HyperText Markup Language) es el estándar de marcado utilizado para definir la estructura de las páginas web, incluyendo encabezados, párrafos, formularios o tablas.
    \item \textbf{JavaScript} es un lenguaje de programación orientado al desarrollo en el lado del cliente, que permite añadir interactividad a las páginas, reaccionar a eventos del usuario y realizar peticiones asíncronas al servidor.
    \item \textbf{Jinja} es un motor de plantillas para Python que se integra con Flask y permite generar HTML de forma dinámica a partir de plantillas y datos, reutilizando estructuras comunes en distintas vistas~\cite{JinjaDocs}.
\end{itemize}

Las plantillas principales se organizan en carpetas por rol (\texttt{admin}), (\texttt{profesional}), (\texttt{paciente}). 
Todas ellas heredan de una plantilla base (\texttt{base.html}) que define la estructura común de menús, cabeceras y mensajes.
Entre las vistas más relevantes se encuentran los distintos paneles de control (\texttt{dashboard.html}) y las páginas de ejecución de sesiones (\texttt{ejecutar\_sesion.html}), donde se combinan los vídeos de ejemplo con la cámara del paciente y los controles necesarios para completar la sesión.

\subsection{Bootstrap y Bootswatch}
Para el diseño visual se ha empleado \textbf{Bootstrap}, un \emph{framework} CSS que ofrece componentes predefinidos (botones, menús, rejillas, formularios) y un sistema de diseño adaptable a diferentes tamaños de pantalla~\cite{pressman_ingenieria_sw}. 
Sobre Bootstrap se han aplicado temas de \textbf{Bootswatch}, que proporcionan estilos alternativos listos para usar sin modificar el marcado HTML~\cite{Bootswatch}.

El uso de estos componentes facilita mantener una apariencia coherente en toda la aplicación y aplicar criterios básicos de accesibilidad, como tamaños de fuente adecuados, contraste suficiente y distribución ordenada de los elementos en la interfaz~\cite{telehealth_accessibility,wcag22}.

\subsection{Soporte para mando SNES}

Además del uso con ratón y teclado, TerapiTrack incorpora soporte para un mando SNES conectado por USB mediante la \textit{Gamepad API} del navegador~\cite{GithubTerapitrack}.
En el área de paciente, las acciones básicas de navegación (mover el foco entre elementos, seleccionar opciones o volver atrás) pueden realizarse con los botones del mando, lo que facilita el uso de la aplicación a personas con dificultades motoras finas o menor familiaridad con el ratón~\cite{telehealth_accessibility}.

Esta funcionalidad se ha implementado mediante un módulo JavaScript que escucha los eventos del mando, traduce las pulsaciones a acciones de la interfaz y actualiza el foco de forma visual para indicar qué elemento está seleccionado en cada momento~\cite{GithubTerapitrack}.


\section{Testing y garantía de calidad}

\subsection{Pytest}
Las pruebas automáticas se han realizado con \textbf{Pytest}, un marco de testing para Python que permite definir casos de prueba y agruparlos en módulos dentro de la carpeta \texttt{tests}~\cite{PytestDocs}.
Pytest ofrece utilidades para preparar datos de prueba mediante \textit{fixtures}, parametrizar pruebas y generar informes con el resultado de la ejecución.

En el contexto de TerapiTrack, se ha utilizado para comprobar el funcionamiento de los modelos de datos y algunos de los flujos principales definidos en los controladores (gestión de usuarios, sesiones y evaluaciones), reduciendo el riesgo de introducir errores al modificar el código~\cite{GithubTerapitrack}.

\section{Despliegue e infraestructura}

\subsection{Heroku}
Para ejecutar la aplicación en un entorno accesible desde internet se ha utilizado \textbf{Heroku}, una plataforma \textit{Platform as a Service} (PaaS)~\cite{HerokuDashboard}.
Heroku permite desplegar aplicaciones a partir de un repositorio Git y definir el proceso de arranque mediante un fichero \texttt{Procfile}, en el que se indica el comando que debe ejecutarse para iniciar la aplicación Flask con un servidor compatible con WSGI~\cite{HerokuTerapitrackApp}.

La configuración concreta del entorno (por ejemplo, la URL de la base de datos PostgreSQL o la clave secreta de Flask) se realiza mediante variables de entorno, que la aplicación lee en tiempo de ejecución a través del módulo de configuración.

Para el almacenamiento de archivos multimedia se ha integrado \textbf{Cloudinary}, un servicio externo que permite gestionar y servir de forma eficiente los vídeos y otros recursos generados por la aplicación~\cite{CloudinaryConsole}.

\section{Control de versiones y colaboración}

\subsection{Git y GitHub}
El código del proyecto se gestiona con \textbf{Git}, un sistema de control de versiones distribuido que mantiene un historial de cambios y permite trabajar con ramas para desarrollar nuevas funcionalidades de forma aislada~\cite{GitDocs}. 
Además, como repositorio remoto se ha empleado \textbf{GitHub}, que ofrece alojamiento para el código, sistema de \textit{issues} para registrar tareas e incidencias y herramientas para revisar cambios antes de integrarlos en la rama principal~\cite{GithubTerapitrack}.

Para el trabajo diario se ha utilizado \textbf{Git Bash} como interfaz de línea de comandos, lo que facilita ejecutar los comandos de Git desde el propio entorno de desarrollo.

\section{Otras herramientas de desarrollo}

Durante el desarrollo también se han utilizado varias herramientas de apoyo:

\begin{itemize}
    \item \textbf{Visual Studio Code}: es un entorno de desarrollo integrado utilizado como editor principal para el código Python, las plantillas y los ficheros de configuración, con extensiones específicas para Flask, Git y trabajo con SQLite.
    \item \textbf{LaTeX Workshop y Overleaf}: son herramientas para la edición y compilación de la memoria y los anexos en \LaTeX{}, combinando el trabajo local en Visual Studio Code con la plataforma colaborativa Overleaf~\cite{UbuPlantillaLatex,OverleafEditor}. 
    \item \textbf{Draw.io}: es una aplicación para la creación de diagramas, utilizada para representar la arquitectura general del sistema, el modelo de datos y los principales flujos de uso.
\end{itemize}

\section{Resumen de herramientas utilizadas}

El conjunto de técnicas y herramientas descrito en este capítulo constituye la base tecnológica de TerapiTrack y sirve como referencia para comprender los aspectos de diseño e implementación que se desarrollan en los capítulos de aspectos relevantes y en los anexos técnicos~\cite{GithubTerapitrack}.

\tablaSmall{Herramientas utilizadas en el proyecto}{l l p{7.5cm}}
{herramientas_terapitrack}
{ \textbf{Herramienta} & \textbf{Ámbito} & \textbf{Descripción} \\}
{
Flask & Backend & Lógica de servidor y gestión de rutas y vistas web. \\
SQLAlchemy & Base de datos & ORM para definir modelos y gestionar consultas sobre bases de datos relacionales. \\
SQLite / PostgreSQL & Base de datos & Motores relacionales usados respectivamente en desarrollo local y en despliegue en la nube. \\
DB Browser for SQLite & Base de datos & Cliente gráfico para inspección, consulta y edición de la base de datos SQLite. \\
Jinja & Plantillas & Motor de plantillas para generar dinámicamente las páginas HTML desde Flask. \\
HTML5 / JavaScript & Frontend & Estructura del contenido y lógica en el navegador para dotar de interactividad a la interfaz. \\
Bootstrap / Bootswatch & Frontend & Framework CSS y temas visuales para construir una interfaz adaptable y coherente. \\
Cloudinary & Almacenamiento & Servicio externo para almacenar y servir archivos multimedia generados por la aplicación. \\
Jira & Gestión & Planificación, seguimiento de tareas y organización en sprints. \\
Git & Versionado & Sistema de control de versiones distribuido para gestionar la evolución del código. \\
Git Bash & Versionado & Uso de Git desde la línea de comandos en el entorno local. \\
GitHub & Repositorio & Alojamiento remoto del código, sistema de \textit{issues} y revisión de cambios. \\
Visual Studio Code & Desarrollo & Entorno integrado de edición, depuración y ejecución del código fuente. \\
LaTeX Workshop / Overleaf & Documentación & Herramientas para la redacción y compilación de la memoria y los anexos en \LaTeX{}. \\
Heroku & Despliegue & Plataforma PaaS utilizada para ejecutar la aplicación Flask y la base de datos en la nube. \\
}

La tabla resume las principales herramientas empleadas en el desarrollo de TerapiTrack, indicando su ámbito de uso y su función principal~\cite{GithubTerapitrack}.