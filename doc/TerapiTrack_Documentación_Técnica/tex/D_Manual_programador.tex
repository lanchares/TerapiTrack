\apendice{Documentación técnica de programación}

\section{Introducción}

Este anexo recoge la documentación técnica de TerapiTrack dirigida a desarrolladores.
El objetivo es describir la organización del código, los componentes principales de la aplicación y los pasos necesarios para compilar, instalar, ejecutar y probar el sistema en un entorno de desarrollo estándar.

De este modo, cualquier desarrollador que se incorpore al proyecto puede entender con rapidez cómo está estructurada la aplicación y qué pasos seguir para trabajar sobre ella.

\section{Estructura de directorios}

El proyecto se organiza siguiendo una estructura modular que separa claramente la lógica de negocio, los modelos de datos, las vistas y las pruebas automatizadas.
A continuación se resumen los directorios más relevantes:

\begin{itemize}
  \item \texttt{src/controladores}: Contiene los controladores de Flask organizados por dominio:
  \begin{itemize}
    \item \texttt{auth\_controlador.py}: Gestión de autenticación, login y logout.
    \item \texttt{admin\_controlador.py}: Operaciones de administración de usuarios y configuración global.
    \item \texttt{profesional\_controlador.py}: Gestión de pacientes, ejercicios, sesiones y evaluaciones desde la perspectiva del profesional.
    \item \texttt{paciente\_controlador.py}: Panel de paciente, ejecución de sesiones y consulta de progreso.
  \end{itemize}
  \item \texttt{src/modelos}: Define los modelos de datos con SQLAlchemy:
  \begin{itemize}
    \item \texttt{usuario.py}, \texttt{paciente.py}, \texttt{profesional.py}: Entidades básicas de usuarios y perfiles sanitarios.
    \item \texttt{ejercicio.py}, \texttt{sesion.py}, \texttt{evaluacion.py}, \texttt{videoRespuesta.py}: Entidades relacionadas con ejercicios, sesiones, vídeos y evaluaciones.
    \item Ficheros de asociaciones para las relaciones muchos a muchos (por ejemplo, paciente--profesional o ejercicio--sesión).
  \end{itemize}
  \item \texttt{src/vistas}: Plantillas HTML Jinja2 organizadas por rol:
  \begin{itemize}
    \item \texttt{admin/}: Vistas del panel de administración.
    \item \texttt{auth/}: Páginas de login y perfil.
    \item \texttt{profesional/}: Pantallas de gestión de sesiones, ejercicios y evaluaciones.
    \item \texttt{paciente/}: Dashboard de paciente, ejecución de sesiones y visualización de progreso.
  \end{itemize}
  \item \texttt{src/static}: Recursos estáticos de la aplicación:
  \begin{itemize}
    \item \texttt{css/}: Hojas de estilo, incluyendo temas basados en Bootstrap y Bootswatch.
    \item \texttt{js/}: Ficheros JavaScript, entre ellos la lógica para la Gamepad API y el control con mando SNES.
    \item \texttt{uploads/ejercicios/}: Vídeos subidos por los profesionales.
    \item \texttt{videos/}: Vídeos base y recursos multimedia de ejemplo.
  \end{itemize}
  \item \texttt{src/forms.py}: Definición de formularios WTForms utilizados en login, gestión de usuarios, creación de ejercicios, sesiones y evaluaciones.
  \item \texttt{src/config.py}: Parámetros de configuración de la aplicación (cadena de conexión a base de datos, claves de seguridad, integración con Cloudinary, etc.).
  \item \texttt{src/extensiones.py}: Inicialización de extensiones de Flask (SQLAlchemy, Flask-Login, CSRF, etc.).
  \item \texttt{tests}: Conjunto de pruebas unitarias e integración sobre modelos y controladores.
  \item \texttt{config}: Archivos JSON de apoyo para la configuración y carga de datos.
  \item \texttt{doc}: Ficheros \LaTeX{} correspondientes a la memoria y a los anexos del TFG.
  \item \texttt{app.py}: Punto de entrada de la aplicación Flask.
  \item \texttt{poblar\_bd.py}: Script para inicializar la base de datos con datos de prueba.
  \item \texttt{Procfile}, \texttt{requirements.txt}, \texttt{pytest.ini}: Ficheros auxiliares para despliegue en Heroku, gestión de dependencias y configuración de pruebas.
\end{itemize}

\section{Manual del programador}

Desde el punto de vista del programador, TerapiTrack se apoya en Flask como framework web principal y en SQLAlchemy como capa de acceso a datos.
La aplicación se construye siguiendo un patrón Modelo--Vista--Controlador adaptado a Flask, en el que:

\begin{itemize}
  \item Los \textbf{modelos} se definen como clases de SQLAlchemy en \texttt{src/modelos}, incluyendo atributos, relaciones y métodos auxiliares.
  \item Los \textbf{controladores} se implementan como \textit{blueprints} de Flask en \texttt{src/controladores}, agrupando rutas según el rol o el dominio funcional.
  \item Las \textbf{vistas} se generan mediante plantillas Jinja2 en \texttt{src/vistas}, combinadas con recursos estáticos en \texttt{src/static}.
\end{itemize}

El flujo de una petición típica comienza cuando el usuario accede a una ruta definida en un controlador, que comprueba la autenticación y los permisos del rol mediante decoradores y Flask-Login.
A continuación, el controlador interactúa con los modelos para consultar o actualizar la base de datos y finalmente devuelve una respuesta HTML renderizada con Jinja o una redirección a otra vista, según el caso de uso.

La lógica específica del mando SNES se implementa en ficheros JavaScript situados en \texttt{src/static/js}.
Estos scripts emplean la Gamepad API del navegador para capturar las entradas del mando y traducirlas a acciones de navegación, selección o retroceso dentro de la interfaz del paciente, manteniendo el foco visual sobre el elemento actualmente activo.

Para extender la aplicación, un desarrollador puede:

\begin{itemize}
  \item Añadir nuevos modelos o campos en \texttt{src/modelos}, actualizando el esquema de la base de datos y el diccionario de datos.
  \item Incorporar rutas adicionales en los controladores existentes o crear nuevos \textit{blueprints} para funcionalidades específicas.
  \item Crear o modificar plantillas en \texttt{src/vistas} para adaptar la interfaz a nuevos flujos de usuario.
  \item Definir nuevos formularios en \texttt{src/forms.py} con campos y validaciones acordes a los requisitos.
\end{itemize}

Esta organización permite localizar de forma rápida dónde debe introducirse cada cambio y reduce el riesgo de mezclar lógica de presentación con acceso a datos.

\section{Compilación, instalación y ejecución del proyecto}

Para ejecutar TerapiTrack en un entorno local se recomienda el siguiente procedimiento:

\begin{enumerate}
  \item \textbf{Clonar el repositorio} desde GitHub:
  \begin{verbatim}
  git clone https://github.com/lanchares/TerapiTrack.git
  cd TerapiTrack
  \end{verbatim}
  \item \textbf{Crear y activar un entorno virtual} de Python:
  \begin{itemize}
    \item Windows:
    \begin{verbatim}
    python -m venv venv
    venv\Scripts\activate
    \end{verbatim}
    \item Linux/Mac:
    \begin{verbatim}
    python -m venv venv
    source venv/bin/activate
    \end{verbatim}
  \end{itemize}
  \item \textbf{Instalar las dependencias} del proyecto:
  \begin{verbatim}
  pip install -r requirements.txt
  \end{verbatim}
  \item \textbf{Configurar las variables de entorno} creando un fichero \texttt{.env} en la raíz del proyecto con, al menos:
  \begin{verbatim}
  SECRET_KEY=tu_clave_secreta_aqui
  DATABASE_URL=sqlite:///TerapiTrack.db
  CLOUDINARY_CLOUD_NAME=tu_cloud_name
  CLOUDINARY_API_KEY=tu_api_key
  CLOUDINARY_API_SECRET=tu_api_secret
  \end{verbatim}
  \item \textbf{Inicializar la base de datos con datos de prueba}:
  \begin{verbatim}
  python poblar_bd.py
  \end{verbatim}
  Este script crea usuarios de ejemplo (administrador, profesionales y pacientes), así como ejercicios, sesiones y evaluaciones de prueba.
  \item \textbf{Ejecutar la aplicación} en modo desarrollo:
  \begin{verbatim}
  python app.py
  \end{verbatim}
  \item \textbf{Acceder a la aplicación} desde el navegador:
  \begin{verbatim}
  http://localhost:5000
  \end{verbatim}
\end{enumerate}

Para el despliegue en producción, el proyecto incluye un \texttt{Procfile} preparado para Heroku y está configurado para utilizar PostgreSQL como motor de base de datos en lugar de SQLite.
La configuración concreta del entorno (URL de la base de datos, claves de Cloudinary, etc.) se define mediante variables de entorno en la plataforma de despliegue.

\subsection{Despliegue en Heroku}

Para el despliegue en producción se ha utilizado Heroku como plataforma PaaS.
El repositorio está configurado con un \texttt{Procfile} de tipo \texttt{web} y un fichero \texttt{runtime.txt} que fija la versión de Python~\cite{HerokuDocs}.
La Figura~\ref{fig:heroku-deploy} muestra un ejemplo de despliegue desde la rama \texttt{Pruebas} a la aplicación \texttt{terapitrack-tfg}, donde se instalan las dependencias listadas en \texttt{requirements.txt} y se lanza el proceso \texttt{gunicorn}.

Una vez desplegada la aplicación, las variables de entorno (\texttt{SECRET\_KEY}, \texttt{DATABASE\_URL}, \texttt{CLOUDINARY\_*}) se configuran mediante \texttt{heroku config:set}.
Además, se ha añadido el addon \texttt{heroku-postgresql} para trabajar con PostgreSQL en producción, como se ilustra en la Figura~\ref{fig:heroku-postgres}, y se han realizado pruebas de inspección remota de ficheros y vídeos en el directorio \texttt{src/static/videos/} mediante el comando \texttt{heroku run} (Figura~\ref{fig:heroku-videos})~\cite{HerokuDocs}.

\begin{figure}[H]
  \centering
  \includegraphics[width=\textwidth]{img/heroku_deploy.png}
  \caption{Despliegue de la aplicación en Heroku e instalación de dependencias desde \texttt{requirements.txt}.}
  \label{fig:heroku-deploy}
\end{figure}

\begin{figure}[H]
  \centering
  \includegraphics[width=\textwidth]{img/heroku_postgres.png}
  \caption{Creación del addon \texttt{heroku-postgresql} y configuración de variables de entorno en Heroku.}
  \label{fig:heroku-postgres}
\end{figure}

\begin{figure}[H]
  \centering
  \includegraphics[width=\textwidth]{img/heroku_videos.png}
  \caption{Listado de vídeos de ejercicios en el directorio \texttt{src/static/videos} mediante \texttt{heroku run}.}
  \label{fig:heroku-videos}
\end{figure}


\section{Pruebas del sistema}

Las pruebas automatizadas de TerapiTrack se han implementado con Pytest, combinando pruebas unitarias sobre los modelos y pruebas de integración sobre los controladores más relevantes~\cite{PytestDocs}.
El objetivo principal es verificar que la lógica de negocio funciona según lo esperado y que los cambios en el código no introducen regresiones en los flujos básicos del sistema.

La ejecución de la batería completa de pruebas puede realizarse con el siguiente comando:

\begin{verbatim}
pytest --cov=src --cov-report=html
\end{verbatim}

Con esta orden se genera un informe de cobertura en la carpeta \texttt{htmlcov}, que puede abrirse en el navegador para analizar qué partes del código están cubiertas por pruebas y cuáles convendría reforzar.
En la versión final del proyecto se han definido 190 tests y se ha alcanzado una cobertura global del 99\%, con todos los modelos y las funciones auxiliares al 100\% y los controladores principales (administrador, profesional, paciente y autenticación) en torno al 99\,\% de líneas cubiertas~\cite{GithubTerapitrack}.
La Figura~\ref{fig:coverage-html} recoge el informe de cobertura generado por \texttt{coverage.py}.

\begin{figure}[H]
  \centering
  \includegraphics[width=\textwidth]{img/cobertura_html.png}
  \caption{Informe HTML de cobertura con 99\,\% de líneas cubiertas en el proyecto.}
  \label{fig:coverage-html}
\end{figure}

\subsection{Pruebas manuales con \texttt{flask shell}}

Además de las pruebas automáticas, se utilizó \texttt{flask shell} para comprobar manualmente la integridad del modelo de datos~\cite{GithubTerapitrack}.
A partir de una base de datos limpia se crearon un usuario, un paciente y un profesional, verificando las relaciones inversas \texttt{usuario.paciente} y \texttt{usuario.profesional}.
A continuación se registró un ejercicio y se asoció al profesional mediante la tabla intermedia \texttt{Ejercicio\_Profesional}, comprobando que los métodos auxiliares \texttt{to\_dict} y \texttt{total\_pacientes} devolvían los campos esperados.

Posteriormente se creó una sesión terapéutica, se añadieron ejercicios a través de \texttt{Ejercicio\_Sesion} y se verificó que los métodos \texttt{duracion\_legible} y \texttt{es\_pendiente} devolvían información coherente con los datos almacenados.
Finalmente se añadió una evaluación y un vídeo de respuesta para esa sesión, comprobando los métodos \texttt{es\_aprobado} y \texttt{resumen}, así como la ruta y la fecha de expiración del vídeo en la tabla \texttt{VideoRespuesta}.
Estos pasos se resumen en la Figura~\ref{fig:flask-shell-pruebas}.

\begin{figure}[H]
  \centering
  \includegraphics[width=\textwidth]{img/prueba_flask_shell.png}
  \caption{Pruebas manuales en \texttt{flask shell} creando usuarios, sesiones, evaluaciones y vídeos de respuesta.}
  \label{fig:flask-shell-pruebas}
\end{figure}

\subsection{Pruebas SQL con DB Browser for SQLite}

Como complemento a las pruebas anteriores, se ejecutaron sentencias SQL directamente sobre la base de datos SQLite utilizando DB Browser for SQLite~\cite{DBBrowser}.
El objetivo era validar operaciones típicas de mantenimiento (actualizaciones y consultas) y comprobar que las restricciones de clave primaria y foránea no impedían los cambios habituales.

En primer lugar se actualizó el nombre y los apellidos de un usuario mediante una sentencia \texttt{UPDATE} y se verificó el resultado con una consulta \texttt{SELECT} sobre la tabla \texttt{Usuario}.
De forma similar, se modificó el estado de una sesión de \texttt{PENDIENTE} a \texttt{COMPLETADA} y se actualizó la fecha de asignación en la tabla \texttt{Sesion}.
Por último, se cambió la puntuación y los comentarios de una evaluación específica en la tabla \texttt{Evaluacion}, comprobando que los nuevos valores se almacenaban correctamente y que la fecha de evaluación se mantenía consistente.
La Figura~\ref{fig:dbbrowser-actualizaciones} muestra uno de estos ejemplos.

\begin{figure}[H]
  \centering
  \includegraphics[width=\textwidth]{img/dbbrowser_actualizaciones.png}
  \caption{Actualización de una evaluación y verificación del resultado mediante consulta SQL en DB Browser for SQLite.}
  \label{fig:dbbrowser-actualizaciones}
\end{figure}

\begin{figure}[H]
  \centering
  \includegraphics[width=\textwidth]{img/dbbrowser_borrados.png}
  \caption{Borrado de una relación N:M y verificación mediante consulta SQL en DB Browser for SQLite.}
  \label{fig:dbbrowser-borrados}
\end{figure}

\section{Pantallas de autenticación y perfil}

Las Figuras~\ref{fig:perfil} a \ref{fig:recuperar-contraseña} recogen las principales pantallas de autenticación y gestión del perfil de usuario.

\begin{figure}[H]
  \centering
  \includegraphics[width=\textwidth]{img/perfil.png}
  \caption{Vista de perfil del usuario autenticado con sus datos básicos.}
  \label{fig:perfil}
\end{figure}

\begin{figure}[H]
  \centering
  \includegraphics[width=\textwidth]{img/cambiar_contraseña.png}
  \caption{Formulario para cambiar la contraseña desde el perfil de usuario.}
  \label{fig:cambiar-contraseña}
\end{figure}

\begin{figure}[H]
  \centering
  \includegraphics[width=\textwidth]{img/recuperar_contraseña.png}
  \caption{Pantalla de recuperación de contraseña mediante correo electrónico.}
  \label{fig:recuperar-contraseña}
\end{figure}

\section{Pantallas del panel de administración}

En esta sección se incluyen capturas adicionales del panel de administración de TerapiTrack que complementan la descripción general realizada en la memoria~\cite{GithubTerapitrack}.
Las Figuras~\ref{fig:admin-crear-usuario} a \ref{fig:admin-desvincular} permiten apreciar con más detalle las operaciones de alta, consulta y edición de usuarios, así como las estadísticas globales del sistema.

\begin{figure}[H]
  \centering
  \includegraphics[width=\textwidth]{img/admin_crear_usuario.png}
  \caption{Formulario de creación de un nuevo usuario administrador.}
  \label{fig:admin-crear-usuario}
\end{figure}

\begin{figure}[H]
  \centering
  \includegraphics[width=\textwidth]{img/admin_ver_usuario.png}
  \caption{Vista de detalle de un usuario administrador con sus datos principales.}
  \label{fig:admin-ver-usuario}
\end{figure}

\begin{figure}[H]
  \centering
  \includegraphics[width=\textwidth]{img/admin_editar_usuario.png}
  \caption{Pantalla de edición de los datos de un profesional sanitario.}
  \label{fig:admin-editar-usuario}
\end{figure}

\begin{figure}[H]
  \centering
  \includegraphics[width=\textwidth]{img/admin_estadisticas.png}
  \caption{Panel de estadísticas del sistema con distribución de roles y registros por mes.}
  \label{fig:admin-estadisticas}
\end{figure}

\begin{figure}[H]
  \centering
  \includegraphics[width=0.5\textwidth]{img/admin_desvincular.png}
  \caption{Diálogo de confirmación para la desvinculación de un paciente y su profesional.}
  \label{fig:admin-desvincular}
\end{figure}

\section{Pantallas del panel del profesional}

En esta sección se recogen capturas adicionales del área del profesional que complementan la descripción resumida incluida en el capítulo de aspectos relevantes~\cite{GithubTerapitrack}.
Las Figuras~\ref{fig:prof-ejercicios} a \ref{fig:prof-ver-evaluacion} muestran con más detalle la biblioteca de ejercicios, la creación de nuevos recursos y las vistas de evaluación.

\begin{figure}[H]
  \centering
  \includegraphics[width=\textwidth]{img/profesional_ejercicios.png}
  \caption{Biblioteca de ejercicios terapéuticos con filtros por tipo, duración y buscador por nombre.}
  \label{fig:prof-ejercicios}
\end{figure}

\begin{figure}[H]
  \centering
  \includegraphics[width=\textwidth]{img/profesional_crear_new_ejercicio.png}
  \caption{Formulario para crear un nuevo ejercicio con nombre, descripción, tipo y vídeo demostrativo.}
  \label{fig:prof-crear-ejercicio}
\end{figure}

\begin{figure}[H]
  \centering
  \includegraphics[width=\textwidth]{img/profesional_ver_sesion.png}
  \caption{Detalle de una sesión programada con la información del paciente y los ejercicios que la componen.}
  \label{fig:prof-ver-sesion}
\end{figure}

\begin{figure}[H]
  \centering
  \includegraphics[width=\textwidth]{img/profesional_ejecutar_sesion.png}
  \caption{Ejecución de una sesión con el vídeo en directo del paciente y el listado de ejercicios pendientes.}
  \label{fig:prof-ejecutar-sesion}
\end{figure}

\begin{figure}[H]
  \centering
  \includegraphics[width=\textwidth]{img/profesional_evaluar_sesion.png}
  \caption{Pantalla de evaluación de una sesión completada, con acceso a los vídeos de cada ejercicio.}
  \label{fig:prof-evaluar-sesion}
\end{figure}

\begin{figure}[H]
  \centering
  \includegraphics[width=\textwidth]{img/profesional_ver_evaluacion.png}
  \caption{Vista de una evaluación ya registrada con la puntuación final y los comentarios del profesional.}
  \label{fig:prof-ver-evaluacion}
\end{figure}

\section{Pantallas del panel del paciente}

En esta sección se incluyen las capturas del área del paciente que complementan la descripción general realizada en la memoria~\cite{GithubTerapitrack}.
Las Figuras~\ref{fig:paciente-dashboard} a \ref{fig:paciente-ejecutar-sesion} muestran las principales vistas disponibles para el paciente, incluyendo el panel inicial, la gestión de sesiones, el listado de ejercicios, el seguimiento de progreso, la ayuda sobre el mando SNES y la ejecución de una sesión.

\begin{figure}[H]
  \centering
  \includegraphics[width=\textwidth]{img/paciente_dashboard.png}
  \caption{Panel principal del paciente con accesos a ejercicios, sesiones, progreso y ayuda.}
  \label{fig:paciente-dashboard}
\end{figure}

\begin{figure}[H]
  \centering
  \includegraphics[width=\textwidth]{img/paciente_mis_sesiones.png}
  \caption{Pantalla \emph{Mis sesiones} con calendario y detalle de la sesión seleccionada.}
  \label{fig:paciente-mis-sesiones}
\end{figure}

\begin{figure}[H]
  \centering
  \includegraphics[width=\textwidth]{img/paciente_ejercicios.png}
  \caption{Listado \emph{Mis ejercicios} con tarjetas de ejercicios y navegación mediante mando SNES.}
  \label{fig:paciente-ejercicios}
\end{figure}

\begin{figure}[H]
  \centering
  \includegraphics[width=\textwidth]{img/paciente_progreso.png}
  \caption{Vista \emph{Mi progreso} con indicadores numéricos y gráfica de evolución.}
  \label{fig:paciente-progreso}
\end{figure}

\begin{figure}[H]
  \centering
  \includegraphics[width=\textwidth]{img/paciente_ayuda.png}
  \caption{Guía de uso del mando SNES con la función de cada botón.}
  \label{fig:paciente-ayuda}
\end{figure}

\begin{figure}[H]
  \centering
  \includegraphics[width=\textwidth]{img/paciente_ejecutar_sesion.png}
  \caption{Ejecución de una sesión desde el punto de vista del paciente, con su vídeo en directo.}
  \label{fig:paciente-ejecutar-sesion}
\end{figure}
