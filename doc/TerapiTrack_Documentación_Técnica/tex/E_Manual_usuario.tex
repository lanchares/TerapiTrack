\apendice{Documentación de usuario}

\section{Introducción}

Este anexo describe el uso de \textit{TerapiTrack} desde el punto de vista de los distintos perfiles de usuario (administrador, profesional sanitario y paciente). El objetivo es explicar qué necesita cada usuario para acceder al sistema, cómo se realiza la instalación básica en un entorno local de pruebas y cuáles son los pasos principales para utilizar la aplicación en el día a día.

\section{Requisitos de usuarios}

\subsection*{Requisitos técnicos}

Para utilizar \textit{TerapiTrack} se recomienda disponer de:

\begin{itemize}
  \item Un ordenador de sobremesa o portátil con conexión a Internet.
  \item Un navegador web moderno (Chrome, Firefox, Edge o similar), actualizado a una versión reciente.
  \item Una cámara web funcional, necesaria para grabar la ejecución de los ejercicios.
  \item Una conexión a Internet estable, suficiente para reproducir y subir vídeos cortos sin cortes frecuentes.
  \item Opcionalmente, un mando USB tipo SNES compatible con la Gamepad API del navegador, recomendado para pacientes con dificultades motoras finas.
\end{itemize}

\subsection*{Perfiles de usuario}

En el sistema existen tres tipos de usuario:

\begin{itemize}
  \item \textbf{Administrador}: Persona responsable de dar de alta usuarios, asignar roles, vincular pacientes con profesionales y configurar parámetros generales del sistema.
  \item \textbf{Profesional sanitario}: Médico, terapeuta, psicólogo o enfermero que crea ejercicios, programa sesiones de rehabilitación y evalúa las grabaciones de sus pacientes.
  \item \textbf{Paciente}: Usuario final que accede a las sesiones programadas, realiza los ejercicios guiados por vídeo y consulta sus evaluaciones y gráficos de progreso.
\end{itemize}

\section{Instalación}

En un entorno real, \textit{TerapiTrack} se desplegaría en un servidor centralizado y los usuarios accederían a través del navegador, sin necesidad de instalación local. Para realizar pruebas en un entorno de desarrollo, es posible ejecutar la aplicación en un equipo propio siguiendo estos pasos básicos:

\begin{enumerate}
  \item Clonar el repositorio del proyecto e instalar las dependencias necesarias de Python.
  \item Configurar las variables de entorno mínimas (clave secreta, cadena de conexión a la base de datos y credenciales de Cloudinary).
  \item Inicializar la base de datos con el script de población, que crea usuarios de ejemplo (administrador, profesionales y pacientes), así como ejercicios, sesiones y evaluaciones de prueba.
  \item Ejecutar la aplicación y acceder a la dirección \texttt{http://localhost:5000} desde el navegador.
\end{enumerate}

En este entorno de pruebas se proporcionan varias cuentas de ejemplo, entre ellas un usuario administrador (\texttt{admin@terapitrack.com}) y diferentes perfiles de profesional y paciente, que permiten recorrer los flujos principales sin necesidad de registrar usuarios nuevos.

\section{Manual del usuario}

\subsection*{Acceso al sistema}

\begin{itemize}
  \item Abra el navegador y acceda a la URL proporcionada (por ejemplo, \texttt{http://localhost:5000} en un entorno local).
  \item Introduzca su correo electrónico y contraseña en el formulario de inicio de sesión.
  \item Tras la autenticación, el sistema le dirigirá automáticamente al panel correspondiente a su rol (administrador, profesional o paciente).
\end{itemize}

\subsection*{Uso para administradores}

Desde el panel de administración el usuario puede:

\begin{itemize}
  \item Crear nuevos usuarios indicando nombre, apellidos, correo, contraseña y rol.
  \item Activar o desactivar cuentas existentes cuando sea necesario.
  \item Vincular pacientes con profesionales mediante una interfaz de asignación sencilla.
  \item Configurar parámetros globales del sistema, como la política de retención de vídeos o límites de almacenamiento.
\end{itemize}

El panel muestra además un resumen de estadísticas básicas del sistema, como el número de usuarios registrados o el volumen de sesiones programadas.

\subsection*{Uso para profesionales sanitarios}

En el área de profesional, el flujo típico de trabajo es el siguiente:

\begin{itemize}
  \item Consultar la lista de pacientes asignados y sus sesiones pendientes o completadas.
  \item Crear nuevos ejercicios terapéuticos subiendo un vídeo demostrativo y rellenando los datos básicos (nombre, descripción, tipo, duración).
  \item Definir sesiones terapéuticas combinando varios ejercicios y asignarlas a pacientes concretos, indicando fecha y hora programadas.
  \item Revisar las sesiones completadas, reproducir los vídeos grabados por el paciente y registrar evaluaciones con una puntuación numérica y comentarios.
  \item Visualizar gráficos de evolución temporal para analizar el progreso de cada paciente a lo largo de varias sesiones.
\end{itemize}

\subsection*{Uso para pacientes}

El área de paciente está simplificada y adaptada para facilitar su manejo:

\begin{itemize}
  \item En el \textbf{dashboard} inicial se muestran las sesiones pendientes en un calendario de varias semanas, junto con un acceso rápido a las evaluaciones recientes y a los gráficos de evolución.
  \item Al seleccionar una sesión pendiente, la aplicación muestra una pantalla dividida en dos partes: el vídeo demostrativo del ejercicio y la imagen de la cámara del paciente. Durante la sesión, el sistema graba automáticamente la ejecución de cada ejercicio.
  \item Una vez completada la sesión, el paciente puede consultar las evaluaciones que el profesional haya registrado, incluyendo puntuaciones y comentarios, y revisar gráficos que resumen su progresión en el tiempo.
\end{itemize}

\subsection*{Uso del mando SNES}

Para los pacientes que utilizan un mando tipo SNES, la interfaz permite navegar sin necesidad de ratón:

\begin{itemize}
  \item La cruceta se emplea para moverse entre los distintos elementos de la pantalla (botones, enlaces, tarjetas de sesión).
  \item El botón \textbf{A} se utiliza para seleccionar y confirmar acciones.
  \item El botón \textbf{B} se reserva para acciones secundarias cuando están disponibles.
  \item El botón \textbf{Y} permite retroceder o cancelar en determinadas pantallas.
\end{itemize}

El mando está activo en las principales pantallas del área de paciente (inicio, listado de sesiones, ejecución de ejercicios y vista de progreso), de modo que el usuario pueda completar una sesión completa sin necesidad de utilizar el teclado o el ratón.
Este esquema de control busca reducir al mínimo el número de movimientos finos necesarios y mantener siempre visibles las opciones disponibles en pantalla.

