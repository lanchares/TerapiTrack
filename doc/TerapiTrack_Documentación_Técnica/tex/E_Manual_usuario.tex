\apendice{Documentación de usuario}

\section{Introducción}

Este anexo describe el uso de TerapiTrack desde el punto de vista de los distintos perfiles de usuario (administrador, profesional sanitario y paciente).
El objetivo es explicar qué necesita cada usuario para acceder al sistema, cómo se realiza la instalación básica en un entorno local de pruebas y cuáles son los pasos principales para utilizar la aplicación en el día a día.

\section{Requisitos de usuarios}

\subsection*{Requisitos técnicos}

Para utilizar TerapiTrack se recomienda disponer de:

\begin{itemize}
  \item Un ordenador de sobremesa o portátil con conexión a Internet.
  \item Un navegador web moderno (Chrome, Firefox, Edge o similar), actualizado a una versión reciente.
  \item Una cámara web funcional, necesaria para grabar la ejecución de los ejercicios.
  \item Una conexión a Internet estable, suficiente para reproducir y subir vídeos cortos sin cortes frecuentes.
  \item Opcionalmente, un mando USB tipo SNES compatible con la Gamepad API del navegador, recomendado para pacientes con dificultades motoras finas.
\end{itemize}

\subsection*{Perfiles de usuario}

En el sistema existen tres tipos de usuario:

\begin{itemize}
  \item \textbf{Administrador}: Persona responsable de dar de alta usuarios, asignar roles, vincular pacientes con profesionales y configurar parámetros generales del sistema.
  \item \textbf{Profesional sanitario}: Médico, terapeuta, psicólogo o enfermero que crea ejercicios, programa sesiones de rehabilitación y evalúa las grabaciones de sus pacientes.
  \item \textbf{Paciente}: Usuario final que accede a las sesiones programadas, realiza los ejercicios guiados por vídeo y consulta sus evaluaciones y gráficos de progreso.
\end{itemize}

\section{Instalación}

En un entorno real, TerapiTrack se desplegaría en un servidor centralizado y los usuarios accederían a través del navegador, sin necesidad de instalación local.
Para realizar pruebas en un entorno de desarrollo, es posible ejecutar la aplicación en un equipo propio siguiendo estos pasos básicos:

\begin{enumerate}
  \item Clonar el repositorio del proyecto e instalar las dependencias necesarias de Python.
  \item Configurar las variables de entorno mínimas (clave secreta, cadena de conexión a la base de datos y credenciales de Cloudinary).
  \item Inicializar la base de datos con el script de población, que crea usuarios de ejemplo (administrador, profesionales y pacientes), así como ejercicios, sesiones y evaluaciones de prueba.
  \item Ejecutar la aplicación y acceder a la dirección \texttt{http://localhost:5000} desde el navegador.
\end{enumerate}

En este entorno de pruebas se proporcionan varias cuentas de ejemplo, entre ellas un usuario administrador (\texttt{admin@terapitrack.com}) y diferentes perfiles de profesional y paciente, que permiten recorrer los flujos principales sin necesidad de registrar usuarios nuevos.

\section{Manual del usuario}

\subsection*{Acceso al sistema}

Tras iniciar sesión, el usuario puede acceder a su perfil para revisar sus datos básicos y, en su caso, modificarlos (Figura~\ref{fig:perfil}).

\begin{figure}[H]
  \centering
  \includegraphics[width=\textwidth]{img/perfil.png}
  \caption{Vista de perfil del usuario autenticado con sus datos básicos.}
  \label{fig:perfil}
\end{figure}

Desde esta pantalla también es posible cambiar la contraseña (Figura~\ref{fig:cambiar-contraseña}) o iniciar el proceso de recuperación mediante correo electrónico (Figura~\ref{fig:recuperar-contraseña}).

\begin{figure}[H]
  \centering
  \includegraphics[width=\textwidth]{img/cambiar_contraseña.png}
  \caption{Formulario para cambiar la contraseña desde el perfil de usuario.}
  \label{fig:cambiar-contraseña}
\end{figure}

\begin{figure}[H]
  \centering
  \includegraphics[width=0.7\textwidth]{img/recuperar_contraseña.png}
  \caption{Pantalla de recuperación de contraseña mediante correo electrónico.}
  \label{fig:recuperar-contraseña}
\end{figure}

\subsection*{Uso para administradores}
El panel de administración permite gestionar usuarios y las vinculaciones paciente–profesional, además de consultar estadísticas básicas del sistema.

La Figura~\ref{fig:admin-usuarios} muestra la pantalla de gestión de usuarios, desde la que el administrador puede buscar y filtrar usuarios por nombre, rol y estado, y acceder a las acciones de ver, editar o desactivar cada cuenta.

\begin{figure}[H]
  \centering
  \includegraphics[width=\textwidth]{img/admin_usuarios.png}
  \caption{Listado de usuarios con filtros y acciones rápidas.}
  \label{fig:admin-usuarios}
\end{figure}

La creación de un nuevo usuario se realiza mediante el formulario de la Figura~\ref{fig:admin-crear-usuario}, donde se introducen los datos personales y se selecciona el rol asignado.

\begin{figure}[H]
  \centering
  \includegraphics[width=\textwidth]{img/admin_crear_usuario.png}
  \caption{Formulario para dar de alta un nuevo usuario.}
  \label{fig:admin-crear-usuario}
\end{figure}

La vista de detalle de un usuario (Figura~\ref{fig:admin-ver-usuario}) permite consultar su información general, la condición médica en el caso de los pacientes y los profesionales asignados.

\begin{figure}[H]
  \centering
  \includegraphics[width=\textwidth]{img/admin_ver_usuario.png}
  \caption{Detalle de un usuario con su información general y específica.}
  \label{fig:admin-ver-usuario}
\end{figure}

Cuando es necesario modificar algún dato, el administrador utiliza la pantalla de edición mostrada en la Figura~\ref{fig:admin-editar-usuario}, donde puede actualizar tanto la información básica como la condición médica y las notas del paciente.

\begin{figure}[H]
  \centering
  \includegraphics[width=\textwidth]{img/admin_editar_usuario.png}
  \caption{Edición de los datos de un usuario paciente.}
  \label{fig:admin-editar-usuario}
\end{figure}

La gestión de las relaciones terapéuticas se realiza desde la lista de vinculaciones paciente-profesional (Figura~\ref{fig:admin-vinculaciones}), que permite filtrar por paciente, profesional y rango de fechas.
Tambien se pueden desvincular relaciones.

\begin{figure}[H]
  \centering
  \includegraphics[width=\textwidth]{img/admin_vinculaciones.png}
  \caption{Listado de vinculaciones paciente-profesional con filtros y acciones.}
  \label{fig:admin-vinculaciones}
\end{figure}

Para crear una nueva vinculación se utiliza el formulario de la Figura~\ref{fig:admin-vincular}, donde se seleccionan el paciente y el profesional correspondientes.

\begin{figure}[H]
  \centering
  \includegraphics[width=\textwidth]{img/admin_vincular.png}
  \caption{Formulario para vincular un paciente con un profesional.}
  \label{fig:admin-vincular}
\end{figure}

Por último, el administrador dispone de un panel de estadísticas (Figura~\ref{fig:admin-estadisticas}) que resume la distribución de roles y la evolución de registros por mes.

\begin{figure}[H]
  \centering
  \includegraphics[width=\textwidth]{img/admin_estadisticas.png}
  \caption{Panel de estadísticas del sistema con distribución de roles y registros por mes.}
  \label{fig:admin-estadisticas}
\end{figure}

\subsection*{Uso para profesionales sanitarios}

El panel del profesional reúne las herramientas necesarias para gestionar pacientes, ejercicios y sesiones terapéuticas.

La Figura~\ref{fig:prof-dashboard} muestra el panel inicial del profesional, donde se resumen las sesiones pendientes y se ofrecen accesos rápidos a pacientes, ejercicios y evaluaciones.

\begin{figure}[H]
  \centering
  \includegraphics[width=\textwidth]{img/profesional_dashboard.png}
  \caption{Panel inicial del profesional con accesos rápidos a las principales secciones.}
  \label{fig:prof-dashboard}
\end{figure}

En la lista de pacientes (Figura~\ref{fig:prof-pacientes}) el profesional puede ver únicamente los pacientes que tiene asignados y acceder al detalle de sus sesiones y evaluaciones.

\begin{figure}[H]
  \centering
  \includegraphics[width=\textwidth]{img/profesional_pacientes.png}
  \caption{Listado de pacientes asignados al profesional.}
  \label{fig:prof-pacientes}
\end{figure}

La biblioteca de ejercicios (Figura~\ref{fig:prof-ejercicios}) permite buscar y filtrar ejercicios existentes por tipo, duración o condición médica, y reproducir el vídeo demostrativo asociado.

\begin{figure}[H]
  \centering
  \includegraphics[width=\textwidth]{img/profesional_ejercicios.png}
  \caption{Biblioteca de ejercicios terapéuticos con filtros y buscador.}
  \label{fig:prof-ejercicios}
\end{figure}

Para crear un nuevo ejercicio se utiliza el formulario de la Figura~\ref{fig:prof-crear-ejercicio}, donde el profesional sube el vídeo demostrativo e introduce el nombre, la descripción y el tipo.

\begin{figure}[H]
  \centering
  \includegraphics[width=\textwidth]{img/profesional_crear_new_ejercicio.png}
  \caption{Creación de un nuevo ejercicio con su vídeo demostrativo.}
  \label{fig:prof-crear-ejercicio}
\end{figure}

La planificación de sesiones comienza en el formulario de creación de sesión (Figura~\ref{fig:prof-crear-sesion}), en el que se seleccionan los ejercicios que la componen, el paciente y la fecha.

\begin{figure}[H]
  \centering
  \includegraphics[width=\textwidth]{img/profesional_crear_new_sesion.png}
  \caption{Definición de una sesión terapéutica combinando varios ejercicios.}
  \label{fig:prof-crear-sesion}
\end{figure}

Una vez creada, la vista de sesión (Figura~\ref{fig:prof-ver-sesion}) permite revisar los ejercicios incluidos y más informacion de la sesión.

\begin{figure}[H]
  \centering
  \includegraphics[width=\textwidth]{img/profesional_ver_sesion.png}
  \caption{Detalle de una sesión con los ejercicios que la componen.}
  \label{fig:prof-ver-sesion}
\end{figure}

Durante la terapia guiada en directo, la pantalla de ejecución de sesión (Figura~\ref{fig:prof-ejecutar-sesion}) muestra una simulación de grabación de la camara del dispositivo, junto al listado de ejercicios pendientes, permitiendo al profesional avanzar de un ejercicio a otro.

\begin{figure}[H]
  \centering
  \includegraphics[width=\textwidth]{img/profesional_ejecutar_sesion.png}
  \caption{Ejecución de una sesión con el vídeo en directo del paciente y los ejercicios pendientes.}
  \label{fig:prof-ejecutar-sesion}
\end{figure}

Tras la sesión, la pantalla de evaluación de sesión (Figura~\ref{fig:prof-evaluar-sesion}) recoge los vídeos de respuesta de cada ejercicio, desde los que se accede a la evaluación individual de cada uno.

\begin{figure}[H]
  \centering
  \includegraphics[width=\textwidth]{img/profesional_evaluar_sesion.png}
  \caption{Vista de evaluación de una sesión completada, con acceso a los vídeos de cada ejercicio.}
  \label{fig:prof-evaluar-sesion}
\end{figure}

Por último, la Figura~\ref{fig:prof-ver-evaluacion} muestra el detalle de una evaluación ya registrada, donde se consultan la puntuación final y los comentarios del profesional.

\begin{figure}[H]
  \centering
  \includegraphics[width=\textwidth]{img/profesional_ver_evaluacion.png}
  \caption{Detalle de una evaluación registrada para un ejercicio de la sesión.}
  \label{fig:prof-ver-evaluacion}
\end{figure}

\subsection*{Uso para pacientes}

El área de paciente está simplificada y adaptada para facilitar su manejo durante la rehabilitación en casa.

Desde la pantalla \emph{Mis sesiones} (Figura~\ref{fig:paciente-mis-sesiones}) el paciente visualiza en un calendario las sesiones programadas para las próximas semanas y puede seleccionar una de ellas para iniciarla.

\begin{figure}[H]
  \centering
  \includegraphics[width=\textwidth]{img/paciente_mis_sesiones.png}
  \caption{Pantalla \emph{Mis sesiones} con calendario y detalle de la sesión seleccionada.}
  \label{fig:paciente-mis-sesiones}
\end{figure}

La vista \emph{Mis ejercicios} (Figura~\ref{fig:paciente-ejercicios}) presenta los ejercicios disponibles en forma de tarjetas con imagen y título, permitiendo al paciente repasarlos y reproducir el vídeo demostrativo asociado.

\begin{figure}[H]
  \centering
  \includegraphics[width=\textwidth]{img/paciente_ejercicios.png}
  \caption{Listado \emph{Mis ejercicios} con tarjetas de ejercicios.}
  \label{fig:paciente-ejercicios}
\end{figure}

Durante la realización de una sesión guiada, la pantalla de ejecución (Figura~\ref{fig:paciente-ejecutar-sesion}) divide la interfaz en dos partes: a la izquierda se muestra el vídeo demostrativo del ejercicio y a la derecha la imagen captada por la cámara del paciente, que se graba automáticamente mientras dura cada ejercicio.

\begin{figure}[H]
  \centering
  \includegraphics[width=\textwidth]{img/paciente_ejecutar_sesion.png}
  \caption{Ejecución de una sesión desde el punto de vista del paciente, con su vídeo en directo.}
  \label{fig:paciente-ejecutar-sesion}
\end{figure}

Tras completar varias sesiones, la pantalla \emph{Mi progreso} (Figura~\ref{fig:paciente-progreso}) permite consultar las evaluaciones registradas por el profesional y visualizar la evolución mediante indicadores numéricos y un gráfico de tendencia.

\begin{figure}[H]
  \centering
  \includegraphics[width=\textwidth]{img/paciente_progreso.png}
  \caption{Vista \emph{Mi progreso} con indicadores y gráfica de evolución.}
  \label{fig:paciente-progreso}
\end{figure}

Por último, la sección de ayuda sobre el mando SNES (Figura~\ref{fig:paciente-ayuda}) explica qué acción realiza cada botón dentro de la aplicación, de modo que el paciente pueda navegar sin necesidad de utilizar el ratón.

\begin{figure}[H]
  \centering
  \includegraphics[width=\textwidth]{img/paciente_ayuda.png}
  \caption{Guía de uso del mando SNES con la función de cada botón.}
  \label{fig:paciente-ayuda}
\end{figure}


\subsection*{Uso del mando SNES}

Para los pacientes que utilizan un mando tipo SNES, la interfaz permite navegar sin necesidad de ratón.
La cruceta se emplea para moverse entre los distintos elementos de la pantalla (botones, enlaces, tarjetas de sesión), el botón \textbf{A} se utiliza para seleccionar y confirmar acciones, el botón \textbf{B} se reserva para acciones secundarias y el botón \textbf{Y} permite retroceder o cancelar en determinadas pantallas.

El mando está activo en todas las pantallas del área de paciente, de modo que el usuario pueda completar una sesión completa sin necesidad de utilizar el teclado o el ratón.
Este esquema de control busca reducir al mínimo el número de movimientos necesarios y mantener siempre visibles las opciones disponibles en pantalla.

