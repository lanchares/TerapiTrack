\capitulo{2}{Objetivos del proyecto}

Este capítulo recoge los objetivos que se persiguen con la realización del proyecto.
En primer lugar se presentan los objetivos funcionales, que describen qué se espera que ofrezca la aplicación a sus usuarios.
A continuación se detallan los objetivos técnicos, relacionados con la forma de construir y desplegar el sistema, y por último se recogen los objetivos personales planteados para el desarrollo del trabajo.

\section{Objetivos funcionales}

\begin{itemize}
    \item Gestionar diferentes perfiles de usuario (administrador, paciente y profesionales sanitarios) con permisos diferenciados según su rol.
    \item Permitir la vinculación de pacientes con los profesionales responsables de su seguimiento terapéutico.
    \item Ofrecer una biblioteca de ejercicios y recursos terapéuticos en formato audiovisual, organizada y filtrable según las necesidades de cada paciente.
    \item Facilitar la creación y programación de sesiones personalizadas de rehabilitación, formadas por combinaciones de ejercicios.
    \item Registrar la realización de las sesiones y asociar las grabaciones de los ejercicios al historial de cada paciente, para su posterior revisión y evaluación.
    \item Proporcionar a los pacientes un entorno sencillo donde puedan consultar sus sesiones programadas, realizar los ejercicios guiados por vídeo y revisar la evolución de sus evaluaciones.
    \item Asegurar la trazabilidad básica de la evolución terapéutica, ofreciendo a los profesionales una visión estructurada de las sesiones realizadas y de las puntuaciones registradas.
\end{itemize}

\section{Objetivos técnicos}

\begin{itemize}
    \item Construir una aplicación web modular que facilite el mantenimiento y la incorporación de nuevas funcionalidades en el futuro.
    \item Diseñar e implementar una base de datos relacional que modele usuarios, pacientes, profesionales, ejercicios, sesiones y evaluaciones, garantizando la integridad de los datos clínicos manejados.
    \item Desarrollar una interfaz web accesible y clara, especialmente pensada para pacientes con posibles dificultades motoras o cognitivas.
    \item Incorporar mecanismos de autenticación, control de acceso por roles y cifrado de contraseñas que contribuyan a la protección de la información sensible.
    \item Preparar la aplicación para su despliegue en un entorno en la nube, permitiendo el acceso remoto al sistema desde distintos dispositivos.
    \item Definir y ejecutar una batería de pruebas sobre los componentes principales del sistema (modelos, controladores y flujos de usuario) que ayude a detectar fallos tanto de diseño como de programación y a comprobar su correcto funcionamiento.
\end{itemize}

\section{Objetivos personales}

\begin{itemize}
    \item Aplicar de forma integrada los conocimientos adquiridos a lo largo del Grado en un proyecto completo, desde el análisis de requisitos hasta el despliegue de la aplicación.
    \item Profundizar en el desarrollo de aplicaciones web orientadas al ámbito sanitario y a la telerehabilitación.
    \item Mejorar la capacidad de trabajo autónomo, planificación y gestión del tiempo en un proyecto de larga duración.
    \item Adquirir experiencia práctica en el uso de herramientas profesionales de desarrollo, control de versiones y documentación técnica.
\end{itemize}
