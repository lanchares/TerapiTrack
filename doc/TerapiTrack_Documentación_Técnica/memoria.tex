\documentclass[a4paper,12pt,twoside]{memoir}

% Castellano
\usepackage[spanish,es-tabla]{babel}
\selectlanguage{spanish}
\usepackage[utf8]{inputenc}
\usepackage[T1]{fontenc}
\usepackage{lmodern} % Scalable font
\usepackage{microtype}
\usepackage{placeins}
\usepackage{silence}
\usepackage{float}
\WarningFilter{latexfont}{Font shape}
\WarningFilter{latexfont}{Some font shapes}

\RequirePackage{booktabs}
\RequirePackage[table]{xcolor}
\RequirePackage{xtab}
\RequirePackage{multirow}

% Links
\PassOptionsToPackage{hyphens}{url}\usepackage[colorlinks]{hyperref}
\hypersetup{
	allcolors = {red}
}
\pdfstringdefDisableCommands{\def\enskip{}} 

% Ecuaciones
\usepackage{amsmath}

% Rutas de fichero / paquete
\newcommand{\ruta}[1]{{\sffamily #1}}

% Párrafos
\nonzeroparskip

% Huérfanas y viudas
\widowpenalty100000
\clubpenalty100000

% Imágenes

% Comando para insertar una imagen en un lugar concreto.
% Los parámetros son:
% 1 --> Ruta absoluta/relativa de la figura
% 2 --> Texto a pie de figura
% 3 --> Tamaño en tanto por uno relativo al ancho de página
\usepackage{graphicx}
\newcommand{\imagen}[3]{
	\begin{figure}[!h]
		\centering
		\includegraphics[width=#3\textwidth]{#1}
		\caption{#2}\label{fig:#1}
	\end{figure}
	\FloatBarrier
}

% Comando para insertar una imagen sin posición.
% Los parámetros son:
% 1 --> Ruta absoluta/relativa de la figura
% 2 --> Texto a pie de figura
% 3 --> Tamaño en tanto por uno relativo al ancho de página
\newcommand{\imagenflotante}[3]{
	\begin{figure}
		\centering
		\includegraphics[width=#3\textwidth]{#1}
		\caption{#2}\label{fig:#1}
	\end{figure}
}

% El comando \figura nos permite insertar figuras comodamente, y utilizando
% siempre el mismo formato. Los parametros son:
% 1 --> Porcentaje del ancho de página que ocupará la figura (de 0 a 1)
% 2 --> Fichero de la imagen
% 3 --> Texto a pie de imagen
% 4 --> Etiqueta (label) para referencias
% 5 --> Opciones que queramos pasarle al \includegraphics
% 6 --> Opciones de posicionamiento a pasarle a \begin{figure}
\newcommand{\figuraConPosicion}[6]{%
  \setlength{\anchoFloat}{#1\textwidth}%
  \addtolength{\anchoFloat}{-4\fboxsep}%
  \setlength{\anchoFigura}{\anchoFloat}%
  \begin{figure}[#6]
    \begin{center}%
      \Ovalbox{%
        \begin{minipage}{\anchoFloat}%
          \begin{center}%
            \includegraphics[width=\anchoFigura,#5]{#2}%
            \caption{#3}%
            \label{#4}%
          \end{center}%
        \end{minipage}
      }%
    \end{center}%
  \end{figure}%
}

%
% Comando para incluir imágenes en formato apaisado (sin marco).
\newcommand{\figuraApaisadaSinMarco}[5]{%
  \begin{figure}%
    \begin{center}%
    \includegraphics[angle=90,height=#1\textheight,#5]{#2}%
    \caption{#3}%
    \label{#4}%
    \end{center}%
  \end{figure}%
}
% Para las tablas
\newcommand{\otoprule}{\midrule [\heavyrulewidth]}
%
% Nuevo comando para tablas pequeñas (menos de una página).
\newcommand{\tablaSmall}[5]{%
 \begin{table}
  \begin{center}
   \rowcolors {2}{gray!35}{}
   \begin{tabular}{#2}
    \toprule
    #4
    \otoprule
    #5
    \bottomrule
   \end{tabular}
   \caption{#1}
   \label{tabla:#3}
  \end{center}
 \end{table}
}

%
% Nuevo comando para tablas pequeñas (menos de una página).
\newcommand{\tablaSmallSinColores}[5]{%
 \begin{table}[H]
  \begin{center}
   \begin{tabular}{#2}
    \toprule
    #4
    \otoprule
    #5
    \bottomrule
   \end{tabular}
   \caption{#1}
   \label{tabla:#3}
  \end{center}
 \end{table}
}

\newcommand{\tablaApaisadaSmall}[5]{%
\begin{landscape}
  \begin{table}
   \begin{center}
    \rowcolors {2}{gray!35}{}
    \begin{tabular}{#2}
     \toprule
     #4
     \otoprule
     #5
     \bottomrule
    \end{tabular}
    \caption{#1}
    \label{tabla:#3}
   \end{center}
  \end{table}
\end{landscape}
}

%
% Nuevo comando para tablas grandes con cabecera y filas alternas coloreadas en gris.
\newcommand{\tabla}[6]{%
  \begin{center}
    \tablefirsthead{
      \toprule
      #5
      \otoprule
    }
    \tablehead{
      \multicolumn{#3}{l}{\small\sl continúa desde la página anterior}\\
      \toprule
      #5
      \otoprule
    }
    \tabletail{
      \hline
      \multicolumn{#3}{r}{\small\sl continúa en la página siguiente}\\
    }
    \tablelasttail{
      \hline
    }
    \bottomcaption{#1}
    \rowcolors {2}{gray!35}{}
    \begin{xtabular}{#2}
      #6
      \bottomrule
    \end{xtabular}
    \label{tabla:#4}
  \end{center}
}

%
% Nuevo comando para tablas grandes con cabecera.
\newcommand{\tablaSinColores}[6]{%
  \begin{center}
    \tablefirsthead{
      \toprule
      #5
      \otoprule
    }
    \tablehead{
      \multicolumn{#3}{l}{\small\sl continúa desde la página anterior}\\
      \toprule
      #5
      \otoprule
    }
    \tabletail{
      \hline
      \multicolumn{#3}{r}{\small\sl continúa en la página siguiente}\\
    }
    \tablelasttail{
      \hline
    }
    \bottomcaption{#1}
    \begin{xtabular}{#2}
      #6
      \bottomrule
    \end{xtabular}
    \label{tabla:#4}
  \end{center}
}

%
% Nuevo comando para tablas grandes sin cabecera.
\newcommand{\tablaSinCabecera}[5]{%
  \begin{center}
    \tablefirsthead{
      \toprule
    }
    \tablehead{
      \multicolumn{#3}{l}{\small\sl continúa desde la página anterior}\\
      \hline
    }
    \tabletail{
      \hline
      \multicolumn{#3}{r}{\small\sl continúa en la página siguiente}\\
    }
    \tablelasttail{
      \hline
    }
    \bottomcaption{#1}
  \begin{xtabular}{#2}
    #5
   \bottomrule
  \end{xtabular}
  \label{tabla:#4}
  \end{center}
}



\definecolor{cgoLight}{HTML}{EEEEEE}
\definecolor{cgoExtralight}{HTML}{FFFFFF}

%
% Nuevo comando para tablas grandes sin cabecera.
\newcommand{\tablaSinCabeceraConBandas}[5]{%
  \begin{center}
    \tablefirsthead{
      \toprule
    }
    \tablehead{
      \multicolumn{#3}{l}{\small\sl continúa desde la página anterior}\\
      \hline
    }
    \tabletail{
      \hline
      \multicolumn{#3}{r}{\small\sl continúa en la página siguiente}\\
    }
    \tablelasttail{
      \hline
    }
    \bottomcaption{#1}
    \rowcolors[]{1}{cgoExtralight}{cgoLight}

  \begin{xtabular}{#2}
    #5
   \bottomrule
  \end{xtabular}
  \label{tabla:#4}
  \end{center}
}



\graphicspath{ {./img/} }

% Capítulos
\chapterstyle{bianchi}
\newcommand{\capitulo}[2]{
	\setcounter{chapter}{#1}
	\setcounter{section}{0}
	\setcounter{figure}{0}
	\setcounter{table}{0}
	\chapter*{\thechapter.\enskip #2}
	\addcontentsline{toc}{chapter}{\thechapter.\enskip #2}
	\markboth{#2}{#2}
}

% Apéndices
\renewcommand{\appendixname}{Apéndice}
\renewcommand*\cftappendixname{\appendixname}

\newcommand{\apendice}[1]{
	%\renewcommand{\thechapter}{A}
	\chapter{#1}
}

\renewcommand*\cftappendixname{\appendixname\ }

% Formato de portada
\makeatletter
\usepackage{xcolor}
\newcommand{\tutor}[1]{\def\@tutor{#1}}
\newcommand{\course}[1]{\def\@course{#1}}
\definecolor{cpardoBox}{HTML}{E6E6FF}
\def\maketitle{
  \null
  \thispagestyle{empty}
  % Cabecera ----------------
\noindent\includegraphics[width=\textwidth]{cabecera}\vspace{1cm}%
  \vfill
  % Título proyecto y escudo informática ----------------
  \colorbox{cpardoBox}{%
    \begin{minipage}{.8\textwidth}
      \vspace{.5cm}\Large
      \begin{center}
      \textbf{TFG del Grado en Ingeniería Informática}\vspace{.6cm}\\
      \textbf{\LARGE\@title{}}
      \end{center}
      \vspace{.2cm}
    \end{minipage}

  }%
  \hfill\begin{minipage}{.20\textwidth}
    \includegraphics[width=\textwidth]{escudoInfor}
  \end{minipage}
  \vfill
  % Datos de alumno, curso y tutores ------------------
  \begin{center}%
  {%
    \noindent\LARGE
    Presentado por \@author{}\\ 
    en Universidad de Burgos --- \@date{}\\
    Tutores: \@tutor{}\\
  }%
  \end{center}%
  \null
  \cleardoublepage
  }
\makeatother

\newcommand{\nombre}{Alberto Lanchares Diez} %%% cambio de comando

% Datos de portada
\title{TerapiTrack: plataforma web de rehabilitación con registro de ejercicios en vídeo

}
\author{\nombre}
\tutor{José Luis Garrido Labrador y José Miguel Ramírez Sanz}
\date{\today}

\begin{document}

\maketitle


\newpage\null\thispagestyle{empty}\newpage


%%%%%%%%%%%%%%%%%%%%%%%%%%%%%%%%%%%%%%%%%%%%%%%%%%%%%%%%%%%%%%%%%%%%%%%%%%%%%%%%%%%%%%%%
\pagestyle{plain}


\noindent\includegraphics[width=\textwidth]{cabecera}\vspace{1cm}

\noindent D. José Luis Garrido Labrador y D. José Miguel Ramírez Sanz, profesores del departamento de Ingeniería Informática, área de Lenguajes y Sistemas Informáticos.

\noindent Exponen:

\noindent Que el alumno D. \nombre, con DNI 71362969H, ha realizado el Trabajo final de Grado en Ingeniería Informática titulado TerapiTrack: plataforma web de rehabilitación con registro de ejercicios en vídeo. 

\noindent Y que dicho trabajo ha sido realizado por el alumno bajo la dirección del que suscribe, en virtud de lo cual se autoriza su presentación y defensa.

\begin{center} %\large
En Burgos, {\large \today}
\end{center}

\vfill\vfill\vfill

% Author and supervisor
\begin{minipage}{0.45\textwidth}
\begin{flushleft} %\large
Vº. Bº. del Tutor:\\[2cm]
D. José Luis Garrido Labrador
\end{flushleft}
\end{minipage}
\hfill
\begin{minipage}{0.45\textwidth}
\begin{flushleft} %\large
Vº. Bº. del co-tutor:\\[2cm]
D. José Miguel Ramírez Sanz
\end{flushleft}
\end{minipage}
\hfill

\vfill

% para casos con solo un tutor comentar lo anterior
% y descomentar lo siguiente
%Vº. Bº. del Tutor:\\[2cm]
%D. nombre tutor


\newpage\null\thispagestyle{empty}\newpage




\frontmatter

% Abstract en castellano
\renewcommand*\abstractname{Resumen}
\begin{abstract}
Este Trabajo de Fin de Grado presenta el diseño e implementación de TerapiTrack, una plataforma web para la gestión de sesiones de telerehabilitación orientadas a pacientes con enfermedad de Parkinson que residen en zonas rurales o alejadas de los centros de referencia.

El sistema permite a profesionales sanitarios crear ejercicios terapéuticos en vídeo, programar sesiones personalizadas, guiarlas seleccionando los ejercicios a realizar y evaluar de forma remota el desempeño de los pacientes;
mientras que éstos pueden realizar los ejercicios en su domicilio, grabándose su ejecución desde el navegador, con la cámara del dispositivo y consultar su progreso a lo largo del tratamiento.

TerapiTrack se ha desarrollado con Python y Flask, utiliza SQLAlchemy sobre SQLite y PostgreSQL para la gestión de datos e integra Cloudinary para el almacenamiento escalable de vídeos.
Además, sigue criterios de accesibilidad web para facilitar su uso por parte de personas con limitaciones motoras o con poca experiencia tecnológica.

El proyecto incluye una arquitectura modular basada en blueprints, un modelo de datos normalizado, pruebas unitarias con cobertura del 99\% sobre los módulos críticos y despliegue en Heroku.
\end{abstract}

\renewcommand*\abstractname{Descriptores}
\begin{abstract}
Telerehabilitación, enfermedad de Parkinson, aplicación web sanitaria, registro de ejercicios en vídeo, Flask, SQLAlchemy, Cloudinary.
\end{abstract}

\clearpage

% Abstract en inglés
\renewcommand*\abstractname{Abstract}
\begin{abstract}
This Final Degree Project presents the design and implementation of TerapiTrack, a web platform for managing telerehabilitation sessions aimed at patients with Parkinson's disease who live in rural areas or far from specialized medical centers.

The system allows healthcare professionals to create therapeutic video exercises, schedule personalized sessions, guide them by selecting the exercises to be performed, and remotely evaluate patient performance.
Meanwhile, patients can carry out their exercises at home, recording themselves directly through their device’s camera in the browser, and view their progress throughout the treatment.

TerapiTrack has been developed using Python and Flask, employs SQLAlchemy with SQLite and PostgreSQL for data management and integrates Cloudinary for scalable video storage.
In addition, it follows web accessibility standards to ensure usability for individuals with motor limitations or limited technological experience.

The project features a modular architecture based on blueprints, a normalized data model, unit tests achieving 99\% coverage on critical modules, and deployment on Heroku.
\end{abstract}

\renewcommand*\abstractname{Keywords}
\begin{abstract}
Telerehabilitation, Parkinson's disease, healthcare web application, video exercise recording, Flask, SQLAlchemy, Cloudinary.
\end{abstract}

\clearpage

% Indices
\tableofcontents

\clearpage

\listoffigures

\clearpage

\listoftables
\clearpage

\mainmatter
\capitulo{1}{Introducción}
A medida que la población envejece y los servicios sanitarios se concentran principalmente en ciudades, muchas personas que residen en zonas rurales encuentran dificultades para acceder a terapias de rehabilitación. Esta situación afecta de manera especial a quienes padecen determinados problemas de salud crónicos, como la enfermedad del Parkinson. Ésta es una patología progresiva que provoca rigidez muscular, temblores y dificultad en el movimiento, y para la cual actualmente no existe cura. Sin embargo, mediante terapia, los pacientes pueden ralentizar el avance de la enfermedad, mejorar su movilidad y preservar durante más tiempo su independencia y calidad de vida. Las limitaciones de movilidad y la lejanía de los centros de referencia complican aún más el acceso, repercutiendo en el bienestar de los pacientes y en el esfuerzo de sus familias.

Ante esta problemática, surge \textbf{TerapiTrack}, una plataforma web diseñada para facilitar el acompañamiento y el seguimiento de terapias, principalmente para personas diagnosticadas con la enfermedad de Parkinson. Sin embargo, su enfoque flexible y modular permite también su uso en otros casos que requieran terapias a distancia, como personas con otras enfermedades crónicas. TerapiTrack busca eliminar barreras geográficas y de accesibilidad, adaptándose a las necesidades y capacidades de distintos tipos de pacientes.

\section{Contexto del problema}
La Enfermedad de Parkinson es una enfermedad neurodegenerativa que suele requerir una combinación de medicación y diversos ejercicios de fisioterapia o estimulación cognitiva, adaptados a cada fase y persona. En España, esta problemática se acentúa especialmente en la conocida “España Vaciada”, donde la mayoría de la población son personas de edad avanzada, grupo especialmente propenso a padecer este tipo de enfermedades. Para estos pacientes, acudir con regularidad a consultas y sesiones presenciales supone un reto considerable, no solo por la falta de servicios sanitarios especializados en muchas regiones rurales, sino también por la distancia a los centros de salud de referencia y las propias limitaciones físicas que provoca la enfermedad. Esta situación también se da en otros perfiles de pacientes que necesitan un seguimiento constante para evitar recaídas y para mantener su autonomía el mayor tiempo posible.

\section{Propuesta de solución}
El objetivo principal de TerapiTrack es acercar la rehabilitación y el control terapéutico al entorno cotidiano del paciente, apoyándose en la tecnología para romper barreras tradicionales. Las principales funcionalidades que aporta la herramienta son:

\begin{itemize}
    \item Permitir que los profesionales sanitarios creen ejercicios y rutinas personalizadas.
    \item Ofrecer la posibilidad de organizar sesiones y ajustar las actividades según la evolución de cada paciente.
    \item Facilitar el seguimiento remoto mediante la grabación y evaluación de los ejercicios realizados.
    \item Proporcionar una vía de comunicación sencilla y segura entre pacientes y especialistas, ayudando a resolver dudas y ajustar tratamientos sin desplazamientos.
    \item Garantizar una interfaz comprensible, accesible y adaptada a usuarios con dificultades motoras o cognitivas.
\end{itemize}

\section{Estructura de la memoria}
La presente memoria se divide en diferentes capítulos que buscan ofrecer una visión global del trabajo realizado:

\begin{itemize}
    \item \textbf{Capítulo 1. Introducción}: Expone el origen del proyecto y contextualiza el problema a resolver.
    \item \textbf{Capítulo 2. Objetivos del proyecto}: Presenta los objetivos perseguidos, tanto generales como específicos.
    \item \textbf{Capítulo 3. Conceptos teóricos}: Revisa los fundamentos técnicos y sanitarios necesarios para comprender la solución desarrollada.
    \item \textbf{Capítulo 4. Técnicas y herramientas}: Describe las tecnologías, herramientas y metodologías empleadas.
    \item \textbf{Capítulo 5. Aspectos relevantes del desarrollo}: Explica las fases principales del desarrollo y las decisiones adoptadas.
    \item \textbf{Capítulo 6. Trabajos relacionados}: Analiza otras soluciones similares y posiciona TerapiTrack respecto a ellas.
    \item \textbf{Capítulo 7. Conclusiones y líneas de trabajo futuras}: Resume los resultados alcanzados y plantea posibles mejoras o ampliaciones.
\end{itemize}

Junto a estos capítulos, se incluyen anexos donde se recopila documentación técnica, manuales y otra información complementaria.

\section{Materiales entregados}
Para facilitar la validación, el uso y la posible evolución del sistema, se entrega junto con la memoria un conjunto de materiales adicionales:

\begin{itemize}
    \item Código fuente completo del proyecto y su historial de versiones.
    \item Definición de la estructura de la base de datos y un conjunto de datos de prueba.
    \item Documentación técnica detallada en los anexos.
    \item Manual de usuario y de instalación.
    \item Batería de pruebas del sistema, con los resultados obtenidos.
\end{itemize}

Este conjunto de materiales pretende que cualquier persona interesada pueda comprender, utilizar y mejorar la solución presentada, contribuyendo así a un mejor acceso a la rehabilitación y al acompañamiento terapéutico a través de medios digitales.

\capitulo{2}{Objetivos del proyecto}

Este capítulo expone de manera clara los propósitos principales del desarrollo de TerapiTrack, distinguiendo entre los objetivos funcionales ―enfocados en las necesidades detectadas para usuarios y profesionales― y los objetivos técnicos, que marcan las directrices para la construcción y despliegue de la plataforma. Esta diferenciación permite visualizar tanto la utilidad práctica del sistema como los aspectos tecnológicos que respaldan su funcionamiento.

\section{Objetivos funcionales}

\begin{itemize}
    \item Gestionar diferentes perfiles de usuario (administrador, paciente, personal sanitario: médico, terapeuta, psicólogo, enfermero) con permisos diferenciados.
    \item Permitir la vinculación de pacientes con los profesionales responsables de su seguimiento.
    \item Ofrecer una biblioteca de ejercicios y recursos terapéuticos en formato audiovisual, filtrable según necesidades clínicas.
    \item Facilitar la creación y programación de sesiones personalizadas de rehabilitación.
    \item Registrar y evaluar el progreso de cada paciente mediante grabaciones, informes y revisión de ejercicios.
    \item Impulsar la comunicación bidireccional y las notificaciones entre los usuarios y los profesionales, para un acompañamiento real y adaptado.
    \item Asegurar la trazabilidad de la evolución terapéutica, con un acceso a la información ajustado a la privacidad y rol de cada usuario.
\end{itemize}

\section{Objetivos técnicos}

\begin{itemize}
    \item Desarrollar la aplicación sobre un framework robusto (Flask) y modular, que facilite tanto el mantenimiento futuro como la ampliación de funcionalidades.
    \item Diseñar e implementar una base de datos relacional segura y eficiente, que garantice la integridad y la protección de la información sensible.
    \item Crear una interfaz accesible y sencilla, especialmente adaptada a personas con dificultades motoras y/o cognitivas.
    \item Aplicar mecanismos de cifrado y control de accesos para preservar la privacidad y la seguridad de los datos almacenados y transferidos.
    \item Implementar el despliegue de la aplicación en la nube (por ejemplo, a través de Heroku), asegurando así disponibilidad remota y alta accesibilidad.
    \item Desarrollar una batería de pruebas automatizadas que permita validar el correcto funcionamiento y la calidad del sistema.
\end{itemize}

\capitulo{3}{Conceptos teóricos}

En este capítulo se revisan los conceptos teóricos básicos que sirven de soporte al proyecto TerapiTrack, tanto en su diseño como en su implementación.
Primero se abordan nociones relacionadas con la enfermedad de Parkinson y la rehabilitación, y después se describen los principios de accesibilidad, los patrones de diseño y los modelos de datos que se han tenido en cuenta durante el desarrollo.

\section{Enfermedades degenerativas y enfermedad de Parkinson}

Las enfermedades degenerativas son trastornos en los que determinadas estructuras del organismo se deterioran de forma progresiva, lo que provoca una pérdida lenta pero continua de funciones.
Cuando el sistema nervioso central se ve afectado, este deterioro suele traducirse en problemas de movimiento, equilibrio, memoria o comunicación que requieren tratamientos prolongados y un seguimiento estrecho~\cite{ninds_parkinson,who_parkinson_2023}.

La enfermedad de Parkinson es un trastorno neurodegenerativo crónico que afecta principalmente al control del movimiento.
Se caracteriza por temblor en reposo, rigidez muscular, lentitud en la iniciación de los movimientos y alteraciones del equilibrio, y actualmente no tiene cura.
Aunque puede aparecer en adultos jóvenes, afecta sobre todo a personas de edad avanzada; se estima que aproximadamente una de cada cien personas mayores de 60 años y en torno a un 2\% de la población española mayor de 65 años padecen la enfermedad,
lo que implica que muchos pacientes presentan limitaciones de movilidad y dependen de terceras personas para desplazarse, especialmente cuando viven en ámbitos rurales~\cite{medicosypacientes_parkinson_mayores60,sen_parkinson_2porciento,who_parkinson_2023}.

La combinación de tratamiento farmacológico y programas de rehabilitación específicos permite reducir la intensidad de los síntomas, mantener la autonomía durante más tiempo y mejorar la calidad de vida de los pacientes~\cite{cleveland_parkinson,who_parkinson_2023}.

\section{Terapia ocupacional y rehabilitación}

La terapia ocupacional es una disciplina sanitaria que ayuda a las personas a mantener o mejorar su autonomía en las actividades significativas de su vida diaria, adaptando tanto las tareas como el entorno cuando existe una enfermedad, lesión o discapacidad~\cite{rehab_hss,nbcot_ot}.
En pacientes con enfermedad de Parkinson, la terapia ocupacional trabaja aspectos como vestirse, la higiene personal, la movilidad dentro del domicilio o la organización de rutinas, con el objetivo de prolongar la autonomía en las actividades básicas y sociales~\cite{rehab_hss}.

La rehabilitación de trastornos del movimiento suele combinar fisioterapia, terapia ocupacional, ejercicio físico estructurado y, en algunos casos, logopedia.
Este tipo de intervención requiere sesiones frecuentes y un seguimiento continuado, por lo que acceder de forma regular a los servicios especializados resulta especialmente complicado para pacientes que viven lejos de los centros de referencia o que tienen dificultades para desplazarse~\cite{parkinson_telerehab_review}.

\section{Telemedicina y telerehabilitación}

La telemedicina hace referencia al uso de las tecnologías de la información y la comunicación para prestar servicios sanitarios a distancia, permitiendo que paciente y profesional interactúen sin encontrarse en el mismo lugar físico.
Dentro de este ámbito, la telerehabilitación se centra en el diseño, la supervisión y la evaluación de programas de rehabilitación mediante herramientas remotas, como videoconferencia o plataformas web específicas~\cite{rosen2009_telerehab_technologies}.

En el caso de la enfermedad de Parkinson, la telerehabilitación ofrece la posibilidad de mantener la frecuencia de las terapias aunque el paciente resida en la “España vaciada” o tenga movilidad reducida.
Permite reducir desplazamientos, facilita la adherencia al tratamiento y hace posible que el profesional revise de forma diferida los ejercicios realizados en casa~\cite{parkinson_telerehab_review,parkinson_telerehab_protocol}.

\section{Accesibilidad y usabilidad en aplicaciones sanitarias}

Las aplicaciones orientadas a la salud deben diseñarse pensando en usuarios con diferentes capacidades motoras, sensoriales o cognitivas, y también en personas de distintas edades, incluidas aquellas con poca experiencia en el uso de ordenadores o dispositivos móviles.
Ni la edad ni el nivel de conocimiento tecnológico deberían convertirse en una barrera a la hora de acceder a programas de telerehabilitación~\cite{telehealth_accessibility}.

Los estándares de accesibilidad web, como las pautas WCAG, recogen cuatro principios generales que sirven de guía para el diseño de interfaces accesibles~\cite{wcag22}.

\begin{itemize}
    \item \textbf{Perceptibilidad}: La información y los componentes de la interfaz deben presentarse de forma que puedan percibirse por distintos tipos de usuarios, por ejemplo cuidando el contraste de color, el tamaño de la tipografía o el uso de textos alternativos en imágenes~\cite{wcag22}.
    \item \textbf{Operabilidad}: Los controles deben poder manejarse con diferentes dispositivos de entrada (ratón, teclado, pantallas táctiles, o mandos) y con un número reducido de acciones, algo especialmente relevante en personas con limitaciones motoras~\cite{telehealth_accessibility,rosen2009_telerehab_technologies}.
    \item \textbf{Comprensibilidad}: Los textos, mensajes de error y flujos de navegación han de ser claros y coherentes, evitando tecnicismos innecesarios y sobrecarga de información~\cite{wcag22}.
    \item \textbf{Robustez}: El contenido debe mostrarse correctamente en distintos navegadores y dispositivos, y ser compatible con tecnologías de apoyo como lectores de pantalla o ampliadores de pantalla~\cite{wcag22}.
\end{itemize}

Estos principios han guiado el diseño de la interfaz de TerapiTrack, priorizando menús sencillos, textos claros y un número reducido de acciones por pantalla~\cite{GithubTerapitrack}.

\section{Patrones de diseño y arquitectura Modelo-Vista-Controlador}

En el desarrollo de aplicaciones de software es habitual apoyarse en patrones de diseño que separan responsabilidades y facilitan la evolución del sistema; en el caso de TerapiTrack, esto se traduce en el uso de una arquitectura de estilo Modelo-Vista-Controlador (MVC)~\cite{gamma_design_patterns}.
Este patrón organiza la aplicación en tres componentes principales:

\begin{itemize}
    \item \textbf{Modelo}: Representa los datos y las reglas de negocio asociadas a cada entidad del dominio, incluyendo la definición de estructuras de almacenamiento y restricciones de integridad.
    \item \textbf{Vista}: Define cómo se presenta la información al usuario, normalmente mediante páginas web o plantillas que muestran formularios, listados y mensajes.
    \item \textbf{Controlador}: Actúa como intermediario entre el modelo y la vista, recibiendo las peticiones del usuario, invocando la lógica necesaria y seleccionando la respuesta adecuada.
\end{itemize}

Este tipo de arquitectura favorece que el código relacionado con la persistencia de datos, la lógica de aplicación y la interfaz de usuario pueda evolucionar de manera relativamente independiente, algo especialmente útil en proyectos que pueden ampliarse en el futuro~\cite{gamma_design_patterns}.

\section{Bases de datos relacionales}

Las bases de datos relacionales permiten organizar la información en tablas relacionadas entre sí mediante claves primarias y foráneas, garantizando la integridad de los datos mediante restricciones y reglas de consistencia.~\cite{date_db}
Este modelo es el que se ha aplicado en TerapiTrack para organizar usuarios, pacientes, profesionales sanitarios, ejercicios, sesiones, evaluaciones y vídeos de respuesta.

En este contexto aparecen relaciones de distinto tipo: \textbf{uno a muchos}, como la que se da entre un paciente y las sesiones que tiene programadas o entre un ejercicio y las veces que se incluye en distintas sesiones;
\textbf{uno a uno}, como la que existe entre la tabla de usuarios y las tablas de paciente o profesional, donde cada registro de usuario se especializa en un único tipo de perfil;
y \textbf{muchos a muchos}, como ocurre al vincular pacientes con varios profesionales sanitarios o al asociar ejercicios concretos a distintas sesiones.~\cite{date_db}

La \textbf{normalización} es el proceso mediante el cual se descompone la información en tablas más pequeñas para reducir la redundancia y evitar anomalías en las operaciones de inserción, actualización y borrado.
En la práctica, esto implica diseñar el esquema de manera que cumpla distintas formas normales (primera, segunda, tercera, etc.), decidiendo qué atributos deben ir en cada tabla y cómo se relacionan entre sí, algo que suele ser una de las partes más delicadas del modelado de datos.~\cite{date_db}

En el caso de las relaciones de muchos a muchos se introduce una tabla intermedia que almacena los pares de elementos relacionados y permite gestionar de forma clara y trazable esas asociaciones (por ejemplo, \textit{Paciente\_Profesional} para la vinculación entre pacientes y profesionales
o \textit{Ejercicio\_Sesion} para la asignación de ejercicios a sesiones), manteniendo al mismo tiempo las formas normales y la integridad referencial del modelo de datos.~\cite{date_db}

\section{Validación y pruebas de software}

La validación de un sistema de software busca comprobar que la solución desarrollada cumple los requisitos definidos y que se comporta de forma correcta en los escenarios de uso previstos.
Para ello se utilizan diferentes tipos de pruebas, que pueden combinarse según las necesidades del proyecto~\cite{pressman_ingenieria_sw}.

Entre las más habituales se encuentran las \textbf{pruebas unitarias}, centradas en módulos o funciones concretas; las \textbf{pruebas de integración}, que comprueban el funcionamiento conjunto de varios componentes; y las \textbf{pruebas funcionales} o de sistema, orientadas a verificar flujos de usuario o casos de uso completos.
Automatizar parte de estas pruebas permite detectar errores de diseño o programación de manera temprana y reduce el riesgo de fallos en fases avanzadas del desarrollo~\cite{pressman_ingenieria_sw}.

Este conjunto de conceptos proporciona el contexto teórico y tecnológico que sustenta la memoria y ayuda a entender las decisiones que se detallan en los capítulos posteriores.

\capitulo{4}{Técnicas y herramientas}
Este capítulo describe las metodologías, tecnologías y utilidades elegidas para el desarrollo de TerapiTrack, justificando cada decisión y recogiendo las principales ventajas de cada alternativa considerada.

\section{Metodología de desarrollo}

\subsection{Gestión ágil con Jira}
Se optó por una organización ágil utilizando \textbf{Jira} para la planificación y el control del trabajo. Esta herramienta ha permitido dividir el proyecto en sprints y tareas distribuidas en tableros, facilitando el seguimiento del progreso real y la adaptación ágil ante imprevistos. Jira es compatible con metodologías como Scrum y Kanban, permitiendo organizar el desarrollo por etapas incrementales y gestionar incidencias de forma eficiente.



\section{Tecnologías de backend}

\subsection{Python y Flask}
\textbf{Flask} ha sido el framework seleccionado para el desarrollo web backend debido a su flexibilidad, ligereza y la facilidad con la que se puede integrar con diferentes extensiones. Permite crear aplicaciones a medida y controlar con detalle la estructura de cada módulo, sin imponer capas innecesarias ni patrones rígidos. Otras plataformas como Django fueron consideradas, aunque Flask resultó más idóneo para un proyecto modular y de tamaño medio, donde la sencillez y el control sobre la arquitectura eran prioritarios.

\textbf{Ventajas de Flask:}
\begin{itemize}
    \item Ligero, flexible y fácil de aprender
    \item Permite una estructura de proyecto a medida
    \item Ecosistema de extensiones muy abundante
    \item Ideal para aplicaciones modulares y APIs RESTful
\end{itemize}

\subsection{SQLAlchemy}
Para la interacción con la base de datos se ha implementado \textbf{SQLAlchemy}, un ORM ampliamente utilizado en el ecosistema Python. SQLAlchemy permite gestionar modelos y relaciones mediante objetos, evitando la escritura directa de SQL y facilitando la migración o ampliación futura del sistema. Su uso minimiza errores y simplifica la validación de datos, además de proporcionar portabilidad entre distintos motores de base de datos.

\subsection{Alternativas}
Se evaluó el uso de otras bibliotecas ORM y la manipulación directa con SQL clásico, pero se descartaron para evitar redundancia y riesgos de inconsistencia.

\section{Gestión de datos}

\subsection{Base de datos: SQLite}
Durante el desarrollo se ha empleado \textbf{SQLite} por su sencillez y portabilidad (no requiere servidor y almacena toda la información en un solo archivo), idónea para pruebas y prototipos rápidos.

\subsection{Herramienta complementaria: DB Browser for SQLite}
Durante la fase inicial, DB Browser for SQLite facilitó la visualización y depuración de la base de datos desde un entorno gráfico.

\section{Tecnologías de frontend}

\subsection{Bootswatch}
El diseño y la apariencia de la interfaz de TerapiTrack se basan en \textbf{Bootswatch}, una colección de temas CSS construidos sobre Bootstrap. Esta decisión permitió aplicar rápidamente estilos accesibles y coherentes, favoreciendo la navegación intuitiva y el cumplimiento de estándares WCAG de accesibilidad web.

\subsection{HTML5, JavaScript y Jinja}
Se empleó \textbf{HTML5} como base estructural de las páginas, \textbf{JavaScript} para funcionalidades interactivas y la grabación web, y el motor \textbf{Jinja} para el renderizado dinámico del frontend en Flask. La combinación de estas tecnologías aporta flexibilidad y compatibilidad con todos los navegadores modernos.

\section{Testing y aseguramiento de la calidad}

\subsection{Pytest}
El testeo automatizado se llevó a cabo con \textbf{Pytest}, por su sintaxis sencilla, el soporte de fixtures reutilizables y la facilidad para parametrizar y agrupar pruebas. Esta herramienta ha permitido obtener cobertura completa en los módulos críticos y facilita la detección precoz de errores antes del despliegue.

\section{Despliegue e infraestructura}

\subsection{Heroku}
\textbf{Heroku} se eligió como plataforma de despliegue por su facilidad de uso, integración con Git y escalado automático de recursos según la carga de trabajo. El sistema de despliegue continuo con \textit{git push} agiliza las pruebas y actualizaciones en entornos cloud, permitiendo centrarse en la funcionalidad sin preocuparse por la gestión del servidor.

\section{Control de versiones y colaboración}

\subsection{Git y GitHub}
Toda la gestión del código se apoyó en \textbf{Git} como sistema distribuido de control de versiones y \textbf{GitHub} como repositorio central, lo que facilitó el trabajo colaborativo y la trazabilidad de los cambios a lo largo de todas las iteraciones. El uso de GitHub permitió registrar issues, realizar revisiones de código y mantener sincronizado el trabajo entre las distintas ramas del repositorio.

\section{Otras herramientas de desarrollo}

\begin{itemize}
\item \textbf{Visual Studio Code}: Entorno principal de edición y depuración, con extensiones dedicadas para Python, Flask y control de versiones.
\item \textbf{Overleaf}: Redacción colaborativa de esta memoria en LaTeX, facilitando la organización de capítulos y anexos.
\item \textbf{Draw.io}: Creación de diagramas para representar la arquitectura y el flujo de datos del sistema.
\end{itemize}

\section{Comparativas y justificación de elecciones}

A lo largo del desarrollo, se compararon tecnologías tanto por facilidad de integración como por el soporte ofrecido por la comunidad, documentación y curva de aprendizaje. Herramientas como Flask y SQLAlchemy se adaptaron perfectamente a los requisitos de modularidad y personalización, mientras que plataformas como Heroku y GitHub ofrecieron un entorno sencillo y ágil para desplegar y gestionar el ciclo de vida completo del proyecto.

Este conjunto de herramientas ha permitido desarrollar TerapiTrack de forma organizada, eficiente y con una sólida base para abordar futuras escalas y mejoras del sistema.

ntr\capitulo{5}{Aspectos relevantes del desarrollo del proyecto}

Este capítulo describe los aspectos más significativos del desarrollo de TerapiTrack desde un punto de vista práctico.
Se presentan las decisiones adoptadas en cuanto al ciclo de vida del proyecto, la arquitectura de la aplicación, el diseño del modelo de datos y la implementación de los flujos de uso más relevantes, así como algunos problemas encontrados y las soluciones que se han aplicado en cada caso.

\section{Ciclo de vida y organización del trabajo}

El proyecto no se ha desarrollado como un bloque único, sino en varias iteraciones en las que se iba ampliando la funcionalidad y ajustando el diseño en función de lo que se iba aprendiendo.
En la práctica, esto ha supuesto combinar una planificación inicial con una forma de trabajar más flexible, en la que se han ido revisando prioridades y corrigiendo decisiones técnicas cuando era necesario \cite{pressman_ingenieria_sw}.

Desde el principio se utilizó Jira para organizar el trabajo en tareas manejables.
Al comienzo había simplemente una lista de tareas generales (por ejemplo, "crear modelo de datos básico" o "pantallas de login y registro"), pero a medida que el proyecto fue creciendo se reorganizó el tablero en secciones por rol (paciente, profesional, administrador) y por tipos de actividad (desarrollo, pruebas, documentación).
En lugar de mantener la lista plana inicial, se crearon épicas que agrupaban funcionalidades relacionadas, se definieron historias de usuario más claras y se desglosaron en tareas pequeñas que resultaban más fáciles de abordar y de dar por cerradas \cite{JiraTerapitrack}.
Este cambio ayudó a localizar más rápido qué parte del sistema se estaba tocando en cada momento y a evitar que se quedaran funcionalidades "a medias".

Git y GitHub se han utilizado de forma continuada para gestionar el código fuente, aunque con una organización de ramas poco convencional.
El repositorio remoto de GitHub ha servido como copia de seguridad y como registro de la evolución del proyecto, pero en lugar de trabajar sobre la rama \texttt{main} y crear ramas auxiliares para funcionalidades concretas, casi todo el desarrollo real se ha llevado a cabo en una rama llamada \texttt{Pruebas} \cite{GithubTerapitrack}.
La idea inicial era experimentar con cambios y funcionalidades en esta rama auxiliar antes de integrarlos en \texttt{main}, manteniendo la rama principal estable y con un historial limpio.

Durante un periodo de varios meses en verano se trabajó de forma intensiva en local sin realizar commits ni subir el código a GitHub, lo que hizo que después se tuvieran que subir bloques grandes de cambios de una sola vez y que algunas tareas marcadas como completadas en Jira no tengan asociado un commit concreto \cite{JiraTerapitrack}.
Una vez que se retomó la rutina de commits más frecuentes, toda esa actividad quedó registrada en la rama \texttt{Pruebas}, desde la que se ha desarrollado la mayor parte de las funcionalidades (gestión de ejercicios, sesiones, vídeos, evaluaciones, tests unitarios con alta cobertura de código (en torno al 99\%), integración de Cloudinary, mejoras de documentación en el código, etc.).

Aunque no siempre se ha logrado, en la última fase se intentó corregir esta situación manteniendo una rutina más disciplinada: cuando se completaba una tarea se procuraba hacer un commit con un mensaje descriptivo y, en la medida de lo posible, realizar el \textit{push} y la actualización del estado de la tarea en Jira de forma casi simultánea.
A raíz de esta experiencia, se ha tomado conciencia de la importancia de mantener una cierta disciplina en el uso de las herramientas de control de versiones: realizar commits con cierta frecuencia, escribir mensajes claros y mantener sincronizado el estado de las tareas en Jira con los cambios reales en el código.
La rama \texttt{Pruebas} se integrará en \texttt{main} mediante un merge final justo antes de la entrega del proyecto, momento en el que el historial completo de desarrollo quedará reflejado en la rama principal y visible en el repositorio para futuras consultas o ampliaciones del sistema \cite{GithubTerapitrack,JiraTerapitrack}.

\section{Arquitectura y organización del código}

La aplicación sigue una arquitectura basada en el patrón Modelo–Vista–Controlador adaptado a Flask, en el que la lógica de presentación, los datos y el control de flujo se mantienen separados \cite{gamma_design_patterns}.
Esta separación facilita localizar rápidamente dónde está definida cada parte del sistema y permite modificar una capa sin afectar a las demás.

\subsection{Estructura de carpetas}

El código fuente se encuentra organizado en la carpeta \texttt{src}, que contiene varios subdirectorios:

\begin{itemize}
  \item \texttt{modelos}: Define las entidades del dominio (Usuario, Paciente, Profesional, Ejercicio, Sesion, Evaluacion, VideoRespuesta) y las tablas de asociación para las relaciones de muchos a muchos.
  \item \texttt{controladores}: Agrupa las rutas y la lógica de negocio por rol (\texttt{auth\_controlador.py}, \texttt{admin\_controlador.py}, \texttt{profesional\_controlador.py}, \texttt{paciente\_controlador.py}).
  \item \texttt{vistas}: Contiene las plantillas Jinja organizadas en subcarpetas por rol (\texttt{admin}, \texttt{profesional}, \texttt{paciente}), todas ellas heredando de \texttt{base.html}.
  \item \texttt{static}: Almacena los recursos estáticos (CSS, imágenes, vídeos de ejemplo).
\end{itemize}

Además, en la raíz de \texttt{src} se encuentran ficheros de configuración y utilidades:

\begin{itemize}
  \item \texttt{app.py}: Punto de entrada de la aplicación, donde se crea la instancia de Flask, se registran los blueprints y se configuran las extensiones.
  \item \texttt{extensiones.py}: Inicializa las extensiones (SQLAlchemy, Flask-Login, Flask-WTF, Bcrypt) de forma centralizada para reutilizarlas en todos los módulos.
  \item \texttt{config.py}: Define la configuración del entorno (base de datos, claves secretas, rutas de subida de archivos).
  \item \texttt{decoradores.py}: Contiene decoradores personalizados para comprobar roles y condiciones de acceso antes de ejecutar las vistas.
  \item \texttt{poblar\_bd.py}: Script que inicializa la base de datos con datos de prueba (usuarios, pacientes, ejercicios, sesiones) para facilitar el desarrollo y las demostraciones.
\end{itemize}

Esta organización ha resultado clave para mantener el proyecto manejable a medida que crecía el número de funcionalidades.

\subsection{Uso de blueprints}

Flask permite organizar las rutas en \emph{blueprints}, módulos independientes que después se registran en la aplicación principal.
En TerapiTrack se ha seguido este patrón para separar la lógica según el rol del usuario \cite{FlaskBlueprintsDocs,RealPythonBlueprints}.

Por ejemplo, \texttt{profesional\_controlador.py} define un blueprint que agrupa las rutas para crear ejercicios, asignar sesiones y evaluar el desempeño de los pacientes, mientras que \texttt{paciente\_controlador.py} contiene las rutas para consultar sesiones asignadas, ejecutar ejercicios y ver el historial de evaluaciones.
De esta forma, cada controlador puede evolucionar de manera relativamente independiente sin mezclar código de distintos roles en un mismo fichero.

\subsection{Decoradores para control de acceso}

Uno de los aspectos que ha requerido más atención ha sido asegurar que cada usuario solo pueda acceder a las funcionalidades que le corresponden.
Para ello se han definido decoradores personalizados en \texttt{decoradores.py} que comprueban el rol del usuario antes de ejecutar una vista.

Por ejemplo, el decorador \texttt{@admin\_required} verifica que el usuario autenticado tenga rol de administrador; si no es así, redirige a la página principal con un mensaje de error.
De forma similar, los decoradores \texttt{@profesional\_required} y \texttt{@paciente\_required} aseguran que las rutas de cada controlador solo sean accesibles para el tipo de usuario adecuado.

Esta solución ha simplificado la lógica de las vistas, evitando tener que repetir las mismas comprobaciones de seguridad en cada ruta.

\section{Diseño del modelo de datos}

El modelo de datos se ha diseñado aplicando principios de normalización para evitar redundancias y mantener la coherencia de la información \cite{date_db}.
Las entidades principales y sus relaciones se describen a continuación.

\subsection{Entidades principales}

\begin{itemize}
  \item \textbf{Usuario}: Representa a cualquier persona que accede al sistema, con campos básicos (nombre, correo, contraseña cifrada) y un campo de rol que indica si es administrador, profesional o paciente.
  \item \textbf{Paciente}: Extiende la información de un usuario de tipo paciente, incluyendo datos clínicos y de contacto.
  \item \textbf{Profesional}: Extiende la información de un usuario de tipo profesional, con datos de especialidad y contacto.
  \item \textbf{Ejercicio}: Define un ejercicio de rehabilitación, con su nombre, descripción, duración estimada y la ruta al vídeo de ejemplo.
  \item \textbf{Sesion}: Representa una sesión terapéutica asignada a un paciente en una fecha concreta, con un estado que indica si está pendiente, en curso o completada.
  \item \textbf{VideoRespuesta}: Almacena la grabación del paciente al realizar un ejercicio dentro de una sesión, junto con la fecha de grabación y la URL del vídeo.
  \item \textbf{Evaluacion}: Contiene la calificación y los comentarios del profesional sobre el desempeño del paciente en un ejercicio concreto de una sesión.
\end{itemize}

\subsection{Relaciones de muchos a muchos}

Algunas relaciones del sistema requieren tablas intermedias para conectar entidades:

\begin{itemize}
  \item \textbf{Paciente-Profesional}: Un paciente puede estar asignado a varios profesionales y un profesional puede atender a varios pacientes.
  \item \textbf{Ejercicio-Sesion}: Una sesión puede incluir varios ejercicios y un mismo ejercicio puede aparecer en distintas sesiones. La tabla intermedia \texttt{EjercicioSesion} también almacena el orden de los ejercicios dentro de la sesión.
\end{itemize}

La decisión de normalizar estas relaciones evita duplicar información y facilita modificar las asignaciones sin afectar a los datos de base de las entidades.

\subsection{Claves e índices}

Todas las tablas tienen una clave primaria autogenerada (\texttt{id}) y claves foráneas que garantizan la integridad referencial.
En algunos casos se han añadido índices adicionales (por ejemplo, sobre el campo \texttt{email} en Usuario o sobre la fecha de la sesión) para mejorar el rendimiento de las consultas más frecuentes.

Durante el desarrollo se detectó que algunas consultas sobre el historial de sesiones de un paciente eran lentas, y la creación de un índice sobre la columna de fecha resolvió el problema sin necesidad de cambiar el modelo.

\section{Flujos de uso más relevantes}

Esta sección describe algunos de los flujos más importantes del sistema, explicando las decisiones técnicas que se han tomado y los problemas que han surgido durante su implementación.

\subsection{Autenticación y gestión de roles}

El flujo de autenticación se basa en Flask-Login, que proporciona sesiones gestionadas automáticamente y decoradores para proteger rutas que requieren autenticación previa.
Cuando un usuario inicia sesión, el sistema verifica su correo y contraseña (cifrada con Bcrypt) y, si son correctos, almacena su identidad en la sesión.

\begin{figure}[H]
  \centering
  \includegraphics[width=0.9\textwidth]{img/login.png}
  \caption{Pantalla de acceso a TerapiTrack con formulario de autenticación.}
  \label{fig:login}
\end{figure}

En la Figura~\ref{fig:login} se muestra el formulario de acceso común a todos los roles del sistema.\cite{GithubTerapitrack}

Una vez autenticado, el rol del usuario determina a qué parte de la aplicación puede acceder.
Los decoradores personalizados (\texttt{@admin\_required}, \texttt{@profesional\_required}, \texttt{@paciente\_required}) comprueban este rol antes de ejecutar cada vista.

Al principio del proyecto, las comprobaciones de rol se hacían directamente en cada ruta, lo que generaba código repetitivo y propenso a errores.
La introducción de decoradores centralizó esta lógica y simplificó el mantenimiento del sistema.

\subsection{Creación y asignación de sesiones terapéuticas}

Un profesional puede crear una sesión para un paciente seleccionando los ejercicios que debe realizar y la fecha en que debe completarla.
Este proceso se implementa mediante un formulario en el que se elige al paciente, se añaden ejercicios desde una lista disponible y se establece el orden en que deben ejecutarse.

Al guardar la sesión, el sistema crea un registro en la tabla \texttt{Sesion} y registros asociados en la tabla intermedia \texttt{EjercicioSesion} para cada ejercicio incluido, almacenando también su posición dentro de la secuencia.

Uno de los problemas que surgió al implementar este flujo fue evitar que se pudieran añadir ejercicios duplicados a la misma sesión.
La solución consistió en incluir una restricción de unicidad sobre la pareja (sesión, ejercicio) en la tabla intermedia y validar en el controlador que no se intente insertar el mismo ejercicio dos veces.

\subsection{Ejecución de sesiones por parte del paciente}

Cuando un paciente accede a su panel, puede ver las sesiones que tiene asignadas para el día actual.
Al seleccionar una sesión, se muestra una página (\texttt{ejecutar\_sesion.html}) que presenta los ejercicios en orden, con la posibilidad de ver el vídeo de ejemplo, grabar la ejecución propia del paciente y pasar al siguiente ejercicio.

Este flujo ha sido uno de los más complejos de implementar, porque requiere coordinar la reproducción de vídeos, la captura desde la cámara del navegador y el almacenamiento de las grabaciones.
Inicialmente se almacenaban los vídeos en una carpeta local del servidor (\texttt{uploads}), pero esto resultaba poco escalable y complicaba el despliegue en Heroku.
La solución fue integrar Cloudinary como servicio externo de almacenamiento, de forma que los vídeos se suben directamente desde el navegador del paciente y el sistema solo guarda la URL en la base de datos \cite{CloudinaryConsole}.

\subsection{Evaluación del desempeño}

Una vez que el paciente ha completado una sesión, el profesional puede acceder al listado de sesiones completadas y revisar los vídeos grabados por el paciente.
Para cada ejercicio de la sesión, el profesional puede asignar una calificación (por ejemplo, de 1 a 5) y añadir comentarios sobre qué aspectos se han realizado correctamente y cuáles necesitan mejorar.

Esta información se almacena en la tabla \texttt{Evaluacion}, vinculada al registro concreto de \texttt{EjercicioSesion}, de forma que cada ejercicio dentro de una sesión tiene su propia evaluación independiente.
Esto permite al paciente consultar más adelante qué ejercicios ha realizado bien y en cuáles debe centrar su atención en futuras sesiones.

\subsection{Resumen de flujos principales}

La tabla \ref{tabla:flujos_uso} resume los flujos de uso más relevantes de TerapiTrack, indicando el actor principal, una descripción breve y las vistas implicadas en cada caso.\cite{GithubTerapitrack}

\begin{table}[htbp]
  \centering
  \caption{Flujos de uso principales}
  \begin{tabular}{p{3cm} p{3cm} p{5cm} p{3cm}}
    \textbf{Flujo} & \textbf{Actor} & \textbf{Descripción} & \textbf{Vistas implicadas} \\
    Autenticación y acceso & Usuario (cualquier rol) & Introduce sus credenciales y, si son correctas, accede al panel correspondiente según su rol. & \texttt{login.html}, paneles \texttt{admin\_dashboard.html}, \texttt{profesional\_dashboard.html}, \texttt{paciente\_dashboard.html} \\
    Creación de sesión terapéutica & Profesional & Selecciona un paciente, elige ejercicios de la biblioteca, define el orden y la fecha de realización y guarda la nueva sesión. & \texttt{crear\_sesion.html}, \texttt{detalle\_sesion.html} \\
    Ejecución de sesión en domicilio & Paciente & Consulta las sesiones asignadas para el día, reproduce el vídeo de ejemplo, graba su ejecución y envía las grabaciones al sistema. & \texttt{mis\_sesiones.html}, \texttt{ejecutar\_sesion.html} \\
    Evaluación de ejercicios grabados & Profesional & Revisa las grabaciones de una sesión completada, asigna una puntuación a cada ejercicio y añade comentarios cualitativos. & \texttt{sesiones\_completadas.html}, \texttt{evaluar\_sesion.html} \\
    Consulta del historial y progreso & Paciente / Profesional & Visualiza el historial de sesiones, evaluaciones asociadas y evolución temporal de las puntuaciones. & \texttt{historial\_paciente.html}, \texttt{progreso\_paciente.html} \\
  \end{tabular}
  \label{tabla:flujos_uso}
\end{table}

\subsection{Visualización del progreso del paciente}

Además de almacenar las evaluaciones individuales de cada ejercicio, TerapiTrack genera gráficos de evolución temporal que muestran de forma agregada las puntuaciones obtenidas por el paciente en sus sesiones.\cite{GithubTerapitrack}
Estos gráficos, implementados con Chart.js en las vistas de profesional y paciente, permiten identificar tendencias de mejora o empeoramiento y facilitan que ambos comprendan de un vistazo cómo ha evolucionado el tratamiento a lo largo del tiempo.\cite{parkinson_telerehab_review,GithubTerapitrack}

\section{Interfaz de usuario}

Además de la organización interna del código y de los flujos de uso descritos en los apartados anteriores, resulta relevante mostrar cómo se materializan estas decisiones en la interfaz web de TerapiTrack para cada uno de los roles definidos en el sistema.\cite{GithubTerapitrack}

\subsection{Interfaz del administrador}

El administrador dispone de un panel de control desde el que puede consultar de un vistazo el estado global del sistema y acceder rápidamente a las principales acciones de gestión.\cite{GithubTerapitrack} En la Figura~\ref{fig:admin-dashboard} se muestran los indicadores de usuarios totales, pacientes activos y profesionales registrados, así como accesos directos para ver usuarios, gestionar vinculaciones y crear nuevas cuentas.

\begin{figure}[H]
  \centering
  \includegraphics[width=\textwidth]{img/admin_dashboard.png}
  \caption{Panel principal de administración con métricas globales y accesos rápidos a las tareas de gestión.}
  \label{fig:admin-dashboard}
\end{figure}

Desde el menú de usuarios, el administrador puede buscar, filtrar y gestionar todas las cuentas registradas en la plataforma.\cite{GithubTerapitrack} La Figura~\ref{fig:admin-usuarios} muestra la pantalla de gestión de usuarios, con filtros por rol y estado y acciones para visualizar el detalle, editar los datos o desactivar temporalmente una cuenta.

\begin{figure}[htbp]
  \centering
  \includegraphics[width=\textwidth]{img/admin_usuarios.png}
  \caption{Pantalla de gestión de usuarios con filtros por rol y estado y acciones de consulta, edición y desactivación.}
  \label{fig:admin-usuarios}
\end{figure}

La relación terapéutica entre pacientes y profesionales se gestiona desde el apartado de vinculaciones.\cite{GithubTerapitrack}
En la Figura~\ref{fig:admin-vinculaciones} se aprecia el listado de vinculaciones paciente-profesional con filtros por profesional y rango de fechas, mientras que la Figura~\ref{fig:admin-vincular} muestra el formulario para crear una nueva vinculación seleccionando un paciente y un profesional concretos.

\begin{figure}[H]
  \centering
  \includegraphics[width=\textwidth]{img/admin_vinculaciones.png}
  \caption{Listado de vinculaciones paciente-profesional con filtros por profesional y fecha de asignación.}
  \label{fig:admin-vinculaciones}
\end{figure}

\begin{figure}[H]
  \centering
  \includegraphics[width=\textwidth]{img/admin_vincular.png}
  \caption{Formulario para crear una nueva vinculación terapéutica entre un paciente y un profesional.}
  \label{fig:admin-vincular}
\end{figure}

Por último, el administrador puede configurar parámetros globales como la política de retención de vídeos, el límite de almacenamiento por usuario o la frecuencia de copias de seguridad.\cite{GithubTerapitrack} La Figura~\ref{fig:admin-configuracion} recoge la pantalla de configuración del sistema, donde estos valores se seleccionan mediante controles sencillos y se guardan de forma centralizada.

\begin{figure}[H]
  \centering
  \includegraphics[width=\textwidth]{img/admin_configuración.png}
  \caption{Pantalla de configuración del sistema con la política de retención de vídeos y otros parámetros globales.}
  \label{fig:admin-configuracion}
\end{figure}

\subsection{Interfaz del profesional}

El profesional dispone de un panel desde el que puede ver de un vistazo cuántos pacientes tiene asignados, cuántas sesiones y evaluaciones están pendientes y acceder rápidamente a las acciones más habituales.\cite{GithubTerapitrack} En la Figura~\ref{fig:prof-dashboard} se muestra este panel principal, con accesos directos a la biblioteca de ejercicios, la creación de nuevas sesiones y la consulta de las sesiones programadas.

\begin{figure}[H]
  \centering
  \includegraphics[width=\textwidth]{img/profesional_dashboard.png}
  \caption{Panel principal del profesional con resumen de pacientes asignados, sesiones pendientes y evaluaciones por realizar.}
  \label{fig:prof-dashboard}
\end{figure}

Desde el menú \emph{Mis pacientes}, el profesional puede filtrar por nombre, condición médica o rango de edad y acceder al detalle de cada caso.\cite{GithubTerapitrack} La Figura~\ref{fig:prof-pacientes} muestra el listado de pacientes asignados, con acciones directas para ver el progreso o crear una nueva sesión personalizada para cada uno.

\begin{figure}[H]
  \centering
  \includegraphics[width=\textwidth]{img/profesional_pacientes.png}
  \caption{Pantalla \emph{Mis pacientes} con filtros avanzados y acciones para ver progreso o crear nuevas sesiones.}
  \label{fig:prof-pacientes}
\end{figure}

Las sesiones se gestionan desde un listado que permite filtrar por estado, paciente y rango de fechas, facilitando el seguimiento del trabajo pendiente.\cite{GithubTerapitrack} En la Figura~\ref{fig:prof-sesiones} se muestra la pantalla \emph{Mis sesiones}, donde cada fila da acceso al detalle y a la ejecución de la sesión correspondiente.

\begin{figure}[H]
  \centering
  \includegraphics[width=\textwidth]{img/profesional_sesiones.png}
  \caption{Listado de sesiones programadas con filtros por estado, paciente y fechas.}
  \label{fig:prof-sesiones}
\end{figure}

La creación de una nueva sesión se realiza mediante un formulario en el que se selecciona el paciente, se fija la fecha programada y se eligen los ejercicios que formarán parte de la sesión.\cite{GithubTerapitrack} La Figura~\ref{fig:prof-crear-sesion} ilustra esta pantalla, donde también se comprueba visualmente que no se repiten ejercicios dentro de la misma sesión.

\begin{figure}[H]
  \centering
  \includegraphics[width=\textwidth]{img/profesional_crear_new_sesion.png}
  \caption{Formulario para crear una nueva sesión terapéutica seleccionando paciente, fecha y ejercicios.}
  \label{fig:prof-crear-sesion}
\end{figure}

Para revisar la evolución de un paciente, el profesional dispone de una vista específica que combina la información básica del caso con una gráfica de progreso generada con Chart.js.\cite{GithubTerapitrack} En la Figura~\ref{fig:prof-progreso} se observa cómo se representan las puntuaciones medias de las sesiones a lo largo del tiempo.

\begin{figure}[H]
  \centering
  \includegraphics[width=\textwidth]{img/profesional_ver_progreso.png}
  \caption{Pantalla de progreso del paciente con gráfica temporal de las puntuaciones de sus sesiones.}
  \label{fig:prof-progreso}
\end{figure}

Finalmente, la evaluación detallada de cada ejercicio se apoya en una vista que coloca lado a lado el vídeo demostrativo y la grabación del paciente, junto con los controles para asignar una puntuación y registrar comentarios.\cite{GithubTerapitrack} La Figura~\ref{fig:prof-evaluar-ejercicio} muestra esta pantalla de evaluación del ejercicio, que constituye el último paso del flujo de trabajo del profesional.

\begin{figure}[H]
  \centering
  \includegraphics[width=\textwidth]{img/profesional_evaluar_ejercicio.png}
  \caption{Evaluación de un ejercicio con el vídeo demostrativo y la grabación del paciente en paralelo.}
  \label{fig:prof-evaluar-ejercicio}
\end{figure}

\subsection{Interfaz del paciente}

El paciente accede a un panel principal muy simplificado, con cuatro accesos directos a ejercicios, sesiones programadas, progreso y ayuda, dispuesto en forma de mandos circulares fáciles de seleccionar.\cite{GithubTerapitrack} Desde esta pantalla puede desplazarse entre las opciones con las flechas del mando SNES y confirmar con los botones de acción, reduciendo la necesidad de movimientos de precisión con el ratón.

La sección \emph{Mis sesiones} muestra un calendario con las próximas semanas y un resumen de la sesión seleccionada, indicando profesional, especialidad y número de ejercicios asignados.\cite{GithubTerapitrack} En \emph{Mis ejercicios}, el paciente visualiza tarjetas con el vídeo de ejemplo, una breve descripción y la duración estimada, pudiendo reproducir cada ejercicio y conocer cuántas veces se le ha asignado.

En \emph{Mi progreso} se presentan indicadores numéricos (evaluaciones totales, mejor nota, media) y una gráfica temporal de las puntuaciones obtenidas, acompañadas de las tarjetas de ejercicios evaluados con los comentarios del profesional.\cite{GithubTerapitrack} Por último, la vista \emph{Ayuda} actúa como guía rápida del mando SNES, explicando qué hace cada botón dentro de la aplicación para que el paciente pueda recordar en cualquier momento cómo desplazarse y seleccionar opciones.

\section{Decisiones de diseño en la interfaz}

La interfaz de TerapiTrack se ha diseñado teniendo en cuenta que muchos de los usuarios finales (pacientes) pueden ser personas mayores con poca experiencia en el uso de aplicaciones web, y que algunos de ellos presentan limitaciones motoras derivadas de la enfermedad de Parkinson \cite{rosen2009_telerehab_technologies,wcag22}.

\subsection{Navegación simplificada}

Cada rol dispone de un menú de navegación adaptado a sus necesidades, con un número reducido de opciones claramente etiquetadas.
Por ejemplo, el menú del paciente incluye únicamente "Mis sesiones", "Ejercicios disponibles" y "Mi perfil", evitando sobrecargar la interfaz con opciones que no va a utilizar.

La plantilla base (\texttt{base.html}) define la estructura común de todas las páginas (cabecera, menú, pie de página) y se encarga de mostrar mensajes de error o confirmación de forma coherente en todo el sistema mediante \emph{flash messages} de Flask.

\subsection{Controles grandes y contraste adecuado}

En las pantallas de ejecución de sesiones se han utilizado botones grandes, con un tamaño de fuente generoso y un espaciado suficiente entre elementos para facilitar la interacción con ratón, teclado o pantalla táctil.
Los colores de fondo y de texto se han elegido siguiendo las recomendaciones de WCAG para asegurar un contraste suficiente, especialmente en mensajes de error (rojo) y de éxito (verde) \cite{wcag22}.

\subsection{Feedback visual claro}

Durante la ejecución de una sesión, el sistema muestra de forma clara en qué ejercicio se encuentra el paciente y cuántos quedan pendientes.
Cuando se graba un vídeo, aparece un indicador de "grabando" que cambia de color, y una vez finalizada la grabación se muestra un mensaje de confirmación antes de pasar al siguiente ejercicio.

Este tipo de feedback ha resultado fundamental durante las pruebas, porque ayuda al paciente a sentir que tiene el control del proceso y reduce la incertidumbre sobre si el sistema está funcionando correctamente.

Además, en el panel del paciente se ha habilitado el uso de un mando SNES como dispositivo de entrada alternativo, de forma que los ejercicios puedan ejecutarse con menos movimientos de precisión y con un patrón de interacción más simple que el ratón tradicional.\cite{GithubTerapitrack,telehealth_accessibility}


\section{Pruebas y validación}

El sistema se ha probado de forma continua durante el desarrollo, combinando pruebas automáticas sobre los modelos y controladores con pruebas manuales sobre los flujos de usuario más complejos.

\subsection{Pruebas unitarias con Pytest}

Se ha implementado un conjunto de pruebas unitarias en la carpeta \texttt{tests}, que cubre las operaciones básicas de los modelos (creación, actualización, borrado y consultas) y varios de los flujos principales de los controladores \cite{PytestDocs}.
Para ejecutar estas pruebas se utiliza el comando \texttt{pytest --cov}, que genera un informe de cobertura por fichero. En la versión final del proyecto se han definido 198 pruebas y se ha alcanzado una cobertura global del 99\%, con todos los modelos y las funciones auxiliares al 100\% y los controladores principales (administrador, profesional, paciente y autenticación) en torno al 99\% de líneas cubiertas \cite{GithubTerapitrack}.
Aunque esta cobertura no garantiza que el código esté libre de errores, sí proporciona una red de seguridad que facilita realizar cambios sin romper funcionalidades ya existentes \cite{PytestDocs}.

\subsection{Pruebas manuales en entornos reales}

Además de las pruebas automáticas, se han realizado pruebas manuales tanto en el entorno de desarrollo como en el despliegue de Heroku, simulando distintos perfiles de usuario (administrador, profesional, paciente) y comprobando el comportamiento del sistema con datos reales y con conexiones a Cloudinary \cite{CloudinaryConsole,GithubTerapitrack}.
En paralelo, se han ejecutado pruebas sobre la base de datos desde \texttt{flask shell} y mediante consultas SQL en DB Browser for SQLite para verificar la creación de entidades, las relaciones de muchos a muchos y operaciones de actualización habituales (cambio de estado de sesiones, modificación de evaluaciones, etc.), que se describen con más detalle en el anexo de documentación técnica \cite{GithubTerapitrack}.
Estas pruebas han permitido detectar problemas que no eran evidentes en el entorno local, como tiempos de respuesta más largos al subir vídeos grandes o errores de permisos en rutas específicas, cuya corrección se ha ido registrando en Jira y consolidando en la rama \texttt{Pruebas} siguiendo el ciclo iterativo descrito en la primera sección de este capítulo \cite{JiraTerapitrack}.

\subsubsection{Análisis estático con SonarQube}

Además de las pruebas automatizadas con \texttt{pytest}, se realizó un análisis estático del código utilizando SonarQube con el objetivo de detectar \emph{code smells}, posibles errores y problemas de mantenibilidad.\cite{GithubTerapitrack} Para ello se configuró un proyecto específico para \textit{TerapiTrack}, se lanzó el escaneo sobre el código de \texttt{src} y se revisaron los informes generados en la interfaz web de la herramienta.

El resultado del análisis indicó que no se detectaban vulnerabilidades de seguridad ni errores críticos, y que la mayor parte de las incidencias correspondían a recomendaciones de estilo (nombres de variables, longitud de funciones o duplicidad leve de código). Estas observaciones se utilizaron para refactorizar algunos controladores y modelos, mejorando la legibilidad y reduciendo el acoplamiento entre módulos, aunque se mantuvieron ciertos \emph{smells} considerados aceptables para el alcance de un prototipo académico.

\begin{figure}[H]
  \centering
  \includegraphics[width=0.9\textwidth]{img/sonarqube_resumen.png}
  \caption{Resumen del análisis estático de \textit{TerapiTrack} en SonarQube.}
  \label{fig:sonarqube-resumen}
\end{figure}

\section{Lecciones aprendidas y aspectos mejorables}

El desarrollo de TerapiTrack ha sido una experiencia de aprendizaje continuo en la que se han ido descubriendo soluciones a medida que surgían los problemas.

Entre los aspectos que han funcionado bien destacan:

\begin{itemize}
  \item La organización del código en blueprints y el uso de decoradores para control de acceso, que han simplificado el mantenimiento y la evolución del sistema.
  \item La decisión de normalizar el modelo de datos y utilizar tablas intermedias para las relaciones de muchos a muchos, que ha facilitado añadir nuevas funcionalidades sin tener que reestructurar la base de datos.
  \item La integración de Cloudinary para el almacenamiento de vídeos, que ha resuelto los problemas de escalabilidad y despliegue que surgían al guardar los archivos en local.
\end{itemize}

Entre los aspectos mejorables se encuentran:

\begin{itemize}
  \item La gestión de ramas en Git podría haber sido más disciplinada desde el principio, realizando merges parciales más frecuentes entre \texttt{Pruebas} y \texttt{main} en lugar de acumular todos los cambios en una sola rama.
  \item La sincronización entre las tareas de Jira y los commits de GitHub no siempre ha sido perfecta, y algunas tareas completadas no tienen un commit asociado claramente identificable.
  \item La cobertura de pruebas podría ampliarse a pruebas de interfaz de usuario y a flujos de interacción entre vistas, aunque esto requeriría configurar un entorno de pruebas más complejo con navegadores automatizados (Selenium, Playwright).
\end{itemize}

A pesar de estas limitaciones, el resultado final es un sistema funcional que cumple los objetivos planteados al inicio del proyecto y que puede servir como base para futuras ampliaciones, como la integración de análisis automático de los vídeos mediante inteligencia artificial o la incorporación de funcionalidades de comunicación en tiempo real entre paciente y profesional.

Los diagramas de arquitectura, los esquemas del modelo de datos y las capturas de las pantallas principales para cada tipo de usuario se incluyen en los apéndices técnicos y en los manuales de usuario, de manera que complementan la descripción general presentada en este capítulo.\cite{GithubTerapitrack}

\capitulo{6}{Trabajos relacionados}

El desarrollo de TerapiTrack se enmarca en un ámbito de investigación activo en el que se han realizado diversos trabajos centrados en la telerehabilitación, el análisis de ejercicios mediante vídeo y la aplicación de técnicas de inteligencia artificial para evaluar el desempeño de pacientes~\cite{cubo2023_telerehab_parkinson}.
Este capítulo presenta un breve resumen de los trabajos más relevantes que han servido como referencia o que comparten objetivos similares con el proyecto actual~\cite{nunez2021_tfg_ejercicios,espinosa2022_tfg_deteccion}.

\section{Trabajos previos del grupo de investigación}

Varios de los tutores y colaboradores del proyecto han participado en trabajos anteriores relacionados con la telerehabilitación y el análisis automático de ejercicios, lo que ha proporcionado una base sólida para el diseño de TerapiTrack~\cite{garrido2021_tfm_infraestructura,ramirez2021_tfm_detectron}.

\subsection{Evaluación de ejercicios de rehabilitación en vídeo}

El Trabajo Fin de Grado de Lucía Núñez Calvo abordó la evaluación automática de ejercicios de rehabilitación a partir de vídeos grabados por los pacientes.
En ese proyecto se exploraron técnicas de extracción de características y comparación de poses para determinar si un paciente estaba realizando correctamente un ejercicio frente a un vídeo de referencia~\cite{nunez2021_tfg_ejercicios}.

TerapiTrack comparte el enfoque de trabajar con vídeos como fuente principal de información, aunque en este caso la evaluación del desempeño la realiza el profesional de forma manual en lugar de aplicar análisis automático~\cite{GithubTerapitrack}.
La experiencia acumulada en ese trabajo ha servido para diseñar la estructura de almacenamiento de vídeos y para prever una posible ampliación futura del sistema que integre técnicas de inteligencia artificial~\cite{nunez2021_tfg_ejercicios}.

\subsection{Detección de ejercicios en vídeos de rehabilitación}

El Trabajo Fin de Grado de Luis Ángel Espinosa Lafuente se centró en la detección automática del tipo de ejercicio que un paciente estaba realizando a partir de la secuencia de poses capturadas en un vídeo~\cite{espinosa2022_tfg_deteccion}.
Este proyecto permitió validar que es posible clasificar ejercicios de forma fiable utilizando modelos entrenados sobre conjuntos de datos etiquetados~\cite{espinosa2022_tfg_deteccion}.

En TerapiTrack cada ejercicio forma parte de una sesión planificada y queda identificado de forma explícita en el modelo de datos, de modo que la aplicación siempre sabe qué ejercicio está realizando el paciente cuando graba su respuesta~\cite{GithubTerapitrack}.
Aunque no se utiliza clasificación automática, el diseño actual facilitaría incorporar en el futuro un módulo de detección que complemente esta identificación y permita asociar métricas adicionales de calidad a cada ejercicio grabado~\cite{espinosa2022_tfg_deteccion}.

\subsection{Infraestructura para telerehabilitación y análisis online}

El Trabajo Fin de Máster de José Luis Garrido Labrador desarrolló una infraestructura completa para gestionar programas de telerehabilitación, incluyendo el almacenamiento de vídeos, la asignación de ejercicios y el seguimiento de la evolución de los pacientes~\cite{garrido2021_tfm_infraestructura}.
Este trabajo sentó las bases arquitectónicas que se han reutilizado parcialmente en TerapiTrack, especialmente en lo relativo a la organización del modelo de datos y la separación de roles~\cite{garrido2021_tfm_infraestructura}.

\subsection{Detección de poses con Detectron2}

El Trabajo Fin de Máster de José Miguel Ramírez Sanz exploró el uso de Detectron2, un \emph{framework} de detección de objetos y poses, para extraer información sobre la posición de las articulaciones de un paciente a partir de vídeos de ejercicios~\cite{ramirez2021_tfm_detectron}.
Aunque TerapiTrack no incorpora análisis automático de poses en su versión actual, los resultados de ese trabajo han servido para entender las limitaciones técnicas y los requisitos de calidad de vídeo que serían necesarios si se decidiera añadir esta funcionalidad en el futuro~\cite{ramirez2021_tfm_detectron}.

\section{Sistemas de telerehabilitación para Parkinson}

Existen en la literatura varios sistemas orientados específicamente a pacientes con enfermedad de Parkinson que combinan telerehabilitación con técnicas de análisis automático~\cite{parkinson_telerehab_review}.

Cubo y colaboradores presentaron un sistema de telerehabilitación basado en técnicas de \emph{deep learning} para evaluar el desempeño de pacientes con Parkinson durante la realización de ejercicios en su domicilio~\cite{cubo2023_telerehab_parkinson}.
El sistema utiliza redes neuronales convolucionales para analizar vídeos y clasificar la calidad de los movimientos, proporcionando \emph{feedback} automático al paciente y al profesional~\cite{cubo2023_telerehab_parkinson}.

TerapiTrack comparte el objetivo de facilitar el seguimiento remoto de pacientes con Parkinson, pero adopta un enfoque más centrado en la gestión del flujo de trabajo clínico (asignación de sesiones, almacenamiento de vídeos, evaluación manual por parte del profesional) que en el análisis automático~\cite{GithubTerapitrack}.
Esta decisión se debe a que la evaluación manual permite al profesional tener en cuenta aspectos cualitativos que son difíciles de capturar mediante algoritmos, y porque la integración de técnicas de inteligencia artificial requiere conjuntos de datos etiquetados y validados que no estaban disponibles al inicio del proyecto~\cite{parkinson_telerehab_protocol}.

\section{Diferencias y aportaciones de TerapiTrack}

Frente a los trabajos relacionados mencionados, TerapiTrack aporta las siguientes características diferenciadoras:

\begin{itemize}
  \item \textbf{Gestión completa del ciclo de trabajo clínico}: TerapiTrack no se centra únicamente en el análisis de vídeos, sino en todo el proceso que va desde la asignación de ejercicios por parte del profesional hasta la evaluación del desempeño del paciente y la consulta del historial de sesiones~\cite{GithubTerapitrack,garrido2021_tfm_infraestructura}.
  \item \textbf{Interfaz accesible y adaptada a los tres roles}: El sistema proporciona vistas específicas para administradores, profesionales y pacientes, con navegación simplificada y criterios de accesibilidad pensados para usuarios con poca experiencia tecnológica o con limitaciones motoras~\cite{rosen2009_telerehab_technologies,wcag22}.
  \item \textbf{Almacenamiento externo de vídeos}: La integración con Cloudinary resuelve los problemas de escalabilidad y despliegue que surgen al almacenar vídeos en el propio servidor, facilitando el uso del sistema en entornos de producción reales~\cite{CloudinaryConsole}.
  \item \textbf{Evaluación cualitativa por parte del profesional}: A diferencia de los sistemas basados en análisis automático, TerapiTrack permite al profesional registrar comentarios y observaciones detalladas sobre cada ejercicio, algo especialmente valioso en el contexto clínico donde el juicio experto sigue siendo fundamental~\cite{parkinson_telerehab_review}.
  \item \textbf{Preparado para integrar IA más adelante}: Aunque la versión actual no realiza análisis automático de poses ni clasificación de ejercicios, el modelo de datos y la forma de almacenar los vídeos están pensados para poder incorporar módulos de visión artificial y \emph{deep learning} en futuras versiones sin rediseñar por completo la aplicación~\cite{nunez2021_tfg_ejercicios,ramirez2021_tfm_detectron,espinosa2022_tfg_deteccion}.
\end{itemize}

Estos aspectos hacen de TerapiTrack un sistema complementario a los trabajos previos, que puede servir como base para futuras ampliaciones que integren técnicas de inteligencia artificial sin renunciar a la supervisión humana del proceso terapéutico~\cite{cubo2023_telerehab_parkinson,GithubTerapitrack}.

\capitulo{7}{Conclusiones y líneas de trabajo futuras}

\section{Conclusiones}

El desarrollo de TerapiTrack ha permitido pasar de una idea inicial de plataforma de telerehabilitación a una herramienta web funcional que gestiona usuarios, ejercicios, sesiones terapéuticas y evaluaciones en un entorno realista.
A lo largo del proyecto se han cubierto los principales objetivos funcionales y técnicos planteados, dejando una base razonablemente sólida sobre la que se pueden apoyar futuras extensiones del sistema.

Desde el punto de vista funcional, la aplicación ofrece paneles diferenciados para administradores, profesionales sanitarios y pacientes, de forma que cada tipo de usuario dispone de un entorno adaptado a sus tareas habituales.
El sistema permite gestionar la vinculación entre pacientes y profesionales, crear y catalogar ejercicios en vídeo, programar sesiones personalizadas y registrar su realización, incluyendo la subida y almacenamiento de las grabaciones generadas en el domicilio del paciente.
Además, se ha incorporado una trazabilidad básica de la evolución terapéutica a través del historial de sesiones y evaluaciones asociadas a cada paciente~\cite{HerokuTerapitrackApp}.

En el ámbito técnico, se ha diseñado una arquitectura modular basada en Flask y \textit{blueprints}, apoyada en un modelo de datos relacional normalizado que separa claramente las entidades de usuario, paciente, profesional, ejercicio, sesión, vídeo de respuesta y evaluación~\cite{SQLAlchemyDocs,FlaskDocs}.
La integración de SQLAlchemy como ORM ha facilitado la evolución del esquema sin perder integridad referencial, mientras que el uso de Cloudinary ha permitido delegar el almacenamiento de vídeos en un servicio específico, evitando problemas de espacio y de ancho de banda en el propio servidor.
El despliegue en Heroku y la configuración diferenciada de los entornos de desarrollo y producción han demostrado que la aplicación puede ejecutarse en la nube con una configuración razonable de recursos~\cite{CloudinaryConsole,HerokuTerapitrackApp,HerokuDashboard}.

Otro aspecto destacable es la incorporación de pruebas automatizadas con \textit{pytest} sobre los módulos más críticos, alcanzando una cobertura de casi el 100\% en modelos, decoradores y componentes de configuración~\cite{PytestDocs}.
Estas pruebas, combinadas con pruebas manuales sobre los flujos de uso principales en el entorno desplegado, han ayudado a detectar errores de diseño e implementación en fases tempranas y han dado una mayor tranquilidad a la hora de refactorizar código o ajustar funcionalidades.
La decisión de seguir una metodología de trabajo iterativa apoyada en Jira también ha sido útil para planificar, seguir el avance real del proyecto y reordenar prioridades cuando ha sido necesario~\cite{JiraTerapitrack}.

Desde una perspectiva más personal, el proyecto ha servido para integrar conocimientos de desarrollo web, bases de datos, control de versiones, despliegue en la nube y documentación técnica en un caso de uso centrado en la mejora de la calidad de vida de personas con enfermedad de Parkinson.

Entre las lecciones aprendidas destacan la utilidad de organizar el código en \textit{blueprints} y emplear decoradores para el control de acceso, la importancia de diseñar un modelo de datos normalizado con tablas intermedias para las relaciones de muchos a muchos y las ventajas de integrar servicios externos como Cloudinary para resolver problemas de escalabilidad en el almacenamiento de vídeos.
El trabajo ha dejado claro, además, lo importante que es mantener cierta disciplina en el uso de herramientas como Git y Jira, así como diseñar pensando en la accesibilidad y en las limitaciones de los usuarios finales desde el principio~\cite{wcag22,telehealth_accessibility}.
Aunque han aparecido dificultades técnicas y de organización en distintos momentos, la experiencia global ha sido muy enriquecedora y ha permitido entregar una primera versión de TerapiTrack que cumple los objetivos planteados y puede utilizarse en un contexto de telerehabilitación supervisada.

\section{Líneas de trabajo futuras}

La versión actual de TerapiTrack cubre el ciclo básico de planificación, realización y evaluación de sesiones terapéuticas, pero deja abiertas varias líneas de mejora que podrían abordarse en trabajos posteriores.
Algunas de estas líneas están relacionadas con la ampliación de la funcionalidad y otras con la incorporación de técnicas de análisis de vídeo más avanzadas o con la adaptación del sistema a un entorno clínico real~\cite{cubo2023_telerehab_parkinson}.
En particular, las mejoras descritas en este apartado forman parte de la evolución del proyecto de telerehabilitación en el que se enmarca TerapiTrack y no están incluidas en el alcance de la implementación realizada en este TFG.

En primer lugar, una evolución natural sería integrar módulos de análisis automático de las grabaciones utilizando técnicas de visión artificial y aprendizaje automático, aprovechando los trabajos previos del grupo en detección de poses y evaluación de ejercicios a partir de vídeo~\cite{nunez2021_tfg_ejercicios,ramirez2021_tfm_detectron,espinosa2022_tfg_deteccion}.
Esto permitiría complementar la evaluación manual del profesional con métricas objetivas sobre la ejecución de los movimientos, generando indicadores de calidad de los ejercicios o alertas cuando se detecten patrones de empeoramiento~\cite{parkinson_telerehab_review,parkinson_telerehab_protocol}.
La integración de estos módulos de procesamiento automático de vídeo se considera parte de la evolución futura del sistema y no se ha abordado en la versión desarrollada en este TFG.

En segundo lugar, sería interesante ampliar las funcionalidades de explotación de datos clínicos y de visualización del progreso del paciente.
Actualmente el sistema ofrece un historial estructurado de sesiones y evaluaciones, pero se podrían incorporar gráficos más completos, filtros por tipo de ejercicio o por profesional y resúmenes periódicos tanto para los profesionales como para los propios pacientes.
Este tipo de información agregada ayudaría a tomar decisiones sobre la adaptación de las terapias y a comunicar de forma más clara los avances o dificultades encontrados durante el tratamiento~\cite{cubo2023_telerehab_parkinson,parkinson_telerehab_review,ninds_parkinson,who_parkinson_2023}.

Además, quedan pendientes algunas mejoras funcionales identificadas durante el desarrollo y registradas en Jira, como la incorporación de notificaciones por correo electrónico (por ejemplo, para recordar sesiones próximas o informar de nuevas evaluaciones) y la posibilidad de exportar las evaluaciones de un paciente a un informe en formato PDF.
Estas tareas no afectan al funcionamiento básico de TerapiTrack, pero su implementación contribuiría a mejorar la comunicación con los usuarios y a facilitar la documentación de los resultados terapéuticos en futuras versiones del sistema.

Otra línea de mejora tiene que ver con la accesibilidad y la experiencia de usuario~\cite{wcag22}.
Aunque la interfaz ya sigue criterios básicos de accesibilidad y se ha pensado para personas mayores con posibles limitaciones motoras, sería recomendable realizar pruebas de usabilidad con pacientes reales y profesionales para identificar barreras concretas y ajustar textos, colores, tamaño de los elementos y flujos de navegación~\cite{rosen2009_telerehab_technologies}.
También podría valorarse la adaptación de la interfaz a dispositivos móviles o tabletas, que en muchos casos resultan más cómodos en el entorno doméstico~\cite{telehealth_accessibility}.

Otra línea de trabajo, ligada al proyecto global de telerehabilitación, sería la incorporación de un sistema de videollamada en tiempo real entre paciente y profesional, de manera que algunas sesiones puedan realizarse de forma síncrona y supervisada~\cite{cubo2023_telerehab_parkinson}.
Este módulo complementaría el modelo actual basado en vídeos grabados y evaluación diferida, pero su diseño e implementación quedan fuera del alcance de este TFG.

Desde el punto de vista de la integración con el ecosistema sanitario, una evolución relevante sería conectar TerapiTrack con otros sistemas clínicos, como historias clínicas electrónicas o plataformas de citación y teleconsulta.
Esta integración permitiría evitar la duplicación de información, mejorar la coordinación entre distintos profesionales y facilitar que TerapiTrack se incorpore como una herramienta más dentro de los procesos asistenciales habituales.
Para ello sería necesario estudiar los requisitos legales y de interoperabilidad, así como los estándares de intercambio de datos empleados en cada entorno~\cite{cubo2023_telerehab_parkinson}.

Por último, cabría explorar la posibilidad de extender el uso de TerapiTrack más allá de la enfermedad de Parkinson, adaptando los catálogos de ejercicios y los criterios de evaluación a otras patologías crónicas que se puedan beneficiar de programas de rehabilitación remota~\cite{rehab_hss,nbcot_ot}.
El diseño modular del sistema, tanto en el modelo de datos como en la interfaz, facilita esta reutilización siempre que se definan de forma adecuada los nuevos perfiles de usuario, los tipos de ejercicio y las necesidades específicas de cada grupo de pacientes~\cite{date_db,pressman_ingenieria_sw}.
De este modo, el trabajo realizado podría servir como base para una plataforma más generalista de telerehabilitación, manteniendo el foco en la calidad, la accesibilidad y la seguridad de la información sanitaria~\cite{cubo2023_telerehab_parkinson}.



\bibliographystyle{plain}
\bibliography{bibliografia}

\end{document}
